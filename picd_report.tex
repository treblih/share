\documentclass[a4paper]{report}
\usepackage{xcolor}
\usepackage{color}
\usepackage{graphicx}
\usepackage{listings}
\usepackage[margin=3cm]{geometry}
\usepackage[colorlinks, linkcolor=blue, anchorcolor=blue, 
            urlcolor=blue]{hyperref}
\title{\textbf{Test Design for Packages of Wind River PICD 1.0}}
\author{Copyright (c) 2010 \textcolor{red}{\textbf{Wind River}} Systems, Inc.}
\date{\today}


\begin{document}
\pagestyle{headings}
\maketitle
\tableofcontents
\lstset{language=c,
        breaklines=true,
        showspaces=false,
        showstringspaces=false,
        showtabs=false,
        tabsize=4,
        backgroundcolor=\color{lightgray},
        basicstyle=\tt,
        directivestyle=\tt,
        identifierstyle=\tt,
        commentstyle=\tt,
        stringstyle=\tt,
        keywordstyle=\color{blue}\tt
}
\hypersetup{
    colorlinks=true, 
    breaklinks=true,
    plainpages=false,
}


% contents start
\chapter{argp-standalone-1.3}
\section{Description}
It is standalone version of argp - part of glibc library. It was separated off glibc, 
primary use it for inclusion in the LSH distribution, but it's useful for any 
package that wants to use argp and at the same time be portable to non-glibc systems. 
The most important is that it no longer builds upon {\tt getopt}; the one or two hairy 
functions of GNU getopt are incorporated with the argp parser. There are longer 
any global variables keeping track of the parser state.
\section{Installation Hierarchy}
\begin{lstlisting}
/usr/lib/libargp.a
\end{lstlisting}
\section{Warning}
\section{Test}
\subsection{use official testsuites \textcolor{green}{[pass]}}
6 testing executable under {\tt testsuites}, 
{\tt ex1 ex3 ex4} are core functionalities, {\tt ex1-test permute-test run-tests}
are wrapper scripts.\\
C programs use {\tt libargp} to parse args, while scripts just call them
with different args:
\begin{lstlisting}
    run-tests -+> ex1
               +> permute-test -> ex3

    ex1-test -> ex1
\end{lstlisting}
{\tt ex4} is called by nobody, however it's almost same as {\tt ex3}, make a patch
to {\tt permute-test} to test {\tt ex4} simultaneously:
\begin{lstlisting}
--- permute-test_orig	2011-03-03 16:23:19.824405001 -0500
+++ permute-test	2011-03-03 16:27:05.194404994 -0500
@@ -2,7 +2,7 @@
 
 # Test the somewhat hairy permuting of arguments.
 
-cat >test.out <<EOF
+cat >test3.out <<EOF
 ARG1 = foo
 ARG2 = bar
 OUTPUT_FILE = -
@@ -10,6 +10,14 @@
 SILENT = no
 EOF
 
+cat >test4.out <<EOF
+ARG1 = foo
+STRINGS = bar
+OUTPUT_FILE = -
+VERBOSE = yes
+SILENT = no
+EOF
+
 die () {
     echo "$@" 1>&2
     exit 1
@@ -18,7 +26,8 @@
 for args in "-v foo bar" \
             "-v -v foo bar" "-v foo -v bar" "-v foo bar -v" \
             "foo -v bar -v" "foo bar -v -v" "foo -v -v bar" ; do
-  ./ex3 $args | diff - test.out  || die "Test failed with args $args"
+  ./ex3 $args | diff - test3.out >/dev/null || die "Test failed with args $args, ex3"
+  ./ex4 $args | diff - test4.out >/dev/null || die "Test failed with args $args, ex4"
 done
 
 exit 0
\end{lstlisting}
Start to test:
\begin{lstlisting}
root@localhost:/root> ./ex1-test 
root@localhost:/root> ./run-tests 
PASS: ex1
PASS: permute
==================
All 2 tests passed
==================
\end{lstlisting}
Checkpoints:
\begin{itemize}
    \item for {\tt ex1-test}, no output is right;
    \item {\tt run-tests} shows the accurate results;
\end{itemize}
\subsection{use official {\tt argp-test} \textcolor{green}{[pass]}}
An executable isn't included in testsuite, try it separately.\\
\subsubsection{help info}
\begin{lstlisting}
root@localhost:/root> ./argp-test --help
Usage: argp-test [OPTION...] STRING STRING...
  or:  argp-test [OPTION...] STRING -
Test program for argp.

  -p, --pid=PID              List the process PID
      --pgrp=PGRP            List processes in the process group PGRP
  -P, -x, --no-parent        Include processes without parents
  -Q, --all-fields           Don't elide unusable fields (normally if there's
                             some reason ps can't print a field for any
                             process, it's removed from the output entirely)
  -r, --reverse, --gratuitously-long-reverse-option
                             Reverse the order of any sort
      --session[=SID]        Add the processes from the session SID (which
                             defaults to the sid of the current process)
  -s, --subopt1              Nested option 1
  -S, --subopt2              Nested option 2

 Here are some more options:
  -f, --foonly[=ZOT]         Glork a foonly (ZOT defaults to 6b8b4567)
  -q, --subopt4              Nested option 4
  -z, --zaza                 Snit a zar

 Some more nested options:
      --subopt3              Nested option 3

  -?, --help                 Give this help list
      --usage                Give a short usage message
  -V, --version              Print program version

Mandatory or optional arguments to long options are also mandatory or optional
for any corresponding short options.

This doc string comes after the options.
Hey!  Some manual formatting!
The current time is: Thu Jan  1 02:21:25 1970

This is the doc string from the sub-arg-parser.

This is some extra text from the sub parser (note that it is preceded by a
blank line).
\end{lstlisting}\null\\
\subsubsection{with no arg}
\begin{lstlisting}
root@localhost:/root> ./argp-test 
NO ARGS
NO SUB ARGS
After parsing: foonly = 0
\end{lstlisting}
Checkpoints:
\begin{itemize}
    \item {\tt NO ARGS};
    \item {\tt NO SUB ARGS};
    \item {\tt foonly} is zero, because didn't specify it;
\end{itemize}
\subsubsection{nested options}
\begin{lstlisting}
root@localhost:/root> ./argp-test  -s a -S b --subopt3 c --subopt4 d -s e
SUB KEY s
SUB KEY S
SUB KEY p
SUB KEY q
SUB KEY s
ARG: a
SUB ARG: b
SUB ARG: c
SUB ARG: d
SUB ARG: e
After parsing: foonly = 0
\end{lstlisting}
Output should be the same. It's right for {\tt --subopt3} with {\tt p} and 
{\tt --subopt4} with {\tt q}, for the sake of:
\begin{lstlisting}
struct argp_option sub_options[] =
{
  {"subopt1",       's',     0,  0, "Nested option 1", 0},
  {"subopt2",       'S',     0,  0, "Nested option 2", 0},
  { 0, 0, 0, 0, "Some more nested options:", 10},
  {"subopt3",       'p',     0,  0, "Nested option 3", 0},
  {"subopt4",       'q',     0,  0, "Nested option 4", 1},
  {0, 0, 0, 0, 0, 0}
};
\end{lstlisting}
\subsubsection{default variable set \textcolor{green}{[pass]}}
By default {\tt foonly} is 0x6b8b4567, if just type {\tt -f}:
\begin{lstlisting}
root@localhost:/root> ./argp-test -f
KEY f
NO ARGS
NO SUB ARGS
After parsing: foonly = 6b8b4567
\end{lstlisting}
Checkpoints:
\begin{itemize}
    \item {\tt KEY} is {\tt f};
    \item {\tt foonly} is 6b8b4567;
\end{itemize}
Set a new value, notice that it receives a decimal, outputs a hex without leading 
{\tt 0x}:
\begin{lstlisting}
root@localhost:/root> ./argp-test -f100
KEY f: 100
NO ARGS
NO SUB ARGS
After parsing: foonly = 64
\end{lstlisting}
Checkpoints:
\begin{itemize}
    \item hex number is what in decimal;
\end{itemize}
%\chapter{argp-standalone-1.3}


\chapter{appweb-3.2.3}
\section{Description}
It is a compact embedded HTTP web server. It is a fast, small-footprint, multi-threaded, 
standards-based, portable server developed for use by embedded devices  
and applications.  It can run as a stand-alone web server or the {\tt appweb} 
library can be embedded in applications.\\\\
{\tt appweb} supports HTTP/1.1, SSL, digest and basic authentication, virtual hosting, 
ranged requests, chunked transfers, file upload and security limits.  {\tt appweb} has an 
{\tt apache} style configuration file and supports the {\tt php} and {\tt ejscript} 
web frameworks.\\\\
{\tt appweb} normally reads the {\tt appweb.conf} file for configuration directives. 
However, if {\tt appweb} is invoked with an IP address or port number on the command line, 
{\tt appweb} will not read the configuration file. Rather it will listen  for  requests  
on  the  specified IP:PORT address. If the PORT component is omitted, {\tt appweb} will 
listen on port 80. If the IP address is omitted and a port is supplied, {\tt appweb} 
will listen on all network interfaces.
\section{Installation Hierarchy}
\begin{lstlisting}

\end{lstlisting}
\section{Warning}
\begin{itemize}
    \item {\tt ./configure --disable-shared --enable-static} to build it fully static;
    \item {\tt --threads numThreads} specify the maximum number of threads to use for 
          the appweb thread pool. This overrides the {\tt ThreadLimit} configuration file
          directive;
    \item Is configured by default to run {\tt CGI} programs;
    \item {\tt appweb 2} uses short options while {\tt appweb 3} uses long options;\\
          For version 2:
\begin{lstlisting}
usage: appweb [-Abdkmv] [-a IP:PORT] [-d docRoot] [-f configFile]
       [-l logSpec] [-r serverRootDir] [-t numThreads]

Options:
  -a IP:PORT     Address to listen on
  -A             Auto-scan for a free port
  -b             Run in background as a daemon
  -d docRoot     Web directory (DocumentRoot)
  -f configFile  Configuration file name
  -k             Kill existing running http
  -l file:level  Log to file at verbosity level
  -r serverRoot  Alternate Home directory

Debug options
  -D             Debug mode (no timeouts)
  -m             Output memory stats
  -t number      Use number of pool threads
  -v             Output version information

Windows options
  -i             Install service
  -g             Go (start) service
  -s             Stop service
  -u             Uninstall service
\end{lstlisting}
          For version 3:
\begin{lstlisting}
appweb [options]
appweb [options] [IPaddress][:port] [documentRoot] 

Options:
--config configFile    # Use named config file instead appweb.conf
--chroot directory     # Change root directory to run more securely (Unix)
--debug                # Run in debug mode
--ejs appSpec          # Create an ejs application at the path
--home directory       # Change to directory to run
--name uniqueName      # Unique name for this instance
--log logFile:level    # Log to file file at verbosity level
--threads maxThreads   # Set maximum worker threads
--version              # Output version information
\end{lstlisting}
    \item Base config file is {\tt /etc/appweb/appweb.conf}. In turn it includes:
    \begin{itemize}
        \item {\tt /etc/appweb/conf/log.conf}
        \item {\tt /etc/appweb/conf/doc.conf}
        \item {\tt /etc/appweb/conf/hosts/*}
        \item {\tt /etc/appweb/conf/apps/*}
        \item {\tt /etc/appweb/conf/tune.conf}
    \end{itemize}
    \item Config file is parsed in a single top-to-bottom pass, the order of 
          directives is important. E.g. define {\tt ServerRoot} before 
          using {\tt LoadModule} directive.
    \item Shouldn't use {\tt basic authentication} if at all possible. 
          Use {\tt digest authentication} instead if it is supported by the clients;
    \item Use {\tt /etc/appweb/php.ini} instead of default {\tt /etc/php/php.ini} 
          of {\tt php};
    \item {\tt git clone http://github.com/embedthis/packages}, download 
          supporting packages, containing {\tt openssl}, {\tt matrixssl}, 
          {\tt php5};
    \item For {\tt openssl}, issue {.config && make && make build-shared};
    \item {\tt --cache-file=ldat\_config\_cache} automatically added by 
          {\tt ldat} leads a configuration error. Delete it in 
          {\tt scripts/packages.mk:1274: 
          --cache-file=ldat\_config\_cache \$(\$(1)\_CONFIG\_OPT)}
\end{itemize}
\section{Test}
\subsection{official {\tt make test}}
Run {\tt make test} after {\tt make}:
\begin{lstlisting}
% make test
    [Notice] Building Embedthis Appweb 3.2.3. For command details, 
             use "make TRACE=1".
      [Test] Starting tests. Test depth: 1, iterations: 1
      [Test] Init-test "test.init": PASSED
      [Test] Run test "regress.01000-chunk": PASSED
      [Test] Run test "basic.put": PASSED
      [Test] Run test "basic.cgi": PASSED
      [Test] Run test "basic.ejs": PASSED
      [Test] Run test "basic.egi": PASSED
      [Test] Run test "basic.read": PASSED
      [Test] Run test "basic.reuse": PASSED
      [Test] Run test "basic.get": PASSED
      [Test] Run test "basic.vhost": PASSED
      [Test] Run test "basic.auth": PASSED
      [Test] Run test "basic.secure": SKIPPED (SSL not enabled)
      [Test] Run test "basic.redirect": PASSED
      [Test] Run test "basic.php": PASSED
      [Test] Run test "basic.misc": PASSED
      [Test] Run test "basic.dir": PASSED
      [Test] Run test "basic.upload": PASSED
      [Test] Run test "basic.alias": PASSED
      [Test] Run test "basic.header": PASSED
      [Test] Run test "basic.post": PASSED
      [Test] Run test "basic.stream": PASSED
      [Test] Run test "basic.query": PASSED
      [Test] Run test "basic.methods": PASSED
      [Test] Run test "basic.chunk": PASSED
      [Test] Run test "basic.callback": PASSED
      [Test] Run test "api.c": PASSED
      [Test] Run test "api.thread": SKIPPED (Run if multithreaded)
      [Test] Run test "api.load": SKIPPED (Runs at depth 3)
      [Test] Run test "api.valgrind": SKIPPED (Run on Linux)
      [Test] Pre-test "stress.pre": SKIPPED 
             (Stress tests run at depth 2 or greater)
      [Test] Run test "auth.digest": PASSED
      [Test] Run test "auth.basic": PASSED
      [Test] Pre-test "http.pre": SKIPPED (Requires http client)
      [Test] Run test "cmd.http": SKIPPED 
             (Test runs at depth 2 or higher with Http client enabled)
      [Test] Pre-test "limits.pre": PASSED
  [Progress] Completed iteration 0
      [Test] Term-test "test.term": PASSED
      [Test] PASSED: 263 tests completed, 0 tests(s) failed, 7 tests(s) skipped. 
             Elapsed time 11.05 secs.
    [Notice] Operation complete.
\end{lstlisting}
Check the status and results.
\subsection{HTTP response \textcolor{green}{[pass]}}
On the board start the daemon.\\
{\tt ps} to check the processes details, it should be an {\tt angel} and an {\tt appweb}
processes:
\begin{lstlisting}
root@localhost:/root> /etc/init.d/appweb start
Starting Embedthis Appweb: [  OK  ]
root@localhost:/root> ps aux |grep appweb
Unknown HZ value! (55) Assume 100.
root   918  0.0  0.1   2772   432 ?     Ss 06:54 0:00 /usr/lib/appweb/bin/angel --daemon --home /etc/appweb /usr/lib/appweb/bin/appweb
nobody 920  1.9  2.1  15072  5336 ?     Ss 06:54 0:00 appweb
root   923  0.0  0.2   2004   536 pts/0 S+ 06:54 0:00 grep appweb
\end{lstlisting}
{\tt angel} is a guardian angel for another program. It starts the designated 
program and then watches over it to restart it should it fail, 
to ensure the service remains available.
\begin{itemize}
    \item[{\tt --daemon}] Run the {\tt angel} as a daemon process. 
          This causes the {\tt angel} to detach from the current shell and run in the background;
    \item[{\tt --home}] Set the home directory in which to start the service.
\end{itemize}
{\tt netstat} to check whether whether the port 7777 ({\tt appweb} default) is listening:
\begin{lstlisting}
root@localhost:/root> netstat -atunp | egrep "(appweb|State)"
Proto Recv-Q Send-Q Local Address Foreign Address State  PID/Program
tcp        0      0 0.0.0.0:7777  0.0.0.0:*       LISTEN 920/appweb
\end{lstlisting}
Then on our desktops open a web-browser to send HTTP requests.\\
The default primary server host is {\tt /var/www/appweb-default}, and 
test dir is {\tt /var/www/appweb-default/test}.\\\\
\subsubsection{{\tt html} \textcolor{green}{[pass]}}
\\[\intextsep]
\begin{minipage}{\textwidth}
\centering
\includegraphics[scale=.45]{appweb/index.png}
\end{minipage}
\\[\intextsep]
Board IP is 128.224.165.254, port 7777.\\
Click the link, get to the main page of doc:
\\[\intextsep]
\begin{minipage}{\textwidth}
\centering
\includegraphics[scale=.45]{appweb/doc.png}
\end{minipage}
\\[\intextsep]
Doc dir is {\tt /usr/lib/appweb/doc}
\subsubsection{{\tt CGI} \textcolor{green}{[pass]}}
{\tt http://128.224.165.254:7777/test/test.cgi}:
\\[\intextsep]
\begin{minipage}{\textwidth}
\centering
\includegraphics[scale=.50]{appweb/cgi.png}
\end{minipage}
\\[\intextsep]
It's written in {\tt bash}.
\subsubsection{{\tt perl} \textcolor{green}{[pass]}}
{\tt http://128.224.165.254:7777/test/test.pl}:
\\[\intextsep]
\begin{minipage}{\textwidth}
\centering
\includegraphics[scale=.50]{appweb/pl.png}
\end{minipage}
\\[\intextsep]
\subsubsection{{\tt python} \textcolor{green}{[pass]}}
{\tt http://128.224.165.254:7777/test/test.py}:
\\[\intextsep]
\begin{minipage}{\textwidth}
\centering
\includegraphics[scale=.50]{appweb/py.png}
\end{minipage}
\\[\intextsep]
\subsubsection{{\tt php} \textcolor{green}{[pass]}}
Add {\tt --with-php=/path/to/phpdir} when config to make it with {\tt php} support.
It just need and add {\tt libphp5.so} to {\tt /usr/lib/appweb/lib/}. Nowadays {\tt php}
has been in our repository, and default installed directly is {\tt /usr/lib/libphp5.so}
{\tt http://128.224.165.254:7777/test/test.php}:
\\[\intextsep]
\begin{minipage}{\textwidth}
\centering
\includegraphics[scale=.50]{appweb/php.png}
\end{minipage}
\\[\intextsep]
Then change the source to:
\begin{lstlisting}
<html>
<head>
    <title> TEST PHP INFO
    </title>
</head>
<body>
    <?php phpinfo(); ?>
</body>
</html>
\end{lstlisting}
And try again, if it shows the {\tt php} config details, means {\tt php} has been set correctly:
\\[\intextsep]
\begin{minipage}{\textwidth}
\left
\includegraphics[scale=.55]{appweb/phpinfo.png}
\end{minipage}
\\[\intextsep]
Checkpoints:
\begin{itemize}
    \item Compare field {\tt System} shown in the web page with {\tt uname -a};
    \item The field {\tt Loaded Configuration File} shows where {\tt php.ini} 
          is. It's a part of {\tt appweb};
\end{itemize}
\subsubsection{{\tt ejs} \textcolor{green}{[pass]}}
{\tt http://128.224.165.254:7777/test/test.ejs}:
\\[\intextsep]
\begin{minipage}{\textwidth}
\centering
\includegraphics[scale=.50]{appweb/ejs.png}
\end{minipage}
\\[\intextsep]
It said that maybe {\tt /var/www/appweb-default/test} isn't writable.\\
Change it's mode from default 544 to 744, and try again:
\\[\intextsep]
\begin{minipage}{\textwidth}
\centering
\includegraphics[scale=.50]{appweb/ejs_ok.png}
\end{minipage}
\\[\intextsep]
Checkpoint:
\begin{itemize}
    \item {\tt server Time} is correct, use {\tt date} to check;
\end{itemize}
\subsection{Check access and error log \textcolor{green}{[pass]}}
The default log dir is {\tt /var/log/appweb}, {\tt access.log} and {\tt error.log}
are there.
\begin{lstlisting}
root@localhost:/root> cat /var/log/appweb/access.log | head -8
128.224.158.134 - - [Thu Jan 01 05:37:23 1970 GMT] "GET /index.html HTTP/1.1" 304 273
128.224.158.134 - - [Thu Jan 01 05:37:23 1970 GMT] "GET /screen.css HTTP/1.1" 304 273
128.224.158.134 - - [Thu Jan 01 05:37:23 1970 GMT] "GET /images/shadow.jpg HTTP/1.1" 304 274
128.224.158.134 - - [Thu Jan 01 05:37:23 1970 GMT] "GET /images/banner.jpg HTTP/1.1" 304 276
128.224.158.134 - - [Thu Jan 01 05:37:23 1970 GMT] "GET /images/bottomShadow.jpg HTTP/1.1" 304 275
128.224.158.134 - - [Thu Jan 01 05:37:23 1970 GMT] "GET /favicon.ico HTTP/1.1" 304 276
128.224.158.134 - - [Thu Jan 01 05:38:03 1970 GMT] "GET /doc/index.html HTTP/1.1" 304 273
128.224.158.134 - - [Thu Jan 01 05:38:03 1970 GMT] "GET /doc/product/index.html HTTP/1.1" 304 274
\end{lstlisting}
Make sure the contents are corresponding to your sent requests. Here, 128.224.158.134
is my laptop's IP, did a lot of {\tt GET} requests. Scanning web pages sends {\tt GET}
requests.\\\\

{\tt error.log} does more than error record, in fact all {\tt stdout} and {\tt stderr}
go into it:
\begin{lstlisting}
root@localhost:/root> cat /var/log/appweb/error.log           
appweb: 1: Started at Mon Feb 14 08:43:15 2011 GMT
appweb: 1: Starting host named: "127.0.0.1:7777"
appweb: 1: HTTP services are ready with max 10 worker threads
\end{lstlisting}
Compare ``with max 10 worker threads'' with the setting in {\tt /etc/appweb/conf/tune.conf}
\begin{lstlisting}
#
#   Maximum number of threads if built multi-threaded. Set to 0 for single-threaded
#   
#
ThreadLimit 10
\end{lstlisting}
It's matched here.
\subsection{{\tt POST} request \textcolor{green}{[pass]}}
Create two {\tt php}
file under {\tt /var/www/appweb-default/test/}:
\begin{lstlisting}
root@localhost:/root> cat /var/www/appweb-default/test/form.php 
<html>
<body>

<form action="welcome.php" method="post">
Name: <input type="text" name="name" />
Age: <input type="text" name="age" />
<input type="submit" />
</form>

</body>
</html>
root@localhost:/root> 
root@localhost:/root> 
root@localhost:/root> cat /var/www/appweb-default/test/welcome.php 
<html>
<body>

Welcome <?php echo $_POST["name"]; ?>.<br />
You are <?php echo $_POST["age"]; ?> years old.

</body>
</html>

\end{lstlisting}
Then browse {\tt http://128.224.165.254:7777/test/form.php}, type {\tt windriver}
as name and {\tt 30} as age:
\\[\intextsep]
\begin{minipage}{\textwidth}
\centering
\includegraphics[scale=.50]{appweb/php_form.png}
\end{minipage}
\\[\intextsep]
\\[\intextsep]
\begin{minipage}{\textwidth}
\centering
\includegraphics[scale=.50]{appweb/php_welcome.png}
\end{minipage}
\\[\intextsep]
Meanwhile check {\tt access.log}:
\begin{lstlisting}
root@localhost:/root> cat /var/log/appweb/access.log | egrep "(form|welcome)"
128.224.158.134 - - [Thu Jan 01 05:07:20 1970 GMT] "GET /test/form.php HTTP/1.1" 200 543
128.224.158.134 - - [Thu Jan 01 05:08:27 1970 GMT] "POST /test/welcome.php HTTP/1.1" 200 437
\end{lstlisting}
Checkpoints:
\begin{itemize}
    \item {form.php} with {\tt GET} request;
    \item {welcome.php} with {\tt POST} request;
\end{itemize}
\subsection{Secure Sockets Layer (SSL) support}
Add {\tt --with-openssl=/path/to} when config.
It just need and add {\tt libssl.so} to {\tt /usr/lib/appweb/lib/}.\\
Notice the default port is 4443, if all ok, browse {https://board's IP:4443}:
\\[\intextsep]
\begin{minipage}{\textwidth}
\centering
\includegraphics[scale=.45]{appweb/ssl.png}
\end{minipage}
\\[\intextsep]
Since we don't register and have no certificate, it's ok for {\tt google chrome} to show
red icons in address bar.
\subsection{IPv6}
It supports IPv6 by default.\\
Relate to \ref{IPv6-appweb}.
%\chapter{appweb}


\chapter{bridge-utils}
\section{Description}
Bridge-utils is a tool for configuring the Linux 2.6 bridge. 
This package contains utilities for configuring the Linux ethernet bridge. 
The Linux ethernet bridge can be used for connecting multiple ethernet devices together. 
The connecting is fully transparent: hosts connected to one ethernet device 
see hosts connected to the other ethernet devices directly.
\section{Installation Hierarchy}
\begin{lstlisting}
etc/conf.d/bridges
usr/sbin/brctl
usr/include/libbridge.h
usr/lib/libbridge.a
\end{lstlisting}
\section{Warning}
\begin{itemize}
    \item Use {\tt autoconf} under the source directory to generate configure file;
    \item The only one utility is {\tt brctl}, which is used to set up, maintain, 
          and inspect the ethernet bridge configuration in the linux kernel;
    \item There's a test directory which contains several test cases written 
          in shell scripts.
\end{itemize}
\section{Test}
\subsection{Help info \textcolor{green}{[pass]}}
\begin{lstlisting}
root@localhost:/root> brctl
Usage: brctl [commands]
commands:
	addbr     	<bridge>		add bridge
	delbr     	<bridge>		delete bridge
	addif     	<bridge> <device>	add interface to bridge
	delif     	<bridge> <device>	delete interface from bridge
	setageing 	<bridge> <time>		set ageing time
	setbridgeprio	<bridge> <prio>		set bridge priority
	setfd     	<bridge> <time>		set bridge forward delay
	sethello  	<bridge> <time>		set hello time
	setmaxage 	<bridge> <time>		set max message age
	setpathcost	<bridge> <port> <cost>	set path cost
	setportprio	<bridge> <port> <prio>	set port priority
	show      				show a list of bridges
	showmacs  	<bridge>		show a list of mac addrs
	showstp   	<bridge>		show bridge stp info
	stp       	<bridge> {on|off}	turn stp on/off
\end{lstlisting}\null\\
\subsection{Bridge modules load automatically \textcolor{green}{[pass]}}
\begin{lstlisting}
root@localhost:/root> lsmod | grep br
root@localhost:/root> brctl addbr br0
Bridge firewalling registered
root@localhost:/root> lsmod | grep br
llc                     3334  2 bridge,stp
stp                     1282  1 bridge
bridge                 59852  0 
\end{lstlisting}\null\\
\subsection{Enable stp (spanning tree protocol) \textcolor{green}{[pass]}}
\begin{lstlisting}
root@localhost:/root> brctl show
bridge name	bridge id		STP enabled	interfaces
br0		8000.000000000000	no		
root@localhost:/root> brctl stp br0 on
root@localhost:/root> brctl show
bridge name	bridge id		STP enabled	interfaces
br0		8000.000000000000	yes	
root@localhost:/root> brctl showstp br0
br0
 bridge id		8000.000000000000
 designated root	8000.000000000000
 root port		   0			path cost		   0
 max age		  20.00			bridge max age		  20.00
 hello time		   2.00			bridge hello time	   2.00
 forward delay		  15.00			bridge forward delay	  15.00
 ageing time		 300.00
 hello timer		   0.00			tcn timer		   0.00
 topology change timer	   0.00			gc timer		   0.00
 flags	
\end{lstlisting}\null\\
\subsection{Add interface eth0 to br0 \textcolor{red}{[bug]}}
The system comes to a halt: 
\begin{lstlisting}
root@localhost:/root> brctl addif br0 eth0
root@localhost:/root>
\end{lstlisting}\null\\
\subsection{Use test cases \textcolor{red}{[bug]}}
There're 6 test cases following the bridge-utils' source:
\begin{lstlisting}
root@localhost:/root/bridge-utils> ls tests
busybr  functest  mkbr  README  rmbr  showme  stresstest
\end{lstlisting}\null\\
The contents of {\tt mkbr}:
\begin{lstlisting}
#! /bin/sh

BR=${1:-"br549"}
ETH=${2:-"eth0"}

# fetch ip of working eth0
IP=`/sbin/ifconfig $ETH | sed -n -e 's/^.*inet addr:\([0-9][0-9\.]*\).*$/\1/p'`
echo "Using IP address $IP"

ifconfig $ETH 0.0.0.0
brctl addbr $BR
brctl addif $BR $ETH
ifconfig $BR $IP
\end{lstlisting}\null\\
Running:
\begin{lstlisting}
root@localhost:/root/bridge-utils/tests> ./mkbr
Using IP address 128.224.165.247
\end{lstlisting}
No response, we know it got stuck in {\tt brctl addif br549 eth0}.\\\\
It ought to be at least 2 ethX cards to run {\tt busybr} and {\tt stresstest}.
But we only have one:
\begin{lstlisting}
root@localhost:/root> ifconfig
eth0      Link encap:Ethernet  HWaddr 00:ED:CD:EF:AA:CC  
          inet addr:128.224.165.247  Bcast:128.224.165.255  Mask:255.255.255.0
          inet6 addr: fe80::2ed:cdff:feef:aacc/64 Scope:Link
          UP BROADCAST RUNNING MULTICAST  MTU:1500  Metric:1
          RX packets:26649 errors:0 dropped:0 overruns:0 frame:0
          TX packets:10638 errors:0 dropped:0 overruns:0 carrier:0
          collisions:0 txqueuelen:1000 
          RX bytes:17259768 (16.4 MiB)  TX bytes:1865246 (1.7 MiB)
          Interrupt:23 

lo        Link encap:Local Loopback  
          inet addr:127.0.0.1  Mask:255.0.0.0
          inet6 addr: ::1/128 Scope:Host
          UP LOOPBACK RUNNING  MTU:16436  Metric:1
          RX packets:8 errors:0 dropped:0 overruns:0 frame:0
          TX packets:8 errors:0 dropped:0 overruns:0 carrier:0
          collisions:0 txqueuelen:0 
          RX bytes:560 (560.0 b)  TX bytes:560 (560.0 b)
\end{lstlisting}\null\\
I try to test it on my laptop while use eth0-wlan0 instead of 2 eth0:
\begin{lstlisting}
% sudo brctl addif br0 wlan0 
can't add wlan0 to bridge br0 Operation not supported
\end{lstlisting}
The explanation of the issue, from it's FAQ:\\\\
``This is a known problem, and it is not caused by the bridge code. 
Many wireless cards don't allow spoofing of the source address. 
It is a firmware restriction with some chipsets. You might find some information 
in the bridge mailing list archives to help. Has anyone found a way to get around 
Wavelan not allowing anything but its own MAC address?
(answer by Michael Renzmann (mrenzmann at compulan.de))\\\\
Well, for 99\% of computer users there will never be a way to get rid of this. 
For this function a special firmware is needed. This firmware can be loaded into 
the RAM of any WaveLAN card, so it could do its job with bridging. But there is no 
documentation on the interface available to the public. The only way to achieve this is 
to have a full version of the hcf library which controls every function of the card and 
also allows accessing the card's RAM. To get this full version Lucent wants to know that 
it will be a financial win for them, also you have to sign an NDA. So be sure that you 
won't most probably get access to this peace of software until Lucent does not change 
its mind in this (which I doubt never will happen).\\\\
If you urgently need to have a wireless LAN card which is able to bridge, you should 
use one of those having the prism chipset onboard (manufactured by Harris Intersil). 
There are drivers for those cards available at www.linux-wlan.com 
(which is the website from Absoval), and I found a mail that says that there is the 
necessary firmware and an upload tool available for Linux to the public. 
If you need additional features of an access point you should also talk to Absoval.''\\\\\\
It's also impossible to get a pseudo eth like 
{\tt eth0:0}, because {\tt eth0:0} and {\tt eth0} have the same MAC meanwhile different IP.
%\chapter{bridge-utils}


\chapter{cyclesoak}
\section{Description}
cyclesoak is a tool for measuring system resource utilisation. 
It uses a "subtractive" algorithm: it measures how much system capacity is still available, 
rather than how much is consumed. This makes it much more accurate than tools such as top, 
which simply attempt to add up the CPU usage of separate user processes.
\section{Installation Hierarchy}
\begin{lstlisting}
/usr/bin/cyclesoak
\end{lstlisting}
\section{Warning}
\begin{itemize}
    \item {\tt cyclesoak} is a utility contained in {\tt zc} package, if only need this, 
          {\tt make cyclesoak}, it's only compile {\tt cyclesoak.c} and {\tt zclib.c};
    \item Regards multicore as mutli-CPU;
    \item The basic is doing something in an infinite loop as the frequency. The so called
          calibration is just do that then record (in a created file {\tt count\_per\_sec}), 
          while calculation is do that again then compare to the previous record;
    \item Calibration is inaccurate, small negative number is accepted;
    \item Must compile with {\tt -O0}, namely no optimization, otherwise we'll get
          huge calibrating result like:
\begin{lstlisting}
calibrating: 1782227513 loops/sec
calibrating: 1797358188 loops/sec
calibrating: 1799841119 loops/sec
calibrating: 1805773112 loops/sec
calibrated OK.  1805773112 loops/sec
\end{lstlisting}
          and during statistics:
\begin{lstlisting}
System load:  0.5% || Free: 104.4%(0)  99.5%(1) 100.9%(2)  99.4%(3)
System load:  1.0% || Free: 104.4%(0)  98.1%(1)  99.3%(2)  98.6%(3)
System load:  0.5% || Free: 104.5%(0)  99.5%(1) 100.9%(2)  99.3%(3)
System load:  0.8% || Free: 102.4%(0)  99.4%(1)  99.6%(2)  99.3%(3)
System load:  0.5% || Free: 101.0%(0) 101.8%(1)  99.7%(2)  99.5%(3)
System load:  0.5% || Free: 101.1%(0)  99.4%(1)  97.1%(2)  99.4%(3)
System load:  0.6% || Free:  99.3%(0)  99.4%(1)  96.9%(2)  99.3%(3)
System load:  0.6% || Free: 102.7%(0)  99.3%(1) 101.0%(2)  99.4%(3)
System load:  0.6% || Free: 102.8%(0)  99.5%(1) 100.9%(2)  99.5%(3)
System load:  0.6% || Free: 102.7%(0)  99.3%(1) 100.8%(2)  99.2%(3)
System load:  0.7% || Free: 102.6%(0)  99.2%(1) 101.5%(2)  99.3%(3)
System load:  1.0% || Free: 102.8%(0)  99.2%(1)  99.4%(2)  98.7%(3)
System load:-1022565063.7% || Free: 8180520617.2%(0)  99.5%(1) 100.9%(2)  99.5%(3)
System load:-7150797494.0% || Free: 102.7%(0) 8172340093.1%(1) 8172340094.7%(2) 8172340093.1%(3)
System load:  0.6% || Free: 102.6%(0)  99.4%(1) 100.8%(2)  99.2%(3)
System load:  0.6% || Free: 102.7%(0)  99.4%(1) 100.9%(2)  99.5%(3)
System load:  0.7% || Free: 102.6%(0)  99.4%(1) 100.8%(2)  99.4%(3)
System load:  0.7% || Free: 102.5%(0)  99.2%(1) 101.1%(2)  99.2%(3)
System load:  0.8% || Free: 102.5%(0)  99.2%(1) 101.4%(2)  99.1%(3)
System load:  0.8% || Free: 101.8%(0)  99.1%(1)  98.9%(2)  99.2%(3)
System load:  0.8% || Free:  94.1%(0)  99.2%(1) 100.7%(2)  99.3%(3)
System load:  0.7% || Free:  93.9%(0)  99.4%(1) 100.8%(2)  99.4%(3)
System load:  2.4% || Free:  95.5%(0) 103.8%(1)  95.7%(2)  94.5%(3)
System load:  0.6% || Free:  99.1%(0) 103.2%(1) 100.9%(2)  99.3%(3)
System load:  0.7% || Free: 100.1%(0) 102.8%(1) 101.1%(2)  99.3%(3)
System load:  1.4% || Free: 100.8%(0) 107.8%(1) 100.2%(2)  98.8%(3)
System load:  1.6% || Free:  99.8%(0) 108.9%(1) 100.1%(2)  98.5%(3)
System load:  0.6% || Free: 100.1%(0) 103.2%(1) 100.9%(2)  99.3%(3)
System load:-2043084996.0% || Free: 8172340092.8%(0) 8172340107.0%(1)  98.2%(2)  96.8%(3)
System load:-5112825321.0% || Free: 100.0%(0) 103.2%(1) 8180520615.3%(2) 8180520614.0%(3)
System load:-1021542497.1% || Free: 100.3%(0) 103.5%(1)  98.0%(2)  95.2%(3)
System load:  0.6% || Free:  99.9%(0)  98.2%(1)  95.6%(2)  99.3%(3)
\end{lstlisting}
    \item The original has several bugs, all tests here are based on patched one;
\end{itemize}
\section{Test}
\subsection{Help info \textcolor{green}{[pass]}}
\begin{lstlisting}
root@localhost:/root> cyclesoak -h
Usage: cyclesoak [-BCdh] [-N nr_cpus] [-p period]

  -B:      Generate per-CPU statistics
  -C:      Calibrate CPU load
  -d:      Debug (more d's, more fun)
  -h:      This message
  -N:      Tell cyclesoak how many CPUs you have
  -p:      Set the load sampling period (seconds)
  -D:      Store statistics in a dbm file
\end{lstlisting}\null\\
\subsection{Calibrating CPU load \textcolor{green}{[pass]}}
\begin{lstlisting}
root@localhost:/root> cyclesoak   
using 1 CPUs
Please run `cyclesoak -C' on an unloaded system

root@localhost:/root> cyclesoak -C
using 1 CPUs
calibrating: 23238 loops/sec
calibrating: 23535 loops/sec
calibrating: 23734 loops/sec
calibrating: 23735 loops/sec
calibrated OK.  23735 loops/sec

root@localhost:/root> cyclesoak 
using 1 CPUs
System load:  1.9%
System load:  0.7%
System load: -0.0%
System load: -0.0%
System load:  0.1%
System load: -0.0%
System load: -0.0%
System load: -0.0%
System load: -0.1%
System load:  0.0%
^C
root@localhost:/root>
\end{lstlisting}
Checkpoints:
\begin{itemize}
    \item Only one CPU;
    \item Running forever;
    \item Calibrating is about 23000 loops/sec;
\end{itemize}
Use {\tt ctrl-c} to escape.\\\\
\subsection{CPU loading changes while doing {\tt ssh} connection 
            and {\tt iperf} testing \textcolor{green}{[pass]}}
Use {\tt -p} option, output every 3 seconds:
\begin{lstlisting}
root@localhost:/root> cyclesoak -p 3
using 1 CPUs
System load:  2.1%
System load:  0.0%
System load:  5.9%
System load: 30.6%
System load: -0.0%
System load:  0.5%
System load:  0.2%
System load:  0.1%
System load: -0.0%
System load:  3.2%
System load:  0.0%
System load: -0.0%
System load: -0.0%
System load: 20.5%
System load: 41.0%
System load: 41.1%
System load: 41.1%
System load: 41.0%
System load: 40.9%
System load: 20.8%
System load:  0.0%
System load: -0.0%
\end{lstlisting}
The first {\tt 30.6\%} is caused by a ssh connection -- {\tt sshd} runs on the board, 
and a {\tt ssh} client runs on the laptop to login.\\
In the end the sudden high loading series are caused by {\tt iperf} tests --\\
{\tt iperf -s -i 1 -w 1M} on the board as server, while \\
{\tt iperf -c 128.224.165.247 -i 1 -w 1M} on the laptop as client.\\
For more details about {\tt iperf}, get to chapter \ref{iperf}.\\\\
\subsection{Store the outputs to {\tt gdbm} (GNU Database Manager) with {\tt iperf} 
            UDP full-duplex testing \textcolor{green}{[pass]}}
{\tt iperf -s -i 1 -w 1M -u} on the board as server,\\
{\tt iperf -c 128.224.165.247 -i 1 -w 1M -u -d} on the laptop as client.
\begin{lstlisting}
root@localhost:/root> cyclesoak -p 2 -D a.dbm
using 1 CPUs
System load:  7.1%
System load:  5.6%
System load:  5.4%
System load:  5.1%
System load:  3.4%
System load:  3.0%
System load:  3.0%
\end{lstlisting}
Much lower than TCP testing.\\\\
I found no gdbm reading tool in the distro repository, so write a simple to check:
\begin{lstlisting}
root@localhost:/root> gdbm_reader a.dbm 
key: 00:21:17, len 9;		val:   3.4, len 6
key: 00:21:15, len 9;		val:   5.1, len 6
key: 00:21:21, len 9;		val:   3.0, len 6
key: 00:20:33, len 9;		val:   4.8, len 6
key: 00:21:09, len 9;		val:   7.1, len 6
key: 00:21:13, len 9;		val:   5.4, len 6
key: 00:20:39, len 9;		val:   2.8, len 6
key: 00:21:11, len 9;		val:   5.6, len 6
key: 00:21:19, len 9;		val:   3.0, len 6
key: 00:20:36, len 9;		val:   3.3, len 6
key: 00:20:42, len 9;		val:   1.7, len 6
\end{lstlisting}
The 2nd column is the system time, haven't set yet.\\
The output sequence is out of time order, don't worry it's reasonable.
This access is not key sequential, but it is guaranteed to visit every key in the database once. 
The order has to do with the hash values.\\\\
Here's the source ({\tt gdbm\_reader.c}):
\begin{lstlisting}
#include <stdio.h>
#include <fcntl.h>
#include <gdbm.h>

int main(int argc, const char *argv[])
{
    GDBM_FILE dbf;
    datum key, val;
    if (!(dbf = gdbm_open("a.dbm", 512, GDBM_READER, O_RDONLY, NULL))) {
        perror("gdbm reader: ");
        fprintf(stderr, "%s\n", gdbm_strerror(gdbm_errno));
        return 1;
    }
    key = gdbm_firstkey(dbf);
    while (key.dptr) {
        val = gdbm_fetch(dbf, key);
        fprintf(stderr, "key: %s, len %d;\t\t", key.dptr, key.dsize);
        fprintf(stderr, "val: %s, len %d\n", val.dptr, val.dsize);
        key = gdbm_nextkey(dbf, key);
    }
    gdbm_close(dbf);
    return 0;
}
\end{lstlisting}\null\\
\subsection{Per CPU statistics \textcolor{green}{[pass]}}
\begin{lstlisting}
root@localhost:/root> cyclesoak -B
using 1 CPUs
System load:  2.1% || Free:  97.9%(0)
System load:  0.9% || Free:  99.1%(0)
System load: -0.0% || Free: 100.0%(0)
System load: -0.0% || Free: 100.0%(0)
System load: -0.0% || Free: 100.0%(0)
System load:  0.0% || Free: 100.0%(0)
System load:  0.5% || Free:  99.5%(0)
System load: -0.0% || Free: 100.0%(0)
System load: -0.0% || Free: 100.0%(0)
System load:  0.0% || Free: 100.0%(0)
System load: -0.0% || Free: 100.0%(0)
System load:  0.1% || Free:  99.9%(0)
System load: -0.0% || Free: 100.0%(0)
System load: -0.0% || Free: 100.0%(0)
System load:  0.0% || Free: 100.0%(0)
System load:  0.0% || Free: 100.0%(0)
^C
root@localhost:/root> 
\end{lstlisting}
Checkpoints:
\begin{itemize}
    \item The Mindspeed target has only one core, so the {\tt Free} part has only one column;
    \item {\tt System load} plus {\tt Free} should be 100\%;
\end{itemize}
\subsection{Debug mode \textcolor{green}{[pass]}}
\begin{lstlisting}
root@localhost:/root> cyclesoak -d     
using 1 CPUs
CPU0: delta=1016, blp_snapshot=23634, old_blp=0, diff=23634 System load:  2.0%
CPU0: delta=985, blp_snapshot=46822, old_blp=23634, diff=23188 System load:  0.8%
CPU0: delta=1003, blp_snapshot=70642, old_blp=46822, diff=23820 System load: -0.1%
CPU0: delta=1003, blp_snapshot=94446, old_blp=70642, diff=23804 System load:  0.0%
CPU0: delta=1003, blp_snapshot=118248, old_blp=94446, diff=23802 System load:  0.0%
CPU0: delta=1003, blp_snapshot=142059, old_blp=118248, diff=23811 System load: -0.0%
CPU0: delta=985, blp_snapshot=165414, old_blp=142059, diff=23355 System load:  0.1%
CPU0: delta=1003, blp_snapshot=189218, old_blp=165414, diff=23804 System load:  0.0%
CPU0: delta=1003, blp_snapshot=213021, old_blp=189218, diff=23803 System load:  0.0%
CPU0: delta=1003, blp_snapshot=236828, old_blp=213021, diff=23807 System load:  0.0%
^C
root@localhost:/root> 
\end{lstlisting}
Checkpoints:
\begin{itemize}
    \item {\tt delta} is about 1000;
    \item {\tt diff} is about 23000 when system is free, because it's the loops/sec;
\end{itemize}
\subsection{Different period \textcolor{green}{[pass]}}
\begin{lstlisting}
root@localhost:/root> ./cyclesoak -p 2 -d
using 1 CPUs
CPU0: delta=1009, blp_snapshot=23468, old_blp=0, diff=23468 System load:  2.0%
CPU0: delta=2006, blp_snapshot=70946, old_blp=23468, diff=47478 System load:  0.3%
CPU0: delta=1989, blp_snapshot=118134, old_blp=70946, diff=47188 System load:  0.0%
CPU0: delta=2007, blp_snapshot=165780, old_blp=118134, diff=47646 System load: -0.0%
CPU0: delta=1989, blp_snapshot=212979, old_blp=165780, diff=47199 System load:  0.0%
CPU0: delta=2006, blp_snapshot=260632, old_blp=212979, diff=47653 System load: -0.1%
CPU0: delta=2007, blp_snapshot=308279, old_blp=260632, diff=47647 System load: -0.0%
CPU0: delta=1989, blp_snapshot=355483, old_blp=308279, diff=47204 System load:  0.0%
CPU0: delta=2007, blp_snapshot=403131, old_blp=355483, diff=47648 System load: -0.0%
^C
root@localhost:/root> 
\end{lstlisting}
Checkpoints:
\begin{itemize}
    \item Same period as we specified;
\end{itemize}
%\chapter{cyclesoak}


\chapter{dnsmasq}
\section{Description}
It is a lightweight DNS, TFTP and DHCP server. It is intended to provide 
coupled DNS and DHCP service to a LAN.\\\\
{\tt dnsmasq}  accepts DNS queries and either answers them from a small, 
local, cache or forwards them to a real, recursive, DNS server.
It loads the contents of {\tt /etc/hosts} so that local hostnames which do 
not appear in the global DNS can be resolved and also answers
DNS queries for DHCP configured hosts.\\\\
The  DHCP server supports static address assignments and multiple networks. 
It automatically sends a sensible default set
of DHCP options, and can be configured to send any desired set of DHCP options, 
including vendor-encapsulated options. It includes a secure, read-only, 
TFTP server to allow net/PXE boot of DHCP hosts and also supports BOOTP.\\\\
Dnsmasq supports IPv6 for DNS, but not DHCP.
\section{Installation Hierarchy}
\begin{lstlisting}
etc/dnsmasq.d/
etc/dnsmasq.conf
etc/rc.d/dnsmasq
usr/sbin/dnsmasq
\end{lstlisting}
\section{Warning}
\begin{itemize}
    \item Also supports {\tt tftp}, but board won't provide that, no need to test;
    \item {\tt dnsmasq --help dhcp} to show all dhcp config options;
    \item Shutdown {\tt NetworkManager} on the client, otherwise:
\begin{lstlisting}
% sudo /etc/init.d/network restart
Shutting down interface wlan0:         [  OK  ]
Shutting down loopback interface:      [  OK  ]
Bringing up loopback interface:        [  OK  ]
Bringing up interface eth0:  Error: 
    Connection activation failed: Device not managed by NetworkManager
                                       [FAILED]
\end{lstlisting}
    \item Sending {\tt SIGHUP} to the {\tt dnsmasq} process will cause it to 
          empty its cache and then re-load {\tt /etc/hosts} and {\tt /etc/resolv.conf}, 
          namely restart. E.g. {\tt killall -s SIGHUP dnsmasq};
    \item Sending {\tt SIGUSR1} will cause it to write cache usage statisticss to the 
          log, typically {\tt /var/log/syslog} or {\tt /var/log/messages};
    \item From its config file, when use with {\tt samba}:
\begin{lstlisting}
# The following DHCP options set up dnsmasq in the same
# way as is specified for the ISC dhcpcd in
# http://www.samba.org/samba/ftp/docs/textdocs/DHCP-Server-
# Configuration.txt
# adapted for a typical dnsmasq installation where the host
# running dnsmasq is also the host running samba.
# you may want to uncomment some or all of them if you use 
# Windows clients and Samba.
#dhcp-option=19,0       # option ip-forwarding off
#dhcp-option=44,0.0.0.0 # set netbios-over-TCP/IP nameserver(s) aka WINS server(s)
#dhcp-option=45,0.0.0.0 # netbios datagram distribution server
#dhcp-option=46,8       # netbios node type
\end{lstlisting}
\end{itemize}
\section{Test}
\subsection{New domain-IP bindings \textcolor{green}{[pass]}}
Add {\tt address=/dnsmasq\_test/192.168.1.100} to {\tt /etc/dnsmasq.conf}, restart
{\tt dnsmasq} (as reload config file), on the laptop use {\tt dig} while specify
the board's IP:
\begin{lstlisting}
% dig dnsmasq_test @128.224.165.245

; <<>> DiG 9.7.2-P3 <<>> dnsmasq_test @128.224.165.245
;; global options: +cmd
;; Got answer:
;; ->>HEADER<<- opcode: QUERY, status: NOERROR, id: 4221
;; flags: qr aa rd ra; QUERY: 1, ANSWER: 1, AUTHORITY: 0, ADDITIONAL: 0

;; QUESTION SECTION:
;dnsmasq_test.			IN	A

;; ANSWER SECTION:
dnsmasq_test.		0	IN	A	192.168.1.100

;; Query time: 2 msec
;; SERVER: 128.224.165.245#53(128.224.165.245)
;; WHEN: Tue Feb 15 19:09:14 2011
;; MSG SIZE  rcvd: 46
\end{lstlisting}
Checkpoints:
\begin{itemize}
    \item In the {\tt ->>HEADER<<-} line, status is {\tt NOERROR};
    \item Has {\tt ANSWER SECTION};
    \item Result IP is same as what we set in {\tt /etc/dnsmasq.conf} on the board;
    \item {\tt SERVER} IP is correct.
\end{itemize}
Notice that {\tt dig} will search the DNS server listed in {\tt /etc/resolv.conf},
if add no board's IP to it, {\tt dig dnsmasq\_test} will show:
\begin{lstlisting}
% dig dnsmasq_test                 

; <<>> DiG 9.7.2-P3 <<>> dnsmasq_test
;; global options: +cmd
;; Got answer:
;; ->>HEADER<<- opcode: QUERY, status: SERVFAIL, id: 1223
;; flags: qr rd ra; QUERY: 1, ANSWER: 0, AUTHORITY: 0, ADDITIONAL: 0

;; QUESTION SECTION:
;dnsmasq_test.			IN	A

;; Query time: 0 msec
;; SERVER: 128.224.160.11#53(128.224.160.11)
;; WHEN: Tue Feb 15 19:18:01 2011
;; MSG SIZE  rcvd: 30
\end{lstlisting}
Status is {\tt SERVFAIL} and no {\tt ANSWER SECTION}.\\
Consequently, specify a seach server to {\tt dig}, or add {\tt nameserver} to
{\tt /etc/resolv.conf}.
While check the port on the board:
\begin{lstlisting}
root@localhost:/root> netstat -atunp | egrep "(dns|State)"
Proto Recv-Q Send-Q Local Address Foreign Address State  PID/Program
tcp        0      0 0.0.0.0:53    0.0.0.0:*       LISTEN 1153/dnsmasq        
tcp        0      0 :::53         :::*            LISTEN 1153/dnsmasq        
udp        0      0 0.0.0.0:53    0.0.0.0:*              1153/dnsmasq        
udp        0      0 :::53         :::*                   1153/dnsmasq
\end{lstlisting}
Checkpoint:
\begin{itemize}
    \item Port is 53
\end{itemize}
\subsection{New IP to exist domain name \textcolor{green}{[pass]}}
Add {\tt address=/yahoo.com/7.7.7.7} to {\tt /etc/dnsmasq.conf}.\\
On the laptop:
\begin{lstlisting}
% dig yahoo.com @128.224.165.245    

; <<>> DiG 9.7.2-P3 <<>> yahoo.com @128.224.165.245
;; global options: +cmd
;; Got answer:
;; ->>HEADER<<- opcode: QUERY, status: NOERROR, id: 50390
;; flags: qr aa rd ra; QUERY: 1, ANSWER: 1, AUTHORITY: 0, ADDITIONAL: 0

;; QUESTION SECTION:
;yahoo.com.			IN	A

;; ANSWER SECTION:
yahoo.com.		0	IN	A	7.7.7.7

;; Query time: 2 msec
;; SERVER: 128.224.165.245#53(128.224.165.245)
;; WHEN: Tue Feb 15 19:25:15 2011
;; MSG SIZE  rcvd: 43

\end{lstlisting}
Checkpoints:
\begin{itemize}
    \item In the {\tt ->>HEADER<<-} line, status is {\tt NOERROR};
    \item Has {\tt ANSWER SECTION};
    \item Result IP is same as what we set in {\tt /etc/dnsmasq.conf} on the board;
    \item {\tt SERVER} IP is correct.
\end{itemize}
\subsection{New domain name to exist IP \textcolor{green}{[pass]}}
\subsubsection{Resolve {\tt fake\_google}}
Add {\tt address=/fake\_google/8.8.8.8} to {\tt /etc/dnsmasq.conf}.\\
8.8.8.8 is a famous Google DNS server:
\begin{lstlisting}
% host 8.8.8.8                       
8.8.8.8.in-addr.arpa domain name pointer google-public-dns-a.google.com.
\end{lstlisting}
This time use {\tt nslookup} on the laptop:
\begin{lstlisting}
% nslookup fake_google 128.224.165.245
Server:		128.224.165.245
Address:	128.224.165.245#53

Name:	fake_google
Address: 8.8.8.8
\end{lstlisting}
Checkpoints:
\begin{itemize}
    \item {\tt Server} IP is correct;
    \item {\tt Name} is what we want;
    \item {\tt Address} is same as what we set in {\tt /etc/dnsmasq.conf} on the board;
\end{itemize}
\subsubsection{Ping {\tt fake\_google}}
Add a new nameserver with boards's IP to {\tt /etc/resolv.conf} on the laptop, comment
{\tt domain} and {\tt search}, then {\tt ping fake\_google}:
\begin{lstlisting}
% cat /etc/resolv.conf
# Generated by NetworkManager
#domain corp.ad.wrs.com
#search corp.ad.wrs.com
nameserver 128.224.165.245
nameserver 128.224.160.11
nameserver 147.11.100.30
nameserver 147.11.1.11

% ping fake_google -c 5
PING fake_google (8.8.8.8) 56(84) bytes of data.
64 bytes from google-public-dns-a.google.com (8.8.8.8): icmp_req=1 ttl=48 time=243 ms
64 bytes from google-public-dns-a.google.com (8.8.8.8): icmp_req=2 ttl=48 time=243 ms
64 bytes from google-public-dns-a.google.com (8.8.8.8): icmp_req=3 ttl=48 time=242 ms
64 bytes from google-public-dns-a.google.com (8.8.8.8): icmp_req=4 ttl=48 time=244 ms
64 bytes from google-public-dns-a.google.com (8.8.8.8): icmp_req=5 ttl=48 time=243 ms

--- fake_google ping statistics ---
5 packets transmitted, 5 received, 0% packet loss, time 4005ms
rtt min/avg/max/mdev = 242.155/243.290/244.653/0.858 ms
\end{lstlisting}
Checkpoints:
\begin{itemize}
    \item Packages sending and receiving are almost ok;
    \item IP is correct;
\end{itemize}
It reverses IP 8.8.8.8 to it's original domain name {\tt google-public-dns-a.google.com}.
\subsection{{\tt SIGHUP} handling \textcolor{red}{[fail]}}
Add a new rule {\tt address/fake\_google\_again} in {\tt /etc/dnsmasq.conf}, then
{\tt killall -s SIGHUP dnsmasq} to send SIGHUP to the running process.\\
On the laptop:
\begin{lstlisting}
% nslookup fake_google_again 128.224.165.245
Server:		128.224.165.245
Address:	128.224.165.245#53

** server can't find fake_google_again: NXDOMAIN
\end{lstlisting}
Checkpoints:
\begin{itemize}
    \item {\tt Server} IP is correct;
    \item {\tt Name} is what we want;
    \item {\tt Address} is same as what we've updated in 
          {\tt /etc/dnsmasq.conf} on the board;
\end{itemize}
Here it seems that sending SIGHUP didn't cause the config files reloaded.
\subsection{{\tt SIGUSR1} handling \textcolor{green}{[pass]}}
Uncomment {\tt #log-queries} at the end of {\tt /etc/dnsmasq.conf} for
more log as debugging purpose. When receive {\tt SIGUSR1}, all dump.\\
On the laptop, use {\tt dig} to query domain {\tt dnsmasq\_test} again, 
get 192.168.1.100 from server 128.224.165.245:
\begin{lstlisting}
% dig dnsmasq_test @128.224.165.245        

; <<>> DiG 9.7.2-P3 <<>> dnsmasq_test @128.224.165.245
;; global options: +cmd
;; Got answer:
;; ->>HEADER<<- opcode: QUERY, status: NOERROR, id: 45010
;; flags: qr aa rd ra; QUERY: 1, ANSWER: 1, AUTHORITY: 0, ADDITIONAL: 0

;; QUESTION SECTION:
;dnsmasq_test.			IN	A

;; ANSWER SECTION:
dnsmasq_test.		0	IN	A	192.168.1.100

;; Query time: 9 msec
;; SERVER: 128.224.165.245#53(128.224.165.245)
;; WHEN: Tue Feb 15 17:22:48 2011
;; MSG SIZE  rcvd: 46
\end{lstlisting}
Then on the board, send {\tt SIGUSR1} and check {\tt /var/log/syslog}:
\begin{lstlisting}
root@localhost:/root> killall -s SIGUSR1 dnsmasq
root@localhost:/root> cat /va/log/syslog
...
Jan  1 01:57:34 localhost dnsmasq[808]: exiting on receipt of SIGTERM
Jan  1 01:57:34 localhost /etc/init.d/dnsmasq: dnsmasq shutdown - OK
Jan  1 01:57:34 localhost dnsmasq[841]: started, version 2.56 cachesize 150
Jan  1 01:57:34 localhost dnsmasq[841]: compile time options: IPv6 GNU-getopt no-DBus no-I18N DHCP TFTP
Jan  1 01:57:34 localhost dnsmasq[841]: failed to access /etc/resolv.conf: No such file or directory
Jan  1 01:57:34 localhost dnsmasq[841]: read /etc/hosts - 2 addresses
Jan  1 01:57:34 localhost /etc/init.d/dnsmasq: dnsmasq startup - OK
Jan  1 01:58:13 localhost dnsmasq[841]: query[A] dnsmasq_test from 128.224.158.134
Jan  1 01:58:13 localhost dnsmasq[841]: config dnsmasq_test is 192.168.1.100
Jan  1 02:00:01 localhost crond[845]: (root) CMD (/usr/lib/sa/sa1 -d 1 1)
Jan  1 02:01:49 localhost dnsmasq[841]: time 7309
Jan  1 02:01:49 localhost dnsmasq[841]: cache size 150, 0/0 cache insertions re-used unexpired cache entries.
Jan  1 02:01:49 localhost dnsmasq[841]: queries forwarded 0, queries answered locally 1
Jan  1 02:01:49 localhost dnsmasq[841]: Host                    Address   Flags  Expires
Jan  1 02:01:49 localhost dnsmasq[841]: localhost               ::1       6FRI   H  
Jan  1 02:01:49 localhost dnsmasq[841]: localhost               127.0.0.1 4FRI   H  
Jan  1 02:01:49 localhost dnsmasq[841]: localhost.localdomain   ::1       6F I   H  
Jan  1 02:01:49 localhost dnsmasq[841]: localhost.localdomain   127.0.0.1 4F I   H  
Jan  1 02:01:49 localhost dnsmasq[841]: localhost4.localdomain4 127.0.0.1 4F I   H  
Jan  1 02:01:49 localhost dnsmasq[841]: localhost6              ::1       6F I   H  
Jan  1 02:01:49 localhost dnsmasq[841]: localhost4              127.0.0.1 4F I   H  
Jan  1 02:01:49 localhost dnsmasq[841]: localhost6.localdomain6 ::1       6F I   H  
\end{lstlisting}
Checkpoints:
\begin{itemize}
    \item A lot of info about process {\tt dnsmasq};
    \item {\tt pid} is correct;
    \item Server IP is correct;
    \item Result IP is correct;
    \item Domain name is what we want;
\end{itemize}
It shows we executed {\tt /etc/init.d/dnsmasq restart} (sent {\tt SIGTERM} to pid 808, 
after that started a new with pid 841), and read {\tt /etc/hosts}, and got a request from
128.224.158.134 (the laptop), finally found the domain {\tt dnsmasq\_test} 
has IP 192,168.1.100.\\
All match.
\subsection{DNS relay \textcolor{green}{[pass]}}
Make {\tt dnsmasq} as a relay, specify other DNS server for some domains.\\\\
Normal query to the board:
\begin{lstlisting}
% nslookup douban.com 128.224.165.245
Server:		128.224.165.245
Address:	128.224.165.245#53

** server can't find douban.com: REFUSED
\end{lstlisting}
Can't find.\\
Add {\tt server=/douban.com/147.11.1.11} to {\tt /etc/dnsmasq.conf}, then try again:
\begin{lstlisting}
% nslookup douban.com 128.224.165.245
Server:		128.224.165.245
Address:	128.224.165.245#53

Non-authoritative answer:
Name:	douban.com
Address: 211.147.4.31
\end{lstlisting}
On laptop, compare with:
\begin{lstlisting}
% host douban.com
douban.com has address 211.147.4.31
douban.com mail is handled by 20 mail2.douban.com.
douban.com mail is handled by 10 mail3.douban.com.
douban.com mail is handled by 15 mail4.douban.com.
douban.com mail is handled by 20 mail.douban.com.
\end{lstlisting}
Checkpoint:
\begin{itemize}
    \item IPs are same;
\end{itemize}
\subsection{DHCP functionality}
Ideal test environment:
\begin{lstlisting}
                                                  
   +-------------+               +---------------+
   |             |eth0           |               |
   |    laptop   +---------------+    board      |
   |    client   |           eth0|    server     |
   |             |               |               |
   +-------------+               +---------------+

\end{lstlisting}
Connect directly, without WAN. I have no board so just use laptop to desktop:
\begin{lstlisting}
                                                  
   +-------------+               +---------------+
   |             |eth0           |               |
   |    laptop   +---------------+    desktop    |
   |    client   |           eth0|    server     |
   |             |               |               |
   +-------------+               +---------------+

\end{lstlisting}
They said on the server, only uncomment and set {\tt expand-hosts}, {\tt domain},
{\tt dhcp-range}, like:
\begin{lstlisting}
expand-hosts
domain=example.com
dhcp-range=192.168.0.50,192.168.0.150,255.255.255.0,12h
\end{lstlisting}
Warning:
\begin{itemize}
    \item Server's IP is in the same subnet with the range, if not, set it manaully.
          Here we set to 192.168.0.2;
    \item Shutdown {\tt NetworkManager} on client;
\end{itemize}
\begin{lstlisting}
% sudo /etc/init.d/network restart
Shutting down interface eth0:                              [  OK  ]
Shutting down interface wlan0:                             [  OK  ]
Shutting down loopback interface:                          [  OK  ]
Bringing up loopback interface:                            [  OK  ]
Bringing up interface eth0:  
Determining IP information for eth0... done.
                                                           [  OK  ]
% ifconfig eth0 | head -3
eth0  Link encap:Ethernet  HWaddr 00:22:15:EB:64:A3  
      inet addr:192.168.0.57  Bcast:192.168.0.255  Mask:255.255.255.0
      inet6 addr: fe80::222:15ff:feeb:64a3/64 Scope:Link
\end{lstlisting}
Then ping from client:
\begin{lstlisting}
% ping 192.168.0.2
PING 192.168.0.2 (192.168.0.2) 56(84) bytes of data.
64 bytes from 128.224.158.139: icmp_req=1 ttl=64 time=2.48 ms
64 bytes from 128.224.158.139: icmp_req=2 ttl=64 time=0.401 ms
64 bytes from 128.224.158.139: icmp_req=3 ttl=64 time=0.395 ms
^C
--- 192.168.0.2 ping statistics ---
3 packets transmitted, 3 received, 0% packet loss, time 2001ms
rtt min/avg/max/mdev = 0.395/1.093/2.483/0.982 ms
\end{lstlisting}
And {\tt arp -n} to check MAC-IP pairs each other:\\
client:
\begin{lstlisting}
% arp -n | head -2
Address      HWtype HWaddress         Flags Mask Iface
192.168.1.10 ether  78:2b:cb:84:ea:05 C          eth0
\end{lstlisting}
server:
\begin{lstlisting}
% arp -n | head -2
Address      HWtype HWaddress         Flags Mask Iface
192.168.1.20 ether  00:22:15:eb:64:a3 C          eth0
\end{lstlisting}
Check leases each other, {\tt /var/lib/dnsmasq/dnsmasq.leases} for server:
\begin{lstlisting}
lease 192.168.0.57 {
  starts 1 2011/02/28 22:10:54;
  ends 4 2011/03/03 22:10:54;
  cltt 1 2011/02/28 22:10:54;
  binding state active;
  next binding state free;
  rewind binding state free;
  hardware ethernet 00:22:15:eb:64:a3;
}
\end{lstlisting}
{\tt /var/lib/dhclient/dhclient-eth0.leases} for client:
\begin{lstlisting}
lease {
  interface "eth0";
  fixed-address 192.168.0.57;
  option subnet-mask 255.255.255.0;
  option routers 192.168.0.1;
  option dhcp-lease-time 259200;
  option dhcp-message-type 5;
  option domain-name-servers 147.11.1.11;
  option dhcp-server-identifier 192.168.0.2;
  option nis-domain "example.com";
  option broadcast-address 192.168.0.255;
  option domain-name "example.com";
  renew 2 2011/03/01 20:45:41;
  rebind 3 2011/03/02 23:53:47;
  expire 4 2011/03/03 08:53:47;
}
\end{lstlisting}
Check whether they match well, the IP, MAC, mask, etc.\\
And the {\tt domain} field of {\tt /etc/hosts} should be the name set in 
{\tt domain} of {\tt /etc/dnsmasq.conf}.
\subsection{binding IP with certain MAC}
Allocate a certain IP to a certain MAC, like 78:2b:cb:84:ea:05, to 192.168.0.60.
Set in {\tt /etc/dnsmasq.conf}:
\begin{lstlisting}
dhcp-host=78:2b:cb:84:ea:05,192.168.0.60
\end{lstlisting}
Checkpoints are the same as previous.
%\chapter{dnsmasq}


\chapter{hotplug2-1.0}
\section{Description}
{\tt hotplug2} is aimed at early Linux user space, that is, {\tt initramfs} 
or {\tt initrd}, possibly also at embedded devices (such as WRT-like routers) 
or very weak machines, such as 386/486 (some still use them as workstations).\\\\
{\tt hotplug2} connects to the uevent netlink socket and reads events. It supports
simple rules for processing the events. The rules allow matching of the
variables obtained by the uevent socket and allows execution of applications,
with those variables set as their environmental variables. Further
documentation of rules syntax is available.\\\\
Alternatively, {\tt hotplug2} can run without any rules, and simply blindly attempt
to modprobe whenever an adequate event is received.\\\\
{\tt hotplug2} supports cold plugging simply by calling the {\tt udevtrigger} binary
internally. As {\tt udevtrigger} is fairly independent on the rest of {\tt udev}, 
it can be easily embedded along with {\tt hotplug2}.\\\\
The advantages over {\tt udev} are:
\begin{itemize}
    \item Faster processing of events - that is especially notable on slower hardware;
    \item Much smaller footprint of the application; 
\end{itemize}
The disadvantages:
\begin{itemize}
    \item Limited rules system, designed rather for trivial operations than
          for complex desktop setups;
\end{itemize}
Therefore, {\tt hotplug2} definitely is not aimed as replacement
for {\tt udev} on desktop, with the exception of {\tt initramfs}/{\tt initrd}.\\\\
{\tt OpenWrt} starts it in its initial scripts, to handle uevents during startup.
\section{Installation Hierarchy}
\begin{lstlisting}
/sbin/hotplug2
/sbin/hotplug2-modwrap
/sbin/coldplug2
/etc/hotplug2.rules
/etc/hotplug2-complex-ownership.rules
/etc/automount.rules
/etc/automount-nofstab.rules
/lib/hotplug2/worker_fork.so
/lib/hotplug2/worker_single.so
\end{lstlisting}
{\tt base-files} also adds these two:
\begin{lstlisting}
/etc/hotplug2-init.rules
/etc/hotplug2-common.rules
\end{lstlisting}
\section{Warning}
\begin{itemize}
    \item 8 official patches needed, from {\tt OpenWrt}, and another one for config rules;
    \item Written in 2008 v1.0 changed the infrastructure of v0.9, what a pity not 
          complete, in {\tt coldplug2.c} (compiled into {\tt coldplug2}):
\begin{lstlisting}
...
static int 

/**
 * Recurrent attempt to find all 'uevent' files. Does not
 * follow symlinks.
 *
 * @1 Base directory
 * 
 * Returns: 0 on success, -1 on failure.
 */
static int coldplug2_uevent_lookup(const char *basepath) {
}
...
\end{lstlisting}
          {\tt coldplug2\_uevent\_lookup} is a key function, {\tt makefile} is lack
          of some rules; But main parts for {\tt hotplug2} is ok, so {\tt OpenWrt}
          dares to use it instead of v0.9;
    \item It depends on the newest version of {\tt base-files}, which has got huge changes;
    %\item Set a new cmdline option, {\tt init=/etc/preinit}, it's {\tt OpenWrt} style;
    \item In {\tt OpenWrt} init process, call it in this way:
\begin{lstlisting}
killall -q hotplug2
[ -x /sbin/hotplug2 ] && /sbin/hotplug2 --override --persistent \
    --set-worker /lib/hotplug2/worker_fork.so \
    --set-rules-file /etc/hotplug2.rules \
    --max-children 1 >/dev/null 2>&1 &
\end{lstlisting}
    \item  {\tt /sbin/hotplug-call} is a shell scrit, which is part of {\tt base-files}:
\begin{lstlisting}
#!/bin/sh
# Copyright (C) 2006-2010 OpenWrt.org

export HOTPLUG_TYPE="$1"

. /etc/functions.sh

PATH=/bin:/sbin:/usr/bin:/usr/sbin
LOGNAME=root
USER=root
export PATH LOGNAME USER

[ \! -z "$1" -a -d /etc/hotplug.d/$1 ] && {
	for script in $(ls /etc/hotplug.d/$1/* 2>&-); do (
		[ -f $script ] && . $script
	); done
}
\end{lstlisting}
          {\tt /etc/hotplug.d/} is in {\tt base-files} too, including {\tt ieee1394/
          iface/ net/ usb/} as default. It uses
          {\tt /etc/hotplug2.rules}, neither {\tt /etc/hotplug2-common.rules} nor 
          {\tt /etc/hotplug2-init.rules};
    \item for the formmer 2 tests, must set {\tt base-files} first; for the left, could
          kill {\tt udev} then start {\tt hotplug2} to test;
\end{itemize}

\section{Test}
\subsection{log hotplug events}
Patched config rule file {\tt /etc/hotplug2-common.rules}:
\begin{lstlisting}
DEVPATH is set {
      nothrottle
      exec logger -s -t hotplug -p daemon.info "name=%DEVICENAME%, path=%DEVPATH%, act=%ACTION%"
}
\end{lstlisting}
For {\tt logger}, {\tt -s} means to system log as well as to standard error, {\tt -t}
means add a tag to every log.\\
Checkpoints:
\begin{itemize}
    \item log like {\tt hotplug: name=xxx, path=xxx, act=xxx} on terminal;
    \item same format log, in {\tt /var/log/messages};
\end{itemize}
\subsection{device node added automatically}
For such rules in {\tt /etc/hotplug2-common.rules}:
\begin{lstlisting}
...
DEVICENAME ~~ (^null$|^full$|^ptmx$|^tty|^zero$|^gpio|^hvc) {
	makedev /dev/%DEVICENAME% 0666
	next-event
}

DEVICENAME ~~ (^tun|^tap[0-9]$) {
	makedev /dev/net/%DEVICENAME% 0644
}

DEVICENAME ~~ ^ppp {
	makedev /dev/%DEVICENAME% 0600
}
...
\end{lstlisting}
It'll add device nodes during the startup.
\begin{itemize}
    \item existences, like {\tt /dev/null, /dev/full, /dev/ttyX, /dev/ppp}...;
    \item modes, compared with what in the rules;
\end{itemize}
\subsection{firmware auto download}
By default:
\begin{lstlisting}
% lsmod | grep rt                                       
xfrm4_mode_transport      981  0 
xfrm6_mode_transport     1013  0 
parport_pc             18037  0 
parport                26215  2 ppdev,parport_pc
iTCO_vendor_support     2070  1 iTCO_wdt
% lsmod | grep usb  
%
\end{lstlisting}
Then plug {\tt TP-LINK} wireless card to the target, on terminal we could see:
\begin{lstlisting}
% hotplug: name=2-1.6, path=/devices/pci0000:00/0000:00:1d.0/usb2/2-1/2-1.6, act=add
hotplug: name=2-1.6:1.0, path=/devices/pci0000:00/0000:00:1d.0/usb2/2-1/2-1.6/2-1.6:1.0, act=add
hotplug: name=rfkill, path=/module/rfkill, act=add
hotplug: name=rfkill, path=/class/rfkill, act=add
hotplug: name=rfkill, path=/devices/virtual/misc/rfkill, act=add
hotplug: name=cfg80211, path=/module/cfg80211, act=add
hotplug: name=ieee80211, path=/class/ieee80211, act=add
FATAL: Module platform:regulatory not found.
hotplug: name=regulatory.0, path=/devices/platform/regulatory.0, act=add
hotplug: name=regulatory.0, path=/devices/platform/regulatory.0, act=change
hotplug: name=mac80211, path=/module/mac80211, act=add
hotplug: name=rt2x00lib, path=/module/rt2x00lib, act=add
hotplug: name=rt2x00usb, path=/module/rt2x00usb, act=add
hotplug: name=rt2500usb, path=/module/rt2500usb, act=add
hotplug: name=rt2500usb, path=/bus/usb/drivers/rt2500usb, act=add
hotplug: name=crc_itu_t, path=/module/crc_itu_t, act=add
hotplug: name=rt73usb, path=/module/rt73usb, act=add
hotplug: name=phy1, path=/devices/pci0000:00/0000:00:1d.0/usb2/2-1/2-1.6/2-1.6:1.0/ieee80211/phy1, act=add
hotplug: name=rfkill0, path=/devices/pci0000:00/0000:00:1d.0/usb2/2-1/2-1.6/2-1.6:1.0/ieee80211/phy1/rfkill0, act=add
hotplug: name=rfkill0, path=/devices/pci0000:00/0000:00:1d.0/usb2/2-1/2-1.6/2-1.6:1.0/ieee80211/phy1/rfkill0, act=change
hotplug: name=arc4, path=/module/arc4, act=add
hotplug: name=wlan0, path=/devices/pci0000:00/0000:00:1d.0/usb2/2-1/2-1.6/2-1.6:1.0/net/wlan0, act=add
hotplug: name=rx-0, path=/devices/pci0000:00/0000:00:1d.0/usb2/2-1/2-1.6/2-1.6:1.0/net/wlan0/queues/rx-0, act=add
hotplug: name=rx-1, path=/devices/pci0000:00/0000:00:1d.0/usb2/2-1/2-1.6/2-1.6:1.0/net/wlan0/queues/rx-1, act=add
hotplug: name=rx-2, path=/devices/pci0000:00/0000:00:1d.0/usb2/2-1/2-1.6/2-1.6:1.0/net/wlan0/queues/rx-2, act=add
hotplug: name=rx-3, path=/devices/pci0000:00/0000:00:1d.0/usb2/2-1/2-1.6/2-1.6:1.0/net/wlan0/queues/rx-3, act=add
hotplug: name=rt73usb-phy1::radio, path=/devices/pci0000:00/0000:00:1d.0/usb2/2-1/2-1.6/2-1.6:1.0/leds/rt73usb-phy1::radio, act=add
hotplug: name=rt73usb-phy1::assoc, path=/devices/pci0000:00/0000:00:1d.0/usb2/2-1/2-1.6/2-1.6:1.0/leds/rt73usb-phy1::assoc, act=add
hotplug: name=rt73usb-phy1::quality, path=/devices/pci0000:00/0000:00:1d.0/usb2/2-1/2-1.6/2-1.6:1.0/leds/rt73usb-phy1::quality, act=add
hotplug: name=rt73usb, path=/bus/usb/drivers/rt73usb, act=add
\end{lstlisting}
It's the format of {\tt "name=\%DEVICENAME\%, path=\%DEVPATH\%, act=\%ACTION\%"}.\\
Tells us loading module 
\begin{itemize}
    \item {\tt rfkill}
    \item {\tt cfg80211}
    \item {\tt mac80211}
    \item {\tt rt2x00lib}
    \item {\tt rt2x00usb}
    \item {\tt rt2500lib}
    \item {\tt rt2500usb}
    \item {\tt crc\_itu\_t}
    \item {\tt rt73usb}
    \item ...
\end{itemize}
and a new class {\tt ieee80211}, and device {\tt wlan0} and 4 queues {\tt rx-0 .. rx-3}, and so forth.\\
Check {\tt /sys/devices/pci0000:00/0000:00:1d.0/usb2/2-1/2-1.6/2-1.6:1.0/net/wlan0}:
\begin{lstlisting}
% pwd
/sys/devices/pci0000:00/0000:00:1d.0/usb2/2-1/2-1.6/2-1.6:1.0/net/wlan0
% ll
total 0
15397 -r--r--r-- 1 root root 4.0K Mar 24 22:21 address
15136 -r--r--r-- 1 root root 4.0K Mar 24 22:21 addr_len
15398 -r--r--r-- 1 root root 4.0K Mar 24 22:21 broadcast
15399 -r--r--r-- 1 root root 4.0K Mar 24 22:21 carrier
15135 lrwxrwxrwx 1 root root    0 Mar 24 22:21 device -> ../../../2-1.6:1.0
15137 -r--r--r-- 1 root root 4.0K Mar 24 22:21 dev_id
15402 -r--r--r-- 1 root root 4.0K Mar 24 22:21 dormant
15401 -r--r--r-- 1 root root 4.0K Mar 24 22:21 duplex
15394 -r--r--r-- 1 root root 4.0K Mar 24 22:21 features
15405 -rw-r--r-- 1 root root 4.0K Mar 24 22:21 flags
15138 -rw-r--r-- 1 root root 4.0K Mar 24 22:21 ifalias
15140 -r--r--r-- 1 root root 4.0K Mar 24 22:21 ifindex
15139 -r--r--r-- 1 root root 4.0K Mar 24 22:21 iflink
15396 -r--r--r-- 1 root root 4.0K Mar 24 22:21 link_mode
15404 -rw-r--r-- 1 root root 4.0K Mar 24 22:21 mtu
15403 -r--r--r-- 1 root root 4.0K Mar 24 22:21 operstate
15480 lrwxrwxrwx 1 root root    0 Mar 24 22:21 phy80211 -> ../../ieee80211/phy3
15459 drwxr-xr-x 2 root root    0 Mar 24 22:21 power
15467 drwxr-xr-x 6 root root    0 Mar 24 22:21 queues
15400 -r--r--r-- 1 root root 4.0K Mar 24 22:21 speed
15407 drwxr-xr-x 2 root root    0 Mar 24 22:21 statistics
15133 lrwxrwxrwx 1 root root    0 Mar 24 22:21 subsystem -> ../../../../../../../../../class/net
15406 -rw-r--r-- 1 root root 4.0K Mar 24 22:21 tx_queue_len
15395 -r--r--r-- 1 root root 4.0K Mar 24 22:21 type
15132 -rw-r--r-- 1 root root 4.0K Mar 24 22:21 uevent
15431 drwxr-xr-x 2 root root    0 Mar 24 22:21 wireless
\end{lstlisting}
Check modules again:
\begin{lstlisting}
% lsmod | egrep "(rt|usb|80211)"
rt73usb                16967  0 
crc_itu_t               1235  1 rt73usb
rt2500usb              14161  0 
rt2x00usb               7729  2 rt73usb,rt2500usb
rt2x00lib              30240  3 rt73usb,rt2500usb,rt2x00usb
mac80211              188716  2 rt2x00usb,rt2x00lib
cfg80211              110951  2 rt2x00lib,mac80211
rfkill                 13652  1 cfg80211
xfrm4_mode_transport      981  0 
xfrm6_mode_transport     1013  0 
parport_pc             18037  0 
parport                26215  2 ppdev,parport_pc
iTCO_vendor_support     2070  1 iTCO_wdt
\end{lstlisting}
All added automatically.\\
Up {\tt wlan0}:
\begin{lstlisting}
% sudo ifconfig wlan0 up
hotplug: name=2-1.6:1.0, path=/devices/pci0000:00/0000:00:1d.0/usb2/2-1/2-1.6/2-1.6:1.0/firmware/2-1.6:1.0, act=add
hotplug: name=2-1.6:1.0, path=/devices/pci0000:00/0000:00:1d.0/usb2/2-1/2-1.6/2-1.6:1.0/firmware/2-1.6:1.0, act=remove
% ifconfig wlan0 
wlan0     Link encap:Ethernet  HWaddr 00:23:CD:1E:9E:E5  
          UP BROADCAST MULTICAST  MTU:1500  Metric:1
          RX packets:0 errors:0 dropped:0 overruns:0 frame:0
          TX packets:0 errors:0 dropped:0 overruns:0 carrier:0
          collisions:0 txqueuelen:1000 
          RX bytes:0 (0.0 b)  TX bytes:0 (0.0 b)
\end{lstlisting}
During the process, automatically download wireless firmware to the hardware, 
which is {\tt /lib/firmware/rt73.bin}.
\subsection{remove wireless card}
Plug-out the usb wireless card:
\begin{lstlisting}
hotplug: name=rt73usb-phy1::assoc, path=/devices/pci0000:00/0000:00:1d.0/usb2/2-1/2-1.6/2-1.6:1.0/leds/rt73usb-phy1::assoc, act=remove
hotplug: name=rt73usb-phy1::radio, path=/devices/pci0000:00/0000:00:1d.0/usb2/2-1/2-1.6/2-1.6:1.0/leds/rt73usb-phy1::radio, act=remove
hotplug: name=rx-0, path=/devices/pci0000:00/0000:00:1d.0/usb2/2-1/2-1.6/2-1.6:1.0/net/wlan0/queues/rx-0, act=remove
hotplug: name=rx-1, path=/devices/pci0000:00/0000:00:1d.0/usb2/2-1/2-1.6/2-1.6:1.0/net/wlan0/queues/rx-1, act=remove
hotplug: name=rx-2, path=/devices/pci0000:00/0000:00:1d.0/usb2/2-1/2-1.6/2-1.6:1.0/net/wlan0/queues/rx-2, act=remove
hotplug: name=rx-3, path=/devices/pci0000:00/0000:00:1d.0/usb2/2-1/2-1.6/2-1.6:1.0/net/wlan0/queues/rx-3, act=remove
hotplug: name=wlan0, path=/devices/pci0000:00/0000:00:1d.0/usb2/2-1/2-1.6/2-1.6:1.0/net/wlan0, act=remove
hotplug: name=rfkill0, path=/devices/pci0000:00/0000:00:1d.0/usb2/2-1/2-1.6/2-1.6:1.0/ieee80211/phy1/rfkill0, act=remove
hotplug: name=phy1, path=/devices/pci0000:00/0000:00:1d.0/usb2/2-1/2-1.6/2-1.6:1.0/ieee80211/phy1, act=remove
hotplug: name=2-1.6:1.0, path=/devices/pci0000:00/0000:00:1d.0/usb2/2-1/2-1.6/2-1.6:1.0, act=remove
hotplug: name=2-1.6, path=/devices/pci0000:00/0000:00:1d.0/usb2/2-1/2-1.6, act=remove

% ifconfig wlan0 
wlan0: error fetching interface information: Device not found
\end{lstlisting}
{\tt wlan0} and respective dirs under /sys/devices all disappear.
\subsection{plug usb hard disk}
{\tt usb\_storage} is auto added, and other checkpoints are the same as above.\\
Also could see:
\begin{lstlisting}
hotplug: name=sdb7, path=/devices/pci0000:00/0000:00:1d.0/usb2/2-1/2-1.6/2-1.6:1.0/host6/target6:0:0/6:0:0:0/block/sdb/sdb7, act=add
hotplug: name=ext2, path=/module/ext2, act=add
hotplug: name=ext2_inode_cache, path=/kernel/slab/ext2_inode_cache, act=add
hotplug: name=fat, path=/module/fat, act=add
hotplug: name=fat_cache, path=/kernel/slab/fat_cache, act=add
hotplug: name=fat_inode_cache, path=/kernel/slab/fat_inode_cache, act=add
hotplug: name=vfat, path=/module/vfat, act=add
hotplug: name=hfs, path=/module/hfs, act=add
hotplug: name=hfs_inode_cache, path=/kernel/slab/hfs_inode_cache, act=add
hotplug: name=hfsplus, path=/module/hfsplus, act=add
hotplug: name=hfsplus_icache, path=/kernel/slab/hfsplus_icache, act=add
hotplug: name=nls_utf8, path=/module/nls_utf8, act=add
hotplug: name=8:22-fuseblk, path=/devices/virtual/bdi/8:22-fuseblk, act=add
\end{lstlisting}
Auto load {\tt ext2}.\\
And new device nodes auto mounted to {\tt /media/}
\begin{lstlisting}
525479 brw-r--r-- 1 root root 8, 16 Mar 24 22:30 /dev/sdb
525576 brw-r--r-- 1 root root 8, 17 Mar 24 22:30 /dev/sdb1
525577 brw-r--r-- 1 root root 8, 18 Mar 24 22:30 /dev/sdb2
525578 brw-r--r-- 1 root root 8, 21 Mar 24 22:30 /dev/sdb5
525579 brw-r--r-- 1 root root 8, 22 Mar 24 22:30 /dev/sdb6
525580 brw-r--r-- 1 root root 8, 23 Mar 24 22:30 /dev/sdb7

% ll /media
total 84K
2 dr-xr-xr-x. 24 root root 4.0K Dec 25 17:34 sdb1
5 drwxrwxrwx   1 root root 8.0K Feb 14 05:59 sdb6
1 drwxr-xr-x  15 root root  16K Dec 31  1969 sdb7
\end{lstlisting}
{\tt sdb2} is extented partition, {\tt sdb5} is {\tt swap}, only
{\tt sdb1 sdb6 sdb7} for storage. \\
Checkpoints:
\begin{itemize}
    \item new directories in {\tt /media}, for storage partition. Here only 
          {\tt sdb1 sdb6 sdb7}, it's right;
    \item new device nodes in {\tt /dev/}, should be all partitions of the disk. 
          Here are five in total, right;
    \item log messages on terminal and {\tt /var/log/messages};
\end{itemize}
After plug-out, respective dirs under {\tt /media} are deleted:
\begin{lstlisting}
% ll /media
total 0
%
\end{lstlisting}
%\chapter{hotplug2-1.0}


\chapter{igmpproxy-0.1}
\section{Description}
It is an IGMP snooper/proxy daemon for routing multicast packets across networks.\\
Pretty useful for {\tt IPTV} services.
\section{Installation Hierarchy}
\begin{lstlisting}
/etc/igmpproxy.conf
/usr/sbin/igmpproxy
\end{lstlisting}
And I got a start script from Internet, put it at {\tt /etc/init.d/igmpproxy}.
\section{Warning}
\begin{itemize}
    \item It'll set {\tt net.ipv4.conf.all.mc\_forwarding} and 
          {\tt net.ipv4.conf.ethx.mc\_forwarding} to 1 implicitly;
    \item \href{http://f0g.is-programmer.com/posts/5458.html}{Here} for some clue;
\end{itemize}
%\chapter{igmpproxy-0.1}


\chapter{iperf}
\label{iperf}
\section{Description}
{\tt iperf} is a tool to measure max TCP bandwidth.
It can test either TCP or UDP throughput.  To perform an iperf test the user must 
establish both a server (to discard traffic) and a client (to generate traffic).\\\\
\section{Installation Hierarchy}
\begin{lstlisting}
usr/bin/iperf
\end{lstlisting}
\section{Warning}
\begin{itemize}
    \item Server is always {\tt LISTEN} with TCP, so if set {\tt -d} on client, 
          when testing done, both {\tt LISTEN}, use {\tt netstat -atnp} to check;
    \item Server is master, specifying the port and the protocol (TCP or UDP);
    \item Server default port is 5001, while using TCP;
    \item Daily options:
    \begin{itemize}
        \item {\tt -s} as server;
        \item {\tt -c host\_ip} as client;
        \item {\tt -i} seconds between periodic bandwidth reports;
        \item {\tt -w} window size (socket buffer size);
        \item {\tt -u} UDP instead of TCP;
        \item {\tt -p} port specification;
        \item {\tt -d} do a bidirectional test simultaneously.
    \end{itemize}
\end{itemize}
\section{Test}
\subsection{Help info \textcolor{green}{[pass]}}
\subsection{Mismatching port or protocol \textcolor{green}{[pass]}}
Start {\tt iperf} server on the board using port 5432 and UDP:
\begin{lstlisting}
root@localhost:/root> iperf -s -i 1 -w 1M -u -p 5432
------------------------------------------------------------
Server listening on UDP port 5432
Receiving 1470 byte datagrams
UDP buffer size:  216 KByte (WARNING: requested 1.00 MByte)
------------------------------------------------------------
\end{lstlisting}
Client on the laptop uses port 5432 but TCP:
\begin{lstlisting}
% iperf -c 128.224.165.247 -i 1 -w 1M -p 5432
connect failed: Connection refused
\end{lstlisting}
Client uses UDP but port 5001:
\begin{lstlisting}
% iperf -c 128.224.165.247 -i 1 -w 1M -u
------------------------------------------------------------
Client connecting to 128.224.165.247, UDP port 5001
Sending 1470 byte datagrams
UDP buffer size:  256 KByte (WARNING: requested 1.00 MByte)
------------------------------------------------------------
[  3] local 128.224.158.134 port 46779 connected with 128.224.165.247 port 5001
[ ID] Interval       Transfer     Bandwidth
[  3]  0.0- 1.0 sec  28.0 GBytes   241 Gbits/sec
[  3]  1.0- 2.0 sec  8.00 GBytes  68.7 Gbits/sec
[  3]  2.0- 3.0 sec  8.00 GBytes  68.7 Gbits/sec
[  3]  3.0- 4.0 sec  8.00 GBytes  68.7 Gbits/sec
[  3]  4.0- 5.0 sec  4.00 GBytes  34.4 Gbits/sec
[  3]  5.0- 6.0 sec  8.00 GBytes  68.7 Gbits/sec
[  3]  6.0- 7.0 sec  8.00 GBytes  68.7 Gbits/sec
[  3]  7.0- 8.0 sec  8.00 GBytes  68.7 Gbits/sec
[  3]  8.0- 9.0 sec  8.00 GBytes  68.7 Gbits/sec
[  3]  9.0-10.0 sec  4.00 GBytes  34.4 Gbits/sec
[  3]  0.0-10.0 sec  92.0 GBytes  78.9 Gbits/sec
[  3] Sent 893 datagrams
read failed: Connection refused
[  3] WARNING: did not receive ack of last datagram after 2 tries.
\end{lstlisting}
Notice that read failed and there's no server report, and 
the server terminal has no output.\\\\
Set to match, on the client:
\begin{lstlisting}
% iperf -c 128.224.165.247 -i 1 -w 1M -p 5432 -u
------------------------------------------------------------
Client connecting to 128.224.165.247, UDP port 5432
Sending 1470 byte datagrams
UDP buffer size:  256 KByte (WARNING: requested 1.00 MByte)
------------------------------------------------------------
[  3] local 128.224.158.134 port 42134 connected with 128.224.165.247 port 5432
[ ID] Interval       Transfer     Bandwidth
[  3]  0.0- 1.0 sec   129 KBytes  1.06 Mbits/sec
[  3]  1.0- 2.0 sec   128 KBytes  1.05 Mbits/sec
[  3]  2.0- 3.0 sec   128 KBytes  1.05 Mbits/sec
[  3]  3.0- 4.0 sec   128 KBytes  1.05 Mbits/sec
[  3]  4.0- 5.0 sec   128 KBytes  1.05 Mbits/sec
[  3]  5.0- 6.0 sec   128 KBytes  1.05 Mbits/sec
[  3]  6.0- 7.0 sec   129 KBytes  1.06 Mbits/sec
[  3]  7.0- 8.0 sec   128 KBytes  1.05 Mbits/sec
[  3]  8.0- 9.0 sec   128 KBytes  1.05 Mbits/sec
[  3]  9.0-10.0 sec   128 KBytes  1.05 Mbits/sec
[  3]  0.0-10.0 sec  1.25 MBytes  1.05 Mbits/sec
[  3] Sent 893 datagrams
[  3] Server Report:
[  3]  0.0-18.0 sec  1.25 MBytes   584 Kbits/sec   8.938 ms    0/  893 (0%)
\end{lstlisting}
On server:
\begin{lstlisting}
[  3] local 128.224.165.247 port 5432 connected with 128.224.158.134 port 42134
[ ID] Interval       Transfer     Bandwidth        Jitter   Lost/Total Datagrams
[  3]  0.0- 1.0 sec  70.3 KBytes   576 Kbits/sec   8.534 ms    0/   49 (0%)
[  3]  1.0- 2.0 sec  71.8 KBytes   588 Kbits/sec   8.920 ms    0/   50 (0%)
[  3]  2.0- 3.0 sec  70.3 KBytes   576 Kbits/sec   8.937 ms    0/   49 (0%)
[  3]  3.0- 4.0 sec  71.8 KBytes   588 Kbits/sec   8.939 ms    0/   50 (0%)
[  3]  4.0- 5.0 sec  71.8 KBytes   588 Kbits/sec   8.938 ms    0/   50 (0%)
[  3]  5.0- 6.0 sec  70.3 KBytes   576 Kbits/sec   8.939 ms    0/   49 (0%)
[  3]  6.0- 7.0 sec  71.8 KBytes   588 Kbits/sec   8.939 ms    0/   50 (0%)
[  3]  7.0- 8.0 sec  70.3 KBytes   576 Kbits/sec   8.939 ms    0/   49 (0%)
[  3]  8.0- 9.0 sec  71.8 KBytes   588 Kbits/sec   8.939 ms    0/   50 (0%)
[  3]  9.0-10.0 sec  71.8 KBytes   588 Kbits/sec   8.940 ms    0/   50 (0%)
[  3] 10.0-11.0 sec  70.3 KBytes   576 Kbits/sec   8.939 ms    0/   49 (0%)
[  3] 11.0-12.0 sec  71.8 KBytes   588 Kbits/sec   9.005 ms    0/   50 (0%)
[  3] 12.0-13.0 sec  71.8 KBytes   588 Kbits/sec   8.939 ms    0/   50 (0%)
[  3] 13.0-14.0 sec  70.3 KBytes   576 Kbits/sec   8.945 ms    0/   49 (0%)
[  3] 14.0-15.0 sec  71.8 KBytes   588 Kbits/sec   8.940 ms    0/   50 (0%)
[  3] 15.0-16.0 sec  70.3 KBytes   576 Kbits/sec   8.940 ms    0/   49 (0%)
[  3] 16.0-17.0 sec  71.8 KBytes   588 Kbits/sec   8.940 ms    0/   50 (0%)
[  3]  0.0-18.0 sec  1.25 MBytes   584 Kbits/sec   8.939 ms    0/  893 (0%)
\end{lstlisting}
\subsection{TCP or UDP state \textcolor{green}{[pass]}}
Server with TCP, no data transfer, state is LISTEN:
\begin{lstlisting}
root@localhost:/root> netstat -atnp | egrep "(5432|State)"
Proto Recv-Q Send-Q Local Address Foreign Address State PID/Program
tcp        0      0 0.0.0.0:5432  0.0.0.0:*       LISTEN 1013/iperf
\end{lstlisting}
With UDP, no data transfer, no state:
\begin{lstlisting}
root@localhost:/root> netstat -aunp | egrep "(5432|State)"
Proto Recv-Q Send-Q Local Address Foreign Address State PID/Program
udp        0      0 0.0.0.0:5432  0.0.0.0:*             977/iperf
\end{lstlisting}
With TCP, transferring:
\begin{lstlisting}
root@localhost:/root> netstat -atnp | egrep "(5432|State)"
Proto Recv-Q Send-Q Local Address        Foreign Address       State       PID/Program
tcp        0      0 0.0.0.0:5432         0.0.0.0:*             LISTEN      1013/iperf
tcp        0      0 128.224.165.247:5432 128.224.158.134:53334 ESTABLISHED 1013/iperf
\end{lstlisting}
With UDP, transferring:
\begin{lstlisting}
root@localhost:/root> netstat -aunp | egrep "(5432|State)"
Proto Recv-Q Send-Q Local Address        Foreign Address       State       PID/Program
udp        0      0 0.0.0.0:5432         0.0.0.0:*                         1047/iperf
udp        0      0 128.224.165.247:5432 128.224.158.134:38557 ESTABLISHED 1047/iperf
\end{lstlisting}
\subsection{Full-duplex, max bandwidth, board as server, UDP \textcolor{green}{[pass]}}
Server:
\begin{lstlisting}
root@localhost:/root> iperf -s -i 1 -w 1M -p 5432 -u
------------------------------------------------------------
Server listening on UDP port 5432
Receiving 1470 byte datagrams
UDP buffer size:  216 KByte (WARNING: requested 1.00 MByte)
------------------------------------------------------------
[  3] local 128.224.165.247 port 5432 connected with 128.224.158.134 port 56511
------------------------------------------------------------
Client connecting to 128.224.158.134, UDP port 5432
Sending 1470 byte datagrams
UDP buffer size:  216 KByte (WARNING: requested 1.00 MByte)
------------------------------------------------------------
[  5] local 128.224.165.247 port 58177 connected with 128.224.158.134 port 5432
[ ID] Interval       Transfer     Bandwidth        Jitter   Lost/Total Datagrams
[  3]  0.0- 1.0 sec  70.3 KBytes   576 Kbits/sec   8.533 ms    0/   49 (0%)
[  5]  0.0- 1.0 sec   128 KBytes  1.05 Mbits/sec
[  3]  1.0- 2.0 sec  71.8 KBytes   588 Kbits/sec   8.923 ms    0/   50 (0%)
[  5]  1.0- 2.0 sec   126 KBytes  1.03 Mbits/sec
[  3]  2.0- 3.0 sec  70.3 KBytes   576 Kbits/sec   8.938 ms    0/   49 (0%)
[  5]  2.0- 3.0 sec   129 KBytes  1.06 Mbits/sec
[  3]  3.0- 4.0 sec  71.8 KBytes   588 Kbits/sec   8.940 ms    0/   50 (0%)
[  5]  3.0- 4.0 sec   128 KBytes  1.05 Mbits/sec
[  3]  4.0- 5.0 sec  71.8 KBytes   588 Kbits/sec   8.939 ms    0/   50 (0%)
[  5]  4.0- 5.0 sec   129 KBytes  1.06 Mbits/sec
[  3]  5.0- 6.0 sec  70.3 KBytes   576 Kbits/sec   8.938 ms    0/   49 (0%)
[  5]  5.0- 6.0 sec   128 KBytes  1.05 Mbits/sec
[  3]  6.0- 7.0 sec  71.8 KBytes   588 Kbits/sec   8.941 ms    0/   50 (0%)
[  5]  6.0- 7.0 sec   129 KBytes  1.06 Mbits/sec
[  3]  7.0- 8.0 sec  70.3 KBytes   576 Kbits/sec   8.937 ms    0/   49 (0%)
[  5]  7.0- 8.0 sec   126 KBytes  1.03 Mbits/sec
[  3]  8.0- 9.0 sec  71.8 KBytes   588 Kbits/sec   8.938 ms    0/   50 (0%)
[  5]  8.0- 9.0 sec   128 KBytes  1.05 Mbits/sec
[  5]  9.0-10.0 sec   128 KBytes  1.05 Mbits/sec
[  5]  0.0-10.0 sec  1.25 MBytes  1.05 Mbits/sec
[  5] Sent 892 datagrams
[  3]  9.0-10.0 sec  71.8 KBytes   588 Kbits/sec   8.941 ms    0/   50 (0%)
[  5] Server Report:
[  5]  0.0- 5.6 sec  1.25 MBytes  1.88 Mbits/sec   4.881 ms    0/  891 (0%)
[  5]  0.0- 5.6 sec  1 datagrams received out-of-order
[  3] 10.0-11.0 sec  70.3 KBytes   576 Kbits/sec   8.938 ms    0/   49 (0%)
[  3] 11.0-12.0 sec  71.8 KBytes   588 Kbits/sec   8.940 ms    0/   50 (0%)
[  3] 12.0-13.0 sec  71.8 KBytes   588 Kbits/sec   8.939 ms    0/   50 (0%)
[  3] 13.0-14.0 sec  70.3 KBytes   576 Kbits/sec   8.938 ms    0/   49 (0%)
[  3] 14.0-15.0 sec  71.8 KBytes   588 Kbits/sec   8.941 ms    0/   50 (0%)
[  3] 15.0-16.0 sec  70.3 KBytes   576 Kbits/sec   8.940 ms    0/   49 (0%)
[  3] 16.0-17.0 sec  71.8 KBytes   588 Kbits/sec   8.942 ms    0/   50 (0%)
[  3]  0.0-18.0 sec  1.25 MBytes   584 Kbits/sec   8.938 ms    0/  893 (0%)
\end{lstlisting}
Client:
\begin{lstlisting}
% iperf -c 128.224.165.247 -i 1 -w 1M -p 5432 -u -d
------------------------------------------------------------
Server listening on UDP port 5432
Receiving 1470 byte datagrams
UDP buffer size:  256 KByte (WARNING: requested 1.00 MByte)
------------------------------------------------------------
------------------------------------------------------------
Client connecting to 128.224.165.247, UDP port 5432
Sending 1470 byte datagrams
UDP buffer size:  256 KByte (WARNING: requested 1.00 MByte)
------------------------------------------------------------
[  4] local 128.224.158.134 port 56511 connected with 128.224.165.247 port 5432
[  3] local 128.224.158.134 port 5432 connected with 128.224.165.247 port 58177
[ ID] Interval       Transfer     Bandwidth
[  4]  0.0- 1.0 sec   129 KBytes  1.06 Mbits/sec
[  3]  0.0- 1.0 sec   228 KBytes  1.87 Mbits/sec   4.757 ms    0/  159 (0%)
[  4]  1.0- 2.0 sec   128 KBytes  1.05 Mbits/sec
[  3]  1.0- 2.0 sec   228 KBytes  1.87 Mbits/sec   5.018 ms    0/  159 (0%)
[  4]  2.0- 3.0 sec   128 KBytes  1.05 Mbits/sec
[  3]  2.0- 3.0 sec   233 KBytes  1.91 Mbits/sec   4.825 ms    0/  162 (0%)
[  4]  3.0- 4.0 sec   128 KBytes  1.05 Mbits/sec
[  3]  3.0- 4.0 sec   230 KBytes  1.88 Mbits/sec   4.746 ms    0/  160 (0%)
[  4]  4.0- 5.0 sec   128 KBytes  1.05 Mbits/sec
[  3]  4.0- 5.0 sec   228 KBytes  1.87 Mbits/sec   5.008 ms    0/  159 (0%)
[  3]  0.0- 5.6 sec  1.25 MBytes  1.88 Mbits/sec   4.882 ms    0/  891 (0%)
[  3]  0.0- 5.6 sec  1 datagrams received out-of-order
[  4]  5.0- 6.0 sec   128 KBytes  1.05 Mbits/sec
[  4]  6.0- 7.0 sec   129 KBytes  1.06 Mbits/sec
[  4]  7.0- 8.0 sec   128 KBytes  1.05 Mbits/sec
[  4]  8.0- 9.0 sec   128 KBytes  1.05 Mbits/sec
[  4]  9.0-10.0 sec   128 KBytes  1.05 Mbits/sec
[  4]  0.0-10.0 sec  1.25 MBytes  1.05 Mbits/sec
[  4] Sent 893 datagrams
[  4] Server Report:
[  4]  0.0-18.0 sec  1.25 MBytes   584 Kbits/sec   8.938 ms    0/  893 (0%)
\end{lstlisting}
Ports match.
\subsection{Full-duplex, max bandwidth, board as server, TCP \textcolor{green}{[pass]}}
Server:
\begin{lstlisting}
root@localhost:/root> iperf -s -i 1 -w 1M -p 5432   
------------------------------------------------------------
Server listening on TCP port 5432
TCP window size:  216 KByte (WARNING: requested 1.00 MByte)
------------------------------------------------------------
[  4] local 128.224.165.247 port 5432 connected with 128.224.158.134 port 36917
------------------------------------------------------------
Client connecting to 128.224.158.134, TCP port 5432
TCP window size:  216 KByte (WARNING: requested 1.00 MByte)
------------------------------------------------------------
[  6] local 128.224.165.247 port 51729 connected with 128.224.158.134 port 5432
[ ID] Interval       Transfer     Bandwidth
[  4]  0.0- 1.0 sec  3.19 MBytes  26.7 Mbits/sec
[  6]  0.0- 1.0 sec  6.12 MBytes  51.4 Mbits/sec
[  4]  1.0- 2.0 sec  3.66 MBytes  30.7 Mbits/sec
[  6]  1.0- 2.0 sec  6.12 MBytes  51.4 Mbits/sec
[  4]  2.0- 3.0 sec  4.30 MBytes  36.1 Mbits/sec
[  6]  2.0- 3.0 sec  6.12 MBytes  51.4 Mbits/sec
[  4]  3.0- 4.0 sec  5.11 MBytes  42.9 Mbits/sec
[  6]  3.0- 4.0 sec  6.00 MBytes  50.3 Mbits/sec
[  4]  4.0- 5.0 sec  5.54 MBytes  46.5 Mbits/sec
[  6]  4.0- 5.0 sec  6.12 MBytes  51.4 Mbits/sec
[  4]  5.0- 6.0 sec  5.68 MBytes  47.6 Mbits/sec
[  6]  5.0- 6.0 sec  5.88 MBytes  49.3 Mbits/sec
[  4]  6.0- 7.0 sec  5.82 MBytes  48.8 Mbits/sec
[  6]  6.0- 7.0 sec  5.75 MBytes  48.2 Mbits/sec
[  4]  7.0- 8.0 sec  5.93 MBytes  49.7 Mbits/sec
[  6]  7.0- 8.0 sec  5.62 MBytes  47.2 Mbits/sec
[  4]  8.0- 9.0 sec  6.04 MBytes  50.7 Mbits/sec
[  6]  8.0- 9.0 sec  5.50 MBytes  46.1 Mbits/sec
[  4]  9.0-10.0 sec  6.00 MBytes  50.3 Mbits/sec
Waiting for server threads to complete. Interrupt again to force quit.
[  6]  9.0-10.0 sec  5.75 MBytes  48.2 Mbits/sec
[  6]  0.0-10.0 sec  59.1 MBytes  49.5 Mbits/sec
[  4] 10.0-11.0 sec  6.19 MBytes  52.0 Mbits/sec
[  4] 11.0-12.0 sec  6.21 MBytes  52.1 Mbits/sec
[  4] 12.0-13.0 sec  6.21 MBytes  52.1 Mbits/sec
[  4] 13.0-14.0 sec  6.21 MBytes  52.1 Mbits/sec
[  4] 14.0-15.0 sec  6.21 MBytes  52.1 Mbits/sec
[  4] 15.0-16.0 sec  6.20 MBytes  52.0 Mbits/sec
[  4] 16.0-17.0 sec  6.21 MBytes  52.1 Mbits/sec
[  4] 17.0-18.0 sec  6.20 MBytes  52.0 Mbits/sec
[  4]  0.0-18.0 sec   101 MBytes  47.0 Mbits/sec
\end{lstlisting}
Client:
\begin{lstlisting}
% iperf -c 128.224.165.247 -i 1 -w 1M -p 5432 -d
------------------------------------------------------------
Server listening on TCP port 5432
TCP window size:  256 KByte (WARNING: requested 1.00 MByte)
------------------------------------------------------------
------------------------------------------------------------
Client connecting to 128.224.165.247, TCP port 5432
TCP window size:  256 KByte (WARNING: requested 1.00 MByte)
------------------------------------------------------------
[  5] local 128.224.158.134 port 36917 connected with 128.224.165.247 port 5432
[  4] local 128.224.158.134 port 5432 connected with 128.224.165.247 port 51729
[ ID] Interval       Transfer     Bandwidth
[  5]  0.0- 1.0 sec  6.00 MBytes  50.3 Mbits/sec
[  4]  0.0- 1.0 sec  11.0 MBytes  92.1 Mbits/sec
[  5]  1.0- 2.0 sec  8.12 MBytes  68.2 Mbits/sec
[  4]  1.0- 2.0 sec  11.0 MBytes  92.1 Mbits/sec
[  5]  2.0- 3.0 sec  9.88 MBytes  82.8 Mbits/sec
[  4]  2.0- 3.0 sec  10.9 MBytes  91.4 Mbits/sec
[  5]  3.0- 4.0 sec  10.5 MBytes  88.1 Mbits/sec
[  4]  3.0- 4.0 sec  10.4 MBytes  87.5 Mbits/sec
[  5]  4.0- 5.0 sec  10.8 MBytes  90.2 Mbits/sec
[  4]  4.0- 5.0 sec  9.94 MBytes  83.4 Mbits/sec
[  4]  0.0- 5.6 sec  59.1 MBytes  88.8 Mbits/sec
[  5]  5.0- 6.0 sec  11.0 MBytes  92.3 Mbits/sec
[  5]  6.0- 7.0 sec  11.1 MBytes  93.3 Mbits/sec
[  5]  7.0- 8.0 sec  11.1 MBytes  93.3 Mbits/sec
[  5]  8.0- 9.0 sec  11.1 MBytes  93.3 Mbits/sec
[  5]  9.0-10.0 sec  11.2 MBytes  94.4 Mbits/sec
[  5]  0.0-10.0 sec   101 MBytes  84.6 Mbits/sec
\end{lstlisting}
\subsection{Full-duplex, max bandwidth, board as client, UDP \textcolor{green}{[pass]}}
Server:
\begin{lstlisting}
% iperf -s -i 1 -w 1M -p 5432 -u
------------------------------------------------------------
Server listening on UDP port 5432
Receiving 1470 byte datagrams
UDP buffer size:  256 KByte (WARNING: requested 1.00 MByte)
------------------------------------------------------------
[  3] local 128.224.158.134 port 5432 connected with 128.224.165.247 port 56050
------------------------------------------------------------
Client connecting to 128.224.165.247, UDP port 5432
Sending 1470 byte datagrams
UDP buffer size:  256 KByte (WARNING: requested 1.00 MByte)
------------------------------------------------------------
[  5] local 128.224.158.134 port 36010 connected with 128.224.165.247 port 5432
[ ID] Interval       Transfer     Bandwidth        Jitter   Lost/Total Datagrams
[  3]  0.0- 1.0 sec   230 KBytes  1.88 Mbits/sec   4.757 ms    0/  160 (0%)
[  5]  0.0- 1.0 sec   129 KBytes  1.06 Mbits/sec
[  3]  1.0- 2.0 sec   228 KBytes  1.87 Mbits/sec   5.009 ms    0/  159 (0%)
[  5]  1.0- 2.0 sec   128 KBytes  1.05 Mbits/sec
[  3]  2.0- 3.0 sec   230 KBytes  1.88 Mbits/sec   4.928 ms    0/  160 (0%)
[  5]  2.0- 3.0 sec   128 KBytes  1.05 Mbits/sec
[  5]  3.0- 4.0 sec   128 KBytes  1.05 Mbits/sec
[  3]  3.0- 4.0 sec   231 KBytes  1.89 Mbits/sec   5.048 ms    0/  161 (0%)
[  5]  4.0- 5.0 sec   128 KBytes  1.05 Mbits/sec
[  3]  4.0- 5.0 sec   230 KBytes  1.88 Mbits/sec   4.985 ms    0/  160 (0%)
[  3]  0.0- 5.6 sec  1.25 MBytes  1.88 Mbits/sec   5.223 ms    0/  892 (0%)
[  5]  5.0- 6.0 sec   128 KBytes  1.05 Mbits/sec
[  5]  6.0- 7.0 sec   129 KBytes  1.06 Mbits/sec
[  5]  7.0- 8.0 sec   128 KBytes  1.05 Mbits/sec
[  5]  8.0- 9.0 sec   128 KBytes  1.05 Mbits/sec
[  5]  9.0-10.0 sec   128 KBytes  1.05 Mbits/sec
[  5]  0.0-10.0 sec  1.25 MBytes  1.05 Mbits/sec
[  5] Sent 893 datagrams
[  5] Server Report:
[  5]  0.0-18.0 sec  1.25 MBytes   584 Kbits/sec   8.942 ms    0/  893 (0%)
\end{lstlisting}
Client:
\begin{lstlisting}
root@localhost:/root> iperf -c 128.224.158.134 -i 1 -w 1M -p 5432 -u -d
------------------------------------------------------------
Client connecting to 128.224.158.134, UDP port 5432
Sending 1470 byte datagrams
UDP buffer size:  216 KByte (WARNING: requested 1.00 MByte)
------------------------------------------------------------
[  3] local 128.224.165.247 port 56050 connected with 128.224.158.134 port 5432
------------------------------------------------------------
Server listening on UDP port 5432
Receiving 1470 byte datagrams
UDP buffer size:  216 KByte (WARNING: requested 1.00 MByte)
------------------------------------------------------------
[  4] local 128.224.165.247 port 5432 connected with 128.224.158.134 port 36010
[ ID] Interval       Transfer     Bandwidth
[  3]  0.0- 1.0 sec   128 KBytes  1.05 Mbits/sec
[  4]  0.0- 1.0 sec  70.3 KBytes   576 Kbits/sec   8.535 ms    0/   49 (0%)
[  3]  1.0- 2.0 sec   128 KBytes  1.05 Mbits/sec
[  4]  1.0- 2.0 sec  71.8 KBytes   588 Kbits/sec   8.923 ms    0/   50 (0%)
[  4]  2.0- 3.0 sec  70.3 KBytes   576 Kbits/sec   8.958 ms    0/   49 (0%)
[  3]  2.0- 3.0 sec   129 KBytes  1.06 Mbits/sec
[  3]  3.0- 4.0 sec   128 KBytes  1.05 Mbits/sec
[  4]  3.0- 4.0 sec  71.8 KBytes   588 Kbits/sec   8.939 ms    0/   50 (0%)
[  3]  4.0- 5.0 sec   126 KBytes  1.03 Mbits/sec
[  4]  4.0- 5.0 sec  71.8 KBytes   588 Kbits/sec   8.937 ms    0/   50 (0%)
[  3]  5.0- 6.0 sec   128 KBytes  1.05 Mbits/sec
[  4]  5.0- 6.0 sec  70.3 KBytes   576 Kbits/sec   8.942 ms    0/   49 (0%)
[  3]  6.0- 7.0 sec   129 KBytes  1.06 Mbits/sec
[  4]  6.0- 7.0 sec  71.8 KBytes   588 Kbits/sec   8.940 ms    0/   50 (0%)
[  3]  7.0- 8.0 sec   128 KBytes  1.05 Mbits/sec
[  4]  7.0- 8.0 sec  70.3 KBytes   576 Kbits/sec   8.937 ms    0/   49 (0%)
[  3]  8.0- 9.0 sec   129 KBytes  1.06 Mbits/sec
[  4]  8.0- 9.0 sec  71.8 KBytes   588 Kbits/sec   8.938 ms    0/   50 (0%)
[  3]  9.0-10.0 sec   126 KBytes  1.03 Mbits/sec
[  3]  0.0-10.0 sec  1.25 MBytes  1.05 Mbits/sec
[  3] Sent 892 datagrams
[  4]  9.0-10.0 sec  71.8 KBytes   588 Kbits/sec   8.938 ms    0/   50 (0%)
[  3] Server Report:
[  3]  0.0- 5.6 sec  1.25 MBytes  1.88 Mbits/sec   5.222 ms    0/  892 (0%)
[  4] 10.0-11.0 sec  70.3 KBytes   576 Kbits/sec   8.941 ms    0/   49 (0%)
[  4] 11.0-12.0 sec  71.8 KBytes   588 Kbits/sec   8.940 ms    0/   50 (0%)
[  4] 12.0-13.0 sec  71.8 KBytes   588 Kbits/sec   8.937 ms    0/   50 (0%)
[  4] 13.0-14.0 sec  70.3 KBytes   576 Kbits/sec   8.941 ms    0/   49 (0%)
[  4] 14.0-15.0 sec  71.8 KBytes   588 Kbits/sec   8.940 ms    0/   50 (0%)
[  4] 15.0-16.0 sec  70.3 KBytes   576 Kbits/sec   8.939 ms    0/   49 (0%)
[  4] 16.0-17.0 sec  71.8 KBytes   588 Kbits/sec   8.939 ms    0/   50 (0%)
[  4]  0.0-18.0 sec  1.25 MBytes   584 Kbits/sec   8.942 ms    0/  893 (0%)
\end{lstlisting}
\subsection{Full-duplex, max bandwidth, board as client, TCP \textcolor{green}{[pass]}}
Server:
\begin{lstlisting}
% iperf -s -i 1 -w 1M -p 5432
------------------------------------------------------------
Server listening on TCP port 5432
TCP window size:  256 KByte (WARNING: requested 1.00 MByte)
------------------------------------------------------------
[  4] local 128.224.158.134 port 5432 connected with 128.224.165.247 port 51728
------------------------------------------------------------
Client connecting to 128.224.165.247, TCP port 5432
TCP window size:  256 KByte (WARNING: requested 1.00 MByte)
------------------------------------------------------------
[  6] local 128.224.158.134 port 36912 connected with 128.224.165.247 port 5432
[ ID] Interval       Transfer     Bandwidth
[  4]  0.0- 1.0 sec  9.84 MBytes  82.5 Mbits/sec
[  6]  0.0- 1.0 sec  10.0 MBytes  83.9 Mbits/sec
[  4]  1.0- 2.0 sec  10.7 MBytes  89.8 Mbits/sec
[  6]  1.0- 2.0 sec  10.9 MBytes  91.2 Mbits/sec
[  4]  2.0- 3.0 sec  10.1 MBytes  85.1 Mbits/sec
[  6]  2.0- 3.0 sec  10.9 MBytes  91.2 Mbits/sec
[  4]  3.0- 4.0 sec  9.60 MBytes  80.5 Mbits/sec
[  6]  3.0- 4.0 sec  10.9 MBytes  91.2 Mbits/sec
[  4]  4.0- 5.0 sec  9.80 MBytes  82.2 Mbits/sec
[  6]  4.0- 5.0 sec  10.8 MBytes  90.2 Mbits/sec
[  4]  0.0- 5.6 sec  55.9 MBytes  84.0 Mbits/sec
[  6]  5.0- 6.0 sec  10.8 MBytes  90.2 Mbits/sec
[  6]  6.0- 7.0 sec  11.1 MBytes  93.3 Mbits/sec
[  6]  7.0- 8.0 sec  11.2 MBytes  94.4 Mbits/sec
[  6]  8.0- 9.0 sec  11.1 MBytes  93.3 Mbits/sec
Waiting for server threads to complete. Interrupt again to force quit.
[  6]  9.0-10.0 sec  11.1 MBytes  93.3 Mbits/sec
[  6]  0.0-10.0 sec   109 MBytes  91.3 Mbits/sec
\end{lstlisting}
Client:
\begin{lstlisting}
root@localhost:/root> iperf -c 128.224.158.134 -i 1 -w 1M -p 5432 -d
------------------------------------------------------------
Server listening on TCP port 5432
TCP window size:  216 KByte (WARNING: requested 1.00 MByte)
------------------------------------------------------------
------------------------------------------------------------
Client connecting to 128.224.158.134, TCP port 5432
TCP window size:  216 KByte (WARNING: requested 1.00 MByte)
------------------------------------------------------------
[  5] local 128.224.165.247 port 51728 connected with 128.224.158.134 port 5432
[  4] local 128.224.165.247 port 5432 connected with 128.224.158.134 port 36912
[ ID] Interval       Transfer     Bandwidth
[  5]  0.0- 1.0 sec  5.25 MBytes  44.0 Mbits/sec
[  4]  0.0- 1.0 sec  5.45 MBytes  45.8 Mbits/sec
[  5]  1.0- 2.0 sec  5.75 MBytes  48.2 Mbits/sec
[  4]  1.0- 2.0 sec  5.82 MBytes  48.8 Mbits/sec
[  5]  2.0- 3.0 sec  6.00 MBytes  50.3 Mbits/sec
[  4]  2.0- 3.0 sec  6.03 MBytes  50.6 Mbits/sec
[  5]  3.0- 4.0 sec  5.88 MBytes  49.3 Mbits/sec
[  4]  3.0- 4.0 sec  6.03 MBytes  50.6 Mbits/sec
[  5]  4.0- 5.0 sec  5.50 MBytes  46.1 Mbits/sec
[  4]  4.0- 5.0 sec  6.05 MBytes  50.7 Mbits/sec
[  5]  5.0- 6.0 sec  5.50 MBytes  46.1 Mbits/sec
[  4]  5.0- 6.0 sec  6.00 MBytes  50.4 Mbits/sec
[  5]  6.0- 7.0 sec  5.25 MBytes  44.0 Mbits/sec
[  4]  6.0- 7.0 sec  6.03 MBytes  50.6 Mbits/sec
[  5]  7.0- 8.0 sec  5.38 MBytes  45.1 Mbits/sec
[  4]  7.0- 8.0 sec  6.00 MBytes  50.3 Mbits/sec
[  5]  8.0- 9.0 sec  5.62 MBytes  47.2 Mbits/sec
[  4]  8.0- 9.0 sec  6.02 MBytes  50.5 Mbits/sec
[  4]  9.0-10.0 sec  5.78 MBytes  48.5 Mbits/sec
[  5]  9.0-10.0 sec  5.62 MBytes  47.2 Mbits/sec
[  5]  0.0-10.0 sec  55.9 MBytes  46.8 Mbits/sec
[  4] 10.0-11.0 sec  6.20 MBytes  52.0 Mbits/sec
[  4] 11.0-12.0 sec  6.21 MBytes  52.1 Mbits/sec
[  4] 12.0-13.0 sec  6.20 MBytes  52.0 Mbits/sec
[  4] 13.0-14.0 sec  6.21 MBytes  52.1 Mbits/sec
[  4] 14.0-15.0 sec  6.21 MBytes  52.1 Mbits/sec
[  4] 15.0-16.0 sec  6.21 MBytes  52.1 Mbits/sec
[  4] 16.0-17.0 sec  6.21 MBytes  52.1 Mbits/sec
[  4] 17.0-18.0 sec  6.21 MBytes  52.1 Mbits/sec
[  4]  0.0-18.0 sec   109 MBytes  50.7 Mbits/sec
\end{lstlisting}
It's obvious that \textcolor{red}{Waiting for server threads to complete. Interrupt again to force quit.}
appears when use TCP. 
\subsection{Half-duplex, max bandwidth, board as server, UDP \textcolor{green}{[pass]}}
Server:
\begin{lstlisting}
root@localhost:/root> iperf -s -i 1 -w 1M -p 5432 -u
------------------------------------------------------------
Server listening on UDP port 5432
Receiving 1470 byte datagrams
UDP buffer size:  216 KByte (WARNING: requested 1.00 MByte)
------------------------------------------------------------
[  3] local 128.224.165.247 port 5432 connected with 128.224.158.134 port 55112
[ ID] Interval       Transfer     Bandwidth        Jitter   Lost/Total Datagrams
[  3]  0.0- 1.0 sec  70.3 KBytes   576 Kbits/sec   8.534 ms    0/   49 (0%)
[  3]  1.0- 2.0 sec  71.8 KBytes   588 Kbits/sec   8.921 ms    0/   50 (0%)
[  3]  2.0- 3.0 sec  70.3 KBytes   576 Kbits/sec   8.938 ms    0/   49 (0%)
[  3]  3.0- 4.0 sec  71.8 KBytes   588 Kbits/sec   8.939 ms    0/   50 (0%)
[  3]  4.0- 5.0 sec  71.8 KBytes   588 Kbits/sec   8.942 ms    0/   50 (0%)
[  3]  5.0- 6.0 sec  70.3 KBytes   576 Kbits/sec   8.941 ms    0/   49 (0%)
[  3]  6.0- 7.0 sec  71.8 KBytes   588 Kbits/sec   8.939 ms    0/   50 (0%)
[  3]  7.0- 8.0 sec  70.3 KBytes   576 Kbits/sec   8.941 ms    0/   49 (0%)
[  3]  8.0- 9.0 sec  71.8 KBytes   588 Kbits/sec   8.938 ms    0/   50 (0%)
[  3]  9.0-10.0 sec  71.8 KBytes   588 Kbits/sec   8.939 ms    0/   50 (0%)
[  3] 10.0-11.0 sec  70.3 KBytes   576 Kbits/sec   8.933 ms    0/   49 (0%)
[  3] 11.0-12.0 sec  71.8 KBytes   588 Kbits/sec   8.940 ms    0/   50 (0%)
[  3] 12.0-13.0 sec  71.8 KBytes   588 Kbits/sec   8.939 ms    0/   50 (0%)
[  3] 13.0-14.0 sec  70.3 KBytes   576 Kbits/sec   8.938 ms    0/   49 (0%)
[  3] 14.0-15.0 sec  71.8 KBytes   588 Kbits/sec   8.940 ms    0/   50 (0%)
[  3] 15.0-16.0 sec  70.3 KBytes   576 Kbits/sec   8.941 ms    0/   49 (0%)
[  3] 16.0-17.0 sec  71.8 KBytes   588 Kbits/sec   8.941 ms    0/   50 (0%)
[  3]  0.0-18.0 sec  1.25 MBytes   584 Kbits/sec   8.939 ms    0/  893 (0%)
------------------------------------------------------------
Client connecting to 128.224.158.134, UDP port 5432
Sending 1470 byte datagrams
UDP buffer size:  216 KByte (WARNING: requested 1.00 MByte)
------------------------------------------------------------
[  3] local 128.224.165.247 port 36815 connected with 128.224.158.134 port 5432
[  3]  0.0- 1.0 sec   126 KBytes  1.03 Mbits/sec
[  3]  1.0- 2.0 sec   129 KBytes  1.06 Mbits/sec
[  3]  2.0- 3.0 sec   128 KBytes  1.05 Mbits/sec
[  3]  3.0- 4.0 sec   129 KBytes  1.06 Mbits/sec
[  3]  4.0- 5.0 sec   128 KBytes  1.05 Mbits/sec
[  3]  5.0- 6.0 sec   126 KBytes  1.03 Mbits/sec
[  3]  6.0- 7.0 sec   128 KBytes  1.05 Mbits/sec
[  3]  7.0- 8.0 sec   129 KBytes  1.06 Mbits/sec
[  3]  8.0- 9.0 sec   128 KBytes  1.05 Mbits/sec
[  3]  9.0-10.0 sec   129 KBytes  1.06 Mbits/sec
[  3]  0.0-10.0 sec  1.25 MBytes  1.05 Mbits/sec
[  3] Sent 893 datagrams
[  3] Server Report:
[  3]  0.0- 5.6 sec  1.25 MBytes  1.88 Mbits/sec   4.755 ms    0/  892 (0%)
[  3]  0.0- 5.6 sec  1 datagrams received out-of-order
\end{lstlisting}
Client:
\begin{lstlisting}
% iperf -c 128.224.165.247 -i 1 -w 1M -p 5432 -r -u
------------------------------------------------------------
Server listening on UDP port 5432
Receiving 1470 byte datagrams
UDP buffer size:  256 KByte (WARNING: requested 1.00 MByte)
------------------------------------------------------------
------------------------------------------------------------
Client connecting to 128.224.165.247, UDP port 5432
Sending 1470 byte datagrams
UDP buffer size:  256 KByte (WARNING: requested 1.00 MByte)
------------------------------------------------------------
[  4] local 128.224.158.134 port 55112 connected with 128.224.165.247 port 5432
[ ID] Interval       Transfer     Bandwidth
[  4]  0.0- 1.0 sec   129 KBytes  1.06 Mbits/sec
[  4]  1.0- 2.0 sec   128 KBytes  1.05 Mbits/sec
[  4]  2.0- 3.0 sec   128 KBytes  1.05 Mbits/sec
[  4]  3.0- 4.0 sec   128 KBytes  1.05 Mbits/sec
[  4]  4.0- 5.0 sec   128 KBytes  1.05 Mbits/sec
[  4]  5.0- 6.0 sec   128 KBytes  1.05 Mbits/sec
[  4]  6.0- 7.0 sec   129 KBytes  1.06 Mbits/sec
[  4]  7.0- 8.0 sec   128 KBytes  1.05 Mbits/sec
[  4]  8.0- 9.0 sec   128 KBytes  1.05 Mbits/sec
[  4]  9.0-10.0 sec   128 KBytes  1.05 Mbits/sec
[  4]  0.0-10.0 sec  1.25 MBytes  1.05 Mbits/sec
[  4] Sent 893 datagrams
[  4] Server Report:
[  4]  0.0-18.0 sec  1.25 MBytes   584 Kbits/sec   8.938 ms    0/  893 (0%)
[  3] local 128.224.158.134 port 5432 connected with 128.224.165.247 port 36815
[  3]  0.0- 1.0 sec   228 KBytes  1.87 Mbits/sec   4.905 ms    0/  159 (0%)
[  3]  1.0- 2.0 sec   230 KBytes  1.88 Mbits/sec   4.863 ms    0/  160 (0%)
[  3]  2.0- 3.0 sec   231 KBytes  1.89 Mbits/sec   4.979 ms    0/  161 (0%)
[  3]  3.0- 4.0 sec   230 KBytes  1.88 Mbits/sec   4.894 ms    0/  160 (0%)
[  3]  4.0- 5.0 sec   230 KBytes  1.88 Mbits/sec   4.835 ms    0/  160 (0%)
[  3]  0.0- 5.6 sec  1.25 MBytes  1.88 Mbits/sec   4.756 ms    0/  892 (0%)
[  3]  0.0- 5.6 sec  1 datagrams received out-of-order
\end{lstlisting}
\subsection{Half-duplex, max bandwidth, board as server, TCP \textcolor{red}{[bug]}}
Server:
\begin{lstlisting}
root@localhost:/root> iperf -s -i 1 -w 1M -p 5432
------------------------------------------------------------
Server listening on TCP port 5432
TCP window size:  216 KByte (WARNING: requested 1.00 MByte)
------------------------------------------------------------
[  4] local 128.224.165.247 port 5432 connected with 128.224.158.134 port 52622
[ ID] Interval       Transfer     Bandwidth
[  4]  0.0- 1.0 sec  6.19 MBytes  52.0 Mbits/sec
[  4]  1.0- 2.0 sec  6.20 MBytes  52.0 Mbits/sec
[  4]  2.0- 3.0 sec  6.21 MBytes  52.1 Mbits/sec
[  4]  3.0- 4.0 sec  6.20 MBytes  52.0 Mbits/sec
[  4]  4.0- 5.0 sec  6.21 MBytes  52.1 Mbits/sec
[  4]  5.0- 6.0 sec  6.20 MBytes  52.0 Mbits/sec
[  4]  6.0- 7.0 sec  6.21 MBytes  52.1 Mbits/sec
[  4]  7.0- 8.0 sec  6.20 MBytes  52.0 Mbits/sec
[  4]  8.0- 9.0 sec  6.20 MBytes  52.0 Mbits/sec
[  4]  9.0-10.0 sec  6.21 MBytes  52.1 Mbits/sec
[  4] 10.0-11.0 sec  6.20 MBytes  52.0 Mbits/sec
[  4] 11.0-12.0 sec  6.20 MBytes  52.0 Mbits/sec
[  4] 12.0-13.0 sec  6.21 MBytes  52.1 Mbits/sec
[  4] 13.0-14.0 sec  6.20 MBytes  52.0 Mbits/sec
[  4] 14.0-15.0 sec  6.20 MBytes  52.0 Mbits/sec
[  4] 15.0-16.0 sec  6.20 MBytes  52.0 Mbits/sec
[  4] 16.0-17.0 sec  6.20 MBytes  52.0 Mbits/sec
[  4]  0.0-18.0 sec   112 MBytes  52.0 Mbits/sec
------------------------------------------------------------
Client connecting to 128.224.158.134, TCP port 5432
TCP window size:  216 KByte (WARNING: requested 1.00 MByte)
------------------------------------------------------------
[  4] local 128.224.165.247 port 53758 connected with 128.224.158.134 port 5432
[  4]  0.0- 1.0 sec  0.00 (null)s  1340785711332496 Bytes/sec
[  4]  1.0- 2.0 sec  0.00 (null)s  1333707604181760 Bytes/sec
[  4]  2.0- 3.0 sec  0.00 (null)s  1342366258248480 Bytes/sec
[  4]  3.0- 4.0 sec  0.00 (null)s  1343053453015680 Bytes/sec
[  4]  4.0- 5.0 sec  0.00 (null)s  1340270314208520 Bytes/sec
[  4]  5.0- 6.0 sec  0.00 (null)s  1333226567844720 Bytes/sec
[  4]  6.0- 7.0 sec  0.00 (null)s  1334119921042080 Bytes/sec
[  4]  7.0- 8.0 sec  0.00 (null)s  1340613911592120 Bytes/sec
[  4]  8.0- 9.0 sec  0.00 (null)s  1343053453015680 Bytes/sec
Waiting for server threads to complete. Interrupt again to force quit.
[  4]  9.0-10.0 sec  0.00 (null)s  1341266746620960 Bytes/sec
[  4]  0.0-10.0 sec  0.00 (null)s  1338041930805848 Bytes/sec
\end{lstlisting}
When server changes role to client, chaos. And sometimes it only shows 
\textcolor{red}{connect failed: Connection refused} after role changing.
Client:
\begin{lstlisting}
% iperf -c 128.224.165.247 -i 1 -w 1M -p 5432 -r
------------------------------------------------------------
Server listening on TCP port 5432
TCP window size:  256 KByte (WARNING: requested 1.00 MByte)
------------------------------------------------------------
------------------------------------------------------------
Client connecting to 128.224.165.247, TCP port 5432
TCP window size:  256 KByte (WARNING: requested 1.00 MByte)
------------------------------------------------------------
[  5] local 128.224.158.134 port 52622 connected with 128.224.165.247 port 5432
[ ID] Interval       Transfer     Bandwidth
[  5]  0.0- 1.0 sec  11.2 MBytes  94.4 Mbits/sec
[  5]  1.0- 2.0 sec  11.1 MBytes  93.3 Mbits/sec
[  5]  2.0- 3.0 sec  11.1 MBytes  93.3 Mbits/sec
[  5]  3.0- 4.0 sec  11.1 MBytes  93.3 Mbits/sec
[  5]  4.0- 5.0 sec  11.2 MBytes  94.4 Mbits/sec
[  5]  5.0- 6.0 sec  11.1 MBytes  93.3 Mbits/sec
[  5]  6.0- 7.0 sec  11.1 MBytes  93.3 Mbits/sec
[  5]  7.0- 8.0 sec  11.1 MBytes  93.3 Mbits/sec
[  5]  8.0- 9.0 sec  11.1 MBytes  93.3 Mbits/sec
Waiting for server threads to complete. Interrupt again to force quit.
[  5]  9.0-10.0 sec  11.1 MBytes  93.3 Mbits/sec
[  5]  0.0-10.0 sec   112 MBytes  93.6 Mbits/sec
\end{lstlisting}
\subsection{Half-duplex, max bandwidth, board as client, UDP \textcolor{green}{[pass]}}
Server:
\begin{lstlisting}
% iperf -s -i 1 -w 1M -p 5432 -u
------------------------------------------------------------
Server listening on UDP port 5432
Receiving 1470 byte datagrams
UDP buffer size:  256 KByte (WARNING: requested 1.00 MByte)
------------------------------------------------------------
[  3] local 128.224.158.134 port 5432 connected with 128.224.165.247 port 57762
[ ID] Interval       Transfer     Bandwidth        Jitter   Lost/Total Datagrams
[  3]  0.0- 1.0 sec   230 KBytes  1.88 Mbits/sec   4.771 ms    0/  160 (0%)
[  3]  1.0- 2.0 sec   228 KBytes  1.87 Mbits/sec   5.023 ms    0/  159 (0%)
[  3]  2.0- 3.0 sec   230 KBytes  1.88 Mbits/sec   4.980 ms    0/  160 (0%)
[  3]  3.0- 4.0 sec   231 KBytes  1.89 Mbits/sec   5.101 ms    0/  161 (0%)
[  3]  4.0- 5.0 sec   230 KBytes  1.88 Mbits/sec   5.022 ms    0/  160 (0%)
[  3]  0.0- 5.6 sec  1.25 MBytes  1.88 Mbits/sec   5.212 ms    0/  892 (0%)
------------------------------------------------------------
Client connecting to 128.224.165.247, UDP port 5432
Sending 1470 byte datagrams
UDP buffer size:  256 KByte (WARNING: requested 1.00 MByte)
------------------------------------------------------------
[  3] local 128.224.158.134 port 60772 connected with 128.224.165.247 port 5432
[  3]  0.0- 1.0 sec   129 KBytes  1.06 Mbits/sec
[  3]  1.0- 2.0 sec   128 KBytes  1.05 Mbits/sec
[  3]  2.0- 3.0 sec   128 KBytes  1.05 Mbits/sec
[  3]  3.0- 4.0 sec   128 KBytes  1.05 Mbits/sec
[  3]  4.0- 5.0 sec   128 KBytes  1.05 Mbits/sec
[  3]  5.0- 6.0 sec   128 KBytes  1.05 Mbits/sec
[  3]  6.0- 7.0 sec   129 KBytes  1.06 Mbits/sec
[  3]  7.0- 8.0 sec   128 KBytes  1.05 Mbits/sec
[  3]  8.0- 9.0 sec   128 KBytes  1.05 Mbits/sec
[  3]  9.0-10.0 sec   128 KBytes  1.05 Mbits/sec
[  3]  0.0-10.0 sec  1.25 MBytes  1.05 Mbits/sec
[  3] Sent 893 datagrams
[  3] Server Report:
[  3]  0.0-18.0 sec  1.25 MBytes   584 Kbits/sec   8.939 ms    0/  893 (0%)
\end{lstlisting}
Client:
\begin{lstlisting}
root@localhost:/root> iperf -c 128.224.158.134 -i 1 -w 1M -p 5432 -r -u
------------------------------------------------------------
Client connecting to 128.224.158.134, UDP port 5432
Sending 1470 byte datagrams
UDP buffer size:  216 KByte (WARNING: requested 1.00 MByte)
------------------------------------------------------------
[  3] local 128.224.165.247 port 57762 connected with 128.224.158.134 port 5432
------------------------------------------------------------
Server listening on UDP port 5432
Receiving 1470 byte datagrams
UDP buffer size:  216 KByte (WARNING: requested 1.00 MByte)
------------------------------------------------------------
[ ID] Interval       Transfer     Bandwidth
[  3]  0.0- 1.0 sec   128 KBytes  1.05 Mbits/sec
[  3]  1.0- 2.0 sec   128 KBytes  1.05 Mbits/sec
[  3]  2.0- 3.0 sec   129 KBytes  1.06 Mbits/sec
[  3]  3.0- 4.0 sec   128 KBytes  1.05 Mbits/sec
[  3]  4.0- 5.0 sec   126 KBytes  1.03 Mbits/sec
[  3]  5.0- 6.0 sec   128 KBytes  1.05 Mbits/sec
[  3]  6.0- 7.0 sec   129 KBytes  1.06 Mbits/sec
[  3]  7.0- 8.0 sec   128 KBytes  1.05 Mbits/sec
[  3]  8.0- 9.0 sec   129 KBytes  1.06 Mbits/sec
[  3]  9.0-10.0 sec   126 KBytes  1.03 Mbits/sec
[  3]  0.0-10.0 sec  1.25 MBytes  1.05 Mbits/sec
[  3] Sent 892 datagrams
[  3] Server Report:
[  3]  0.0- 5.6 sec  1.25 MBytes  1.88 Mbits/sec   5.212 ms    0/  892 (0%)
[  4] local 128.224.165.247 port 5432 connected with 128.224.158.134 port 60772
[  4]  0.0- 1.0 sec  70.3 KBytes   576 Kbits/sec   8.538 ms    0/   49 (0%)
[  4]  1.0- 2.0 sec  71.8 KBytes   588 Kbits/sec   8.924 ms    0/   50 (0%)
[  4]  2.0- 3.0 sec  70.3 KBytes   576 Kbits/sec   8.937 ms    0/   49 (0%)
[  4]  3.0- 4.0 sec  71.8 KBytes   588 Kbits/sec   8.940 ms    0/   50 (0%)
[  4]  4.0- 5.0 sec  71.8 KBytes   588 Kbits/sec   8.939 ms    0/   50 (0%)
[  4]  5.0- 6.0 sec  70.3 KBytes   576 Kbits/sec   8.934 ms    0/   49 (0%)
[  4]  6.0- 7.0 sec  71.8 KBytes   588 Kbits/sec   8.940 ms    0/   50 (0%)
[  4]  7.0- 8.0 sec  70.3 KBytes   576 Kbits/sec   8.939 ms    0/   49 (0%)
[  4]  8.0- 9.0 sec  71.8 KBytes   588 Kbits/sec   8.939 ms    0/   50 (0%)
[  4]  9.0-10.0 sec  71.8 KBytes   588 Kbits/sec   8.939 ms    0/   50 (0%)
[  4] 10.0-11.0 sec  70.3 KBytes   576 Kbits/sec   8.940 ms    0/   49 (0%)
[  4] 11.0-12.0 sec  71.8 KBytes   588 Kbits/sec   8.939 ms    0/   50 (0%)
[  4] 12.0-13.0 sec  71.8 KBytes   588 Kbits/sec   8.940 ms    0/   50 (0%)
[  4] 13.0-14.0 sec  70.3 KBytes   576 Kbits/sec   8.940 ms    0/   49 (0%)
[  4] 14.0-15.0 sec  71.8 KBytes   588 Kbits/sec   8.940 ms    0/   50 (0%)
[  4] 15.0-16.0 sec  70.3 KBytes   576 Kbits/sec   8.940 ms    0/   49 (0%)
[  4] 16.0-17.0 sec  71.8 KBytes   588 Kbits/sec   8.940 ms    0/   50 (0%)
[  4]  0.0-18.0 sec  1.25 MBytes   584 Kbits/sec   8.940 ms    0/  893 (0%)
\end{lstlisting}
\subsection{Half-duplex, max bandwidth, board as client, TCP \textcolor{red}{[bug]}}
Server:
\begin{lstlisting}
% iperf -s -i 1 -w 1M -p 5432
------------------------------------------------------------
Server listening on TCP port 5432
TCP window size:  256 KByte (WARNING: requested 1.00 MByte)
------------------------------------------------------------
[  4] local 128.224.158.134 port 5432 connected with 128.224.165.247 port 49139
[ ID] Interval       Transfer     Bandwidth
[  4]  0.0- 1.0 sec  11.2 MBytes  93.6 Mbits/sec
[  4]  1.0- 2.0 sec  11.2 MBytes  94.1 Mbits/sec
[  4]  2.0- 3.0 sec  11.2 MBytes  93.8 Mbits/sec
[  4]  3.0- 4.0 sec  11.2 MBytes  94.1 Mbits/sec
[  4]  4.0- 5.0 sec  11.2 MBytes  94.1 Mbits/sec
[  4]  0.0- 5.6 sec  62.5 MBytes  94.0 Mbits/sec
------------------------------------------------------------
Client connecting to 128.224.165.247, TCP port 5432
TCP window size:  256 KByte (WARNING: requested 1.00 MByte)
------------------------------------------------------------
[  4] local 128.224.158.134 port 56924 connected with 128.224.165.247 port 5432
[  4]  0.0- 1.0 sec  295699908518744 bits  2701037667745639701427803326008s/sec
[  4]  1.0- 2.0 sec  293054208472440 bits  2676870817243653427641026981818s/sec
[  4]  2.0- 3.0 sec  291065638614855 bits  2658706448789206664582670265137s/sec
[  4]  3.0- 4.0 sec  292349833836060 bits  2670436786084842911136555056255s/sec
[  4]  4.0- 5.0 sec  293792942847180 bits  2683618703580942505926204854969s/sec
[  4]  5.0- 6.0 sec  293792942847180 bits  2683618703580942505926204854969s/sec
[  4]  6.0- 7.0 sec  293792942847180 bits  2683618703580942505926204854969s/sec
[  4]  7.0- 8.0 sec  293792942847180 bits  2683618703580942505926204854969s/sec
[  4]  8.0- 9.0 sec  293792942847180 bits  2683618703580942505926204854969s/sec
Waiting for server threads to complete. Interrupt again to force quit.
[  4]  0.0-10.0 sec  2934927246525179 bits  2679337382742252316720736684707s/sec
\end{lstlisting}
Client:
\begin{lstlisting}
root@localhost:/root> iperf -c 128.224.158.134 -i 1 -w 1M -p 5432 -r
------------------------------------------------------------
Server listening on TCP port 5432
TCP window size:  216 KByte (WARNING: requested 1.00 MByte)
------------------------------------------------------------
------------------------------------------------------------
Client connecting to 128.224.158.134, TCP port 5432
TCP window size:  216 KByte (WARNING: requested 1.00 MByte)
------------------------------------------------------------
[  5] local 128.224.165.247 port 49139 connected with 128.224.158.134 port 5432
[ ID] Interval       Transfer     Bandwidth
[  5]  0.0- 1.0 sec  6.25 MBytes  52.4 Mbits/sec
[  5]  1.0- 2.0 sec  6.25 MBytes  52.4 Mbits/sec
[  5]  2.0- 3.0 sec  6.25 MBytes  52.4 Mbits/sec
[  5]  3.0- 4.0 sec  6.12 MBytes  51.4 Mbits/sec
[  5]  4.0- 5.0 sec  6.25 MBytes  52.4 Mbits/sec
[  5]  5.0- 6.0 sec  6.25 MBytes  52.4 Mbits/sec
[  5]  6.0- 7.0 sec  6.25 MBytes  52.4 Mbits/sec
[  5]  7.0- 8.0 sec  6.25 MBytes  52.4 Mbits/sec
[  5]  8.0- 9.0 sec  6.25 MBytes  52.4 Mbits/sec
Waiting for server threads to complete. Interrupt again to force quit.
[  5]  9.0-10.0 sec  6.25 MBytes  52.4 Mbits/sec
[  5]  0.0-10.0 sec  62.5 MBytes  52.4 Mbits/sec
\end{lstlisting}
\subsection{A strange bug \textcolor{red}{[bug]}}
\begin{itemize}
    \item Full-duplex;
    \item TCP;
    \item Max bandwidth;
\end{itemize}
Then the board comes to a halt again. But when board and a desktop do the same test, 
nothing bad happen, including the 
\textcolor{red}{Waiting for server threads to complete. Interrupt again to force quit.}
\subsection{Network traffic \textcolor{green}{[pass]}}
After some tests, check network traffic:
\begin{lstlisting}
root@localhost:/root> ifconfig eth0 | grep RX
          RX packets:422722 errors:0 dropped:0 overruns:0 frame:0
          RX bytes:513308213 (489.5 MiB)  TX bytes:150841217 (143.8 MiB)
root@localhost:/root> ifconfig eth0 | egrep "(RX|TX)"
          RX packets:422827 errors:0 dropped:0 overruns:0 frame:0
          TX packets:272087 errors:0 dropped:0 overruns:0 carrier:0
          RX bytes:513316625 (489.5 MiB)  TX bytes:150847151 (143.8 MiB)
\end{lstlisting}
Then do it again:
\begin{lstlisting}
root@localhost:/root> iperf -c 128.224.158.134 -i 1 -w 1M -p 5432
------------------------------------------------------------
Client connecting to 128.224.158.134, TCP port 5432
TCP window size:  216 KByte (WARNING: requested 1.00 MByte)
------------------------------------------------------------
[  3] local 128.224.165.247 port 44292 connected with 128.224.158.134 port 5432
[ ID] Interval       Transfer     Bandwidth
[  3]  0.0- 1.0 sec  6.25 MBytes  52.4 Mbits/sec
[  3]  1.0- 2.0 sec  6.25 MBytes  52.4 Mbits/sec
[  3]  2.0- 3.0 sec  4.88 MBytes  40.9 Mbits/sec
[  3]  3.0- 4.0 sec  6.12 MBytes  51.4 Mbits/sec
[  3]  4.0- 5.0 sec  6.38 MBytes  53.5 Mbits/sec
[  3]  5.0- 6.0 sec  6.25 MBytes  52.4 Mbits/sec
[  3]  6.0- 7.0 sec  6.12 MBytes  51.4 Mbits/sec
[  3]  7.0- 8.0 sec  6.25 MBytes  52.4 Mbits/sec
[  3]  8.0- 9.0 sec  6.25 MBytes  52.4 Mbits/sec
[  3]  9.0-10.0 sec  6.25 MBytes  52.4 Mbits/sec
[  3]  0.0-10.0 sec  61.2 MBytes  51.2 Mbits/sec
\end{lstlisting}
Check again:
\begin{lstlisting}
root@localhost:/root> ifconfig eth0 | egrep "(RX|TX)"
          RX packets:445698 errors:0 dropped:0 overruns:0 frame:0
          TX packets:316707 errors:0 dropped:0 overruns:0 carrier:0
          RX bytes:514831458 (490.9 MiB)  TX bytes:218028711 (207.9 MiB)
\end{lstlisting}
The board as client has sent about 61.2 MBytes while received 1.5 MBytes during the test.\\
Almost match.
\subsection{Using IPv6 address}
Relate to \ref{IPv6-iperf}


\chapter{libcli}
\section{Description}
It provides a consistant Cisco style command-line environment for remote clients, 
with a few common features between every implemtation. The library is not 
accessed by itself, rather the software which uses it listens on a defined port for a 
{\tt telnet} connection. This connection is handed off to {\tt libcli} for processing.
{\tt libcli} includes support for command history, command line editing and 
filtering of command output. This package contains the files necessary for 
developing applications with {\tt libcli}.
\section{Installation Hierarchy}
\begin{lstlisting}
usr/lib/libcli.so
usr/lib/libcli.so.1.9
usr/lib/libcli.so.1.9.5
usr/include/libcli.h
\end{lstlisting}
\section{Warning}
\begin{itemize}
    \item {\tt make} will generate {\tt clitest} which is an executable but won't be
          moved to installation directory by default;
    \item More than one terminal are needed when use {\tt clitest}, so use {\tt screen}
          which has been already in wrlinux repo but not included by default, or start
          {\tt sshd} server on the board to multi-login from our desktops.
\end{itemize}
\section{Test}
\subsection{Establish connection with wrong username or wrong password \textcolor{green}{[pass]}}
On the first terminal:
\begin{lstlisting}
root@localhost:/root> ./clitest
Listening on port 8000
\end{lstlisting}
On the second terminal, the default username is fred, password is nerk.\\
First time use wrong username, second time use wrong password.\\
After connected, {\tt router>} shows:
\begin{lstlisting}
root@localhost:/root> telnet 127.0.0.1 8000
Trying 127.0.0.1...
Connected to 127.0.0.1.
Escape character is '^]'.
libcli test environment

Username: frek
Password: 

Access denied

Username: fred
Password: 

Access denied

Username: fred
Password:

router> 
\end{lstlisting}
Then on the first terminal:
\begin{lstlisting}
 * accepted connection from 127.0.0.1
\end{lstlisting}
While:
\begin{lstlisting}
root@localhost:/root> netstat -atunp| egrep "(clitest|telnet|State)"
Proto Recv-Q Send-Q Local Address   Foreign Address State       PID/Program
tcp        0      0 0.0.0.0:8000    0.0.0.0:*       LISTEN      614/clitest
tcp        0      0 127.0.0.1:40851 127.0.0.1:8000  ESTABLISHED 648/telnet
tcp        0      0 127.0.0.1:8000  127.0.0.1:40851 ESTABLISHED 649/clitest
\end{lstlisting}
\subsection{Idle to timeout \textcolor{green}{[pass]}}
The default timeout is 60 seconds:
\begin{lstlisting}
router> Custom idle timeout
Connection closed by foreign host.
root@localhost:/root>
\end{lstlisting}
\subsection{Help info \textcolor{green}{[pass]}}
Type {\tt help} or {\tt ?}, kind of different:
\begin{lstlisting}
router> help

Commands available:
  help                 Show available commands
  quit                 Disconnect
  logout               Disconnect
  exit                 Exit from current mode
  history              Show a list of previously run commands
  enable               Turn on privileged commands
  test                 
  show regular         Show the how many times cli_regular has run
  show counters        Show the counters that the system uses
  show junk            
  debug regular        Enable cli_regular() callback debugging

router> 
  help                 Show available commands
  quit                 Disconnect
  logout               Disconnect
  exit                 Exit from current mode
  history              Show a list of previously run commands
  enable               Turn on privileged commands
  test                 
  show                 
  debug                

\end{lstlisting}
\subsection{Arguments processing \textcolor{green}{[pass]}}
\begin{lstlisting}
router> test hello world -n -k -p
called cmd_test with "test"
5 arguments:
        hello
        world
        -n
        -k
        -p

router> test -a -b
called cmd_test with "test"
2 arguments:
        -a
        -b

\end{lstlisting}
\subsection{Secret mode \textcolor{green}{[pass]}}
The password is {\tt topsecret}:
\begin{lstlisting}
router> enable
Password: 
router# help

Commands available:
  help                 Show available commands
  quit                 Disconnect
  logout               Disconnect
  exit                 Exit from current mode
  history              Show a list of previously run commands
  enable               Turn on privileged commands
  disable              Turn off privileged commands
  configure terminal   Configure from the terminal
  test                 
  set                  
  show regular         Show the how many times cli_regular has run
  show counters        Show the counters that the system uses
  show junk            
  debug regular        Enable cli_regular() callback debugging

\end{lstlisting}
\subsection{Configure mode \textcolor{green}{[pass]}}
Type {\tt configure} then {\tt ?} to show for now only terminal could be set:
\begin{lstlisting}
router# configure 
  terminal             Configure from the terminal

router# configure terminal

router(config)# 

router(config)# help

Commands available:
  help                 Show available commands
  quit                 Disconnect
  logout               Disconnect
  exit                 Exit from current mode
  history              Show a list of previously run commands
  interface            Configure an interface
\end{lstlisting}
Type {\tt interface} then {\tt ?} to check what interface is available:
\begin{lstlisting}
router(config)# interface 
  test0/0
\end{lstlisting}
Then (pay attention to the changes):
\begin{lstlisting}
router(config)# interface help
Unknown interface help

router(config)# interface 
  test0/0

router(config)# interface test0/0

router(config-test)# 
\end{lstlisting}
\subsection{Regular debug \textcolor{green}{[pass]}}
\begin{lstlisting}
router> debug regular
cli_regular() debugging is enabled

router> Regular callback - 21 times so far

router> Regular callback - 22 times so far

router> Regular callback - 23 times so far

router> show regular
cli_regular() has run 23 times

router> Regular callback - 24 times so far

router> Regular callback - 25 times so far

router> Regular callback - 26 times so far

router> Regular callback - 27 times so far

router> debug regular
cli_regular() debugging is disabled
\end{lstlisting}
\subsection{Show command history \textcolor{green}{[pass]}}
\begin{lstlisting}
router# history     

Command history:
  0. hlep
  1. help
  2. enable
  3. show regular
  4. show junk
\end{lstlisting}
\subsection{Cisco command line style \textcolor{green}{[pass]}}
Type some keys, then {\tt ?}, it will show the current 
possible commands for prompt or nothing:
\begin{lstlisting}
router> 
router> sh
  show                 

router> us 

router> h
  help                 Show available commands
  history              Show a list of previously run commands
\end{lstlisting}
%\chapter{libcli}

\chapter{IPv6}
\section{Warning}
\begin{itemize}
    \item Base on Mr. Wen's IPv6 Test Design;
    \item Use {\tt ifconfig eth0 inet6 add xxx} to add manually:
\begin{lstlisting}                                   
 +----------+--------------------+------------------+
 |          | global             |   site-local     |
 +----------+--------------------+------------------+
 |one       | 2001:db8:0:f101::1 | fec0:0:0:f101::1 |
 +----------+--------------------+------------------+ 
 |the other | 2001:db8:0:f101::2 | fec0:0:0:f101::2 |
 +----------+--------------------+------------------+
\end{lstlisting}                                     
    \item For these tests, didn't connect to Internet outside, therefore site-local
          and global addresses both work;
\end{itemize}
\section{Test}
\subsection{use {\tt appweb}}
\label{IPv6-appweb}
\subsubsection{site-local address}
\\[\intextsep]
\begin{minipage}{\textwidth}
\centering
\includegraphics[scale=.45]{appweb/site.png}
\end{minipage}
\\[\intextsep]
On the server to check:
\begin{lstlisting}
% netstat -A inet6 -atunp | grep 7777          
tcp 0 0 fec0:0:0:f101::1:40528 fec0:0:0:f101::2:7777 ESTABLISHED 2469/chrome
\end{lstlisting}
Server is fec0:0:0:f101::2, TCP, port 7777
\subsubsection{global address}
\\[\intextsep]
\begin{minipage}{\textwidth}
\centering
\includegraphics[scale=.45]{appweb/global.png}
\end{minipage}
\\[\intextsep]
Check on the server:
\begin{lstlisting}
% netstat -A inet6 -atunp | grep 7777         
tcp 0 0 2001:db8:0:f101::1:60682 2001:db8:0:f101::2:7777 ESTABLISHED 2469/chrome  
tcp 0 0 2001:db8:0:f101::1:60685 2001:db8:0:f101::2:7777 ESTABLISHED 2469/chrome  
tcp 0 0 2001:db8:0:f101::1:60686 2001:db8:0:f101::2:7777 ESTABLISHED 2469/chrome
\end{lstlisting}
Server is 2001:db8:0:f101::2, TCP, port 7777
\subsection{use {\tt iperf}}
\label{IPv6-iperf}
\subsubsection{site-local address}
\begin{lstlisting}
% iperf -s -V -i 1 -w 100K -u
------------------------------------------------------------
Server listening on UDP port 5001
Receiving 1470 byte datagrams
UDP buffer size:  200 KByte (WARNING: requested  100 KByte)
------------------------------------------------------------
[  3] local fec0:0:0:f101::2 port 5001 connected with fec0:0:0:f101::1 port 46957
[ ID] Interval       Transfer     Bandwidth        Jitter   Lost/Total Datagrams
[  3]  0.0- 1.0 sec   128 KBytes  1.05 Mbits/sec   0.022 ms    0/   89 (0%)
[  3]  1.0- 2.0 sec   128 KBytes  1.05 Mbits/sec   0.030 ms    0/   89 (0%)
[  3]  2.0- 3.0 sec   128 KBytes  1.05 Mbits/sec   0.020 ms    0/   89 (0%)
[  3]  3.0- 4.0 sec   128 KBytes  1.05 Mbits/sec   0.049 ms    0/   89 (0%)
[  3]  4.0- 5.0 sec   128 KBytes  1.05 Mbits/sec   0.025 ms    0/   89 (0%)
[  3]  5.0- 6.0 sec   129 KBytes  1.06 Mbits/sec   0.024 ms    0/   90 (0%)
[  3]  6.0- 7.0 sec   128 KBytes  1.05 Mbits/sec   0.067 ms    0/   89 (0%)
[  3]  7.0- 8.0 sec   128 KBytes  1.05 Mbits/sec   0.029 ms    0/   89 (0%)
[  3]  8.0- 9.0 sec   128 KBytes  1.05 Mbits/sec   0.010 ms    0/   89 (0%)
[  3]  9.0-10.0 sec   128 KBytes  1.05 Mbits/sec   0.020 ms    0/   89 (0%)
[  3]  0.0-10.0 sec  1.25 MBytes  1.05 Mbits/sec   0.021 ms    0/  893 (0%)
\end{lstlisting}
{\tt -V} means use IPv6.\\
Default port is 5001.
\begin{lstlisting}
% netstat -A inet6 -atunp | grep 5001
udp 0 0 fec0:0:0:f101::1:50080 fec0:0:0:f101::2:5001 ESTABLISHED 3164/iperf
% ip -6 neigh show
fec0:0:0:f101::1 dev eth0 lladdr 78:2b:cb:84:ea:05 REACHABLE
\end{lstlisting}
Server found its neighbor fec0:0:0:f101::1, namely the client.\\
While on the client:
\begin{lstlisting}
% iperf -c fec0:0:0:f101::2 -V  -i 1 -w 100K -u
------------------------------------------------------------
Client connecting to fec0:0:0:f101::2, UDP port 5001
Sending 1470 byte datagrams
UDP buffer size:  200 KByte (WARNING: requested  100 KByte)
------------------------------------------------------------
[  3] local fec0:0:0:f101::1 port 46957 connected with fec0:0:0:f101::2 port 5001
[ ID] Interval       Transfer     Bandwidth
[  3]  0.0- 1.0 sec   129 KBytes  1.06 Mbits/sec
[  3]  1.0- 2.0 sec   128 KBytes  1.05 Mbits/sec
[  3]  2.0- 3.0 sec   128 KBytes  1.05 Mbits/sec
[  3]  3.0- 4.0 sec   128 KBytes  1.05 Mbits/sec
[  3]  4.0- 5.0 sec   128 KBytes  1.05 Mbits/sec
[  3]  5.0- 6.0 sec   128 KBytes  1.05 Mbits/sec
[  3]  6.0- 7.0 sec   129 KBytes  1.06 Mbits/sec
[  3]  7.0- 8.0 sec   128 KBytes  1.05 Mbits/sec
[  3]  8.0- 9.0 sec   128 KBytes  1.05 Mbits/sec
[  3]  9.0-10.0 sec   128 KBytes  1.05 Mbits/sec
[  3]  0.0-10.0 sec  1.25 MBytes  1.05 Mbits/sec
[  3] Sent 893 datagrams
[  3] Server Report:
[  3]  0.0-10.0 sec  1.25 MBytes  1.05 Mbits/sec   0.020 ms    0/  893 (0%)
\end{lstlisting}
\subsubsection{global address}
Remain the server unchanged, just change the address on the client, namely connect
to server's global address:
\begin{lstlisting}
% iperf -c 2001:db8:0:f101::2 -V -i 1 -w 100K -u
------------------------------------------------------------
Client connecting to 2001:db8:0:f101::2, UDP port 5001
Sending 1470 byte datagrams
UDP buffer size:  200 KByte (WARNING: requested  100 KByte)
------------------------------------------------------------
[  3] local 2001:db8:0:f101::1 port 53081 connected with 2001:db8:0:f101::2 port 5001
[ ID] Interval       Transfer     Bandwidth
[  3]  0.0- 1.0 sec   129 KBytes  1.06 Mbits/sec
[  3]  1.0- 2.0 sec   128 KBytes  1.05 Mbits/sec
[  3]  2.0- 3.0 sec   128 KBytes  1.05 Mbits/sec
[  3]  3.0- 4.0 sec   128 KBytes  1.05 Mbits/sec
[  3]  4.0- 5.0 sec   128 KBytes  1.05 Mbits/sec
[  3]  5.0- 6.0 sec   128 KBytes  1.05 Mbits/sec
[  3]  6.0- 7.0 sec   129 KBytes  1.06 Mbits/sec
[  3]  7.0- 8.0 sec   128 KBytes  1.05 Mbits/sec
[  3]  8.0- 9.0 sec   128 KBytes  1.05 Mbits/sec
[  3]  9.0-10.0 sec   128 KBytes  1.05 Mbits/sec
[  3]  0.0-10.0 sec  1.25 MBytes  1.05 Mbits/sec
[  3] Sent 893 datagrams
[  3] Server Report:
[  3]  0.0-10.0 sec  1.25 MBytes  1.05 Mbits/sec   0.012 ms    0/  893 (0%)
\end{lstlisting}
While on the server:
\begin{lstlisting}
[  3] local 2001:db8:0:f101::2 port 5001 connected with 2001:db8:0:f101::1 port 53081
[ ID] Interval       Transfer     Bandwidth        Jitter   Lost/Total Datagrams
[  3]  0.0- 1.0 sec   128 KBytes  1.05 Mbits/sec   0.015 ms    0/   89 (0%)
[  3]  1.0- 2.0 sec   128 KBytes  1.05 Mbits/sec   0.009 ms    0/   89 (0%)
[  3]  2.0- 3.0 sec   128 KBytes  1.05 Mbits/sec   0.012 ms    0/   89 (0%)
[  3]  3.0- 4.0 sec   128 KBytes  1.05 Mbits/sec   0.012 ms    0/   89 (0%)
[  3]  4.0- 5.0 sec   128 KBytes  1.05 Mbits/sec   0.012 ms    0/   89 (0%)
[  3]  5.0- 6.0 sec   129 KBytes  1.06 Mbits/sec   0.014 ms    0/   90 (0%)
[  3]  6.0- 7.0 sec   128 KBytes  1.05 Mbits/sec   0.014 ms    0/   89 (0%)
[  3]  7.0- 8.0 sec   128 KBytes  1.05 Mbits/sec   0.012 ms    0/   89 (0%)
[  3]  8.0- 9.0 sec   128 KBytes  1.05 Mbits/sec   0.010 ms    0/   89 (0%)
[  3]  9.0-10.0 sec   128 KBytes  1.05 Mbits/sec   0.014 ms    0/   89 (0%)
[  3]  0.0-10.0 sec  1.25 MBytes  1.05 Mbits/sec   0.013 ms    0/  893 (0%)
\end{lstlisting}
Check neighbor:
\begin{lstlisting}
% ip -6 neigh show
fec0:0:0:f101::1 dev eth0 lladdr 78:2b:cb:84:ea:05 STALE
fe80::7a2b:cbff:fe84:ea05 dev eth0 lladdr 78:2b:cb:84:ea:05 REACHABLE
2001:db8:0:f101::1 dev eth0 lladdr 78:2b:cb:84:ea:05 REACHABLE
\end{lstlisting}
Note the previous connection with site-local address is {\tt STALE} now.\\
I'm lack of relative knowledge to figure out the appearance of link-local address
fe80 here.
\subsection{use {\tt ssh}}
Remote login via ssh with IPv6:
\subsubsection{site-local address}
\begin{lstlisting}
% ssh hask@fec0:0:0:f101::1
The authenticity of host 'fec0:0:0:f101::1 (fec0:0:0:f101::1)' can't be established.
RSA key fingerprint is 2d:d4:4e:70:8d:87:d1:68:2b:80:08:a1:17:ad:91:3f.
Are you sure you want to continue connecting (yes/no)? yes
Warning: Permanently added 'fec0:0:0:f101::1' (RSA) to the list of known hosts.
hask@fec0:0:0:f101::1's password: 
%
\end{lstlisting}
It works.
\subsubsection{global address}
\begin{lstlisting}
% ssh hask@2001:db8:0:f101::1
The authenticity of host '2001:db8:0:f101::1 (2001:db8:0:f101::1)' can't be established.
RSA key fingerprint is 2d:d4:4e:70:8d:87:d1:68:2b:80:08:a1:17:ad:91:3f.
Are you sure you want to continue connecting (yes/no)? yes
Warning: Permanently added '2001:db8:0:f101::1' (RSA) to the list of known hosts.
hask@2001:db8:0:f101::1's password: 
Last login: Sat Mar 12 13:33:11 2011 from fec0:0:0:f101::2
\end{lstlisting}
Again it tells us the previous login is from fec0:0:0:f101::2, it's right.
%\chapter{IPv6}


\chapter{mmiotool}
\section{Description}
A simple tool written in C (200+ SLOC) manipulates {\tt /dev/mem} via {\tt mmap}, 
to have a directy access of non-RAM memory.
\section{Installation Hierarchy}
\begin{lstlisting}
/usr/sbin/mmiotool
\end{lstlisting}
\section{Warning}
\begin{itemize}
    \item Machine registers have mode of read or write. If write to a the read-only, 
          our naive {\tt mmiotool} also shows {\tt Wrote to ...}, in reality it didn't;
\end{itemize}
\section{Test}
\subsection{help info \textcolor{green}{[pass]}}
\begin{lstlisting}
root@localhost:/root> mmiotool 
Usage: mmiotool [-r PHYSADDR] [-w PHYSADDR VALUE] [-R PHYSADDR RANGE] [-m SLEEP] -h
          -r, --read PHYSADDR        : Read 32 bits from physical memory address PHYSADDR 
          -w, --write PHYSADDR VALUE : Write VALUE at physical memory address PHYSADDR 
          -R, --mread PHYSADDR RANGE : Read RANGE bytes starting from physical address PHYSADDR 
          -m, --msleep SLEEP         : Sleep for SLEEP milliseconds 
          -h, --help                 : print this help message.
\end{lstlisting}
\subsection{read specific registers \textcolor{red}{[mismatch with manual]}}
According to Page 44, {\sf Comcerto M83xxx Register Reference Guide}, physical 
addresss of clock base on APB bus is at 0x100b0000:
\begin{lstlisting}
function            address         default value
ARM_CLK_CNTRL       0x100b0000      0x30020144
ARM_CLK_CNTRL_2     0x100b0004      0xb1441414
AHB_CLK_CNTRL       0x100b0008      0x908200c7
AHB_CLK_CNTRL_2     0x100b000c      0x90c73044
PHY_CLK_CNTRL       0x100b0010      0x300200f9
PHY_CLK_CNTRL_2     0x100b0014      0xf0f97444
IPsec_DDR_PCIe_USB_CLK_CNTRL
                    0x100b0018      0x3c244440
TDMCLK_CNTRL        0x100b001c      0xc192a737
\end{lstlisting}
Use it to check the physical range:
\begin{lstlisting}
root@localhost:/root> mmiotool -R 0x100b0000 0x20

          :         +0         +4         +8         +C
0x100b0000: 0x10020144 0x31441414 0x1081007c 0x10310022 
                   +10        +14        +18        +1C
            0x100200f9 0x70f902aa 0x2c243440 0xc192a737
\end{lstlisting}
Only {\tt TDMCLK\_CNTRL} (0x100b001c - 0xc192a737) is same as guide specified. 
It ought to be different as the machine run and some bits changed, however some bits
are reserved, default value should be kept, e.g. for the 1st, from 0x30020144 to
0x10020144, bit 28 changed from 1 to 0, while 28~22 in the guide:
\\[\intextsep]
\begin{minipage}{\textwidth}
\centering
\includegraphics[scale=.45]{mmiotool/clk_cntrl.png}
\end{minipage}
\\[\intextsep]
Write a module to check in kernel. The BSP telling us the static remap of clock base
on APB bus is {\tt #define APB\_VADDR(x) ((x) - COMCERTO\_AHB\_APB\_BASE + 
APB\_VADDR\_BASE)}, namely -0x10000000+0xfc000000:
\begin{lstlisting}
#include <linux/init.h>
#include <linux/module.h>
#include <mach/comcerto-1000.h>

static int mmiotool_init(void)
{
    int i;
    for (i = 0; i < 8; ++i) {
        printk(KERN_INFO "0x%08x\n", 
              *(unsigned *)(APB_VADDR(0x100b0000) + (i << 2)));
    }
    return 0;
}

static void mmiotool_exit(void){}

module_init(mmiotool_init);
module_exit(mmiotool_exit);

MODULE_LICENSE("Dual BSD/GPL");
MODULE_ALIAS("mmiotool test");
\end{lstlisting}
On the board insert, compare the outputs with above:
\begin{lstlisting}
root@localhost:/root> insmod mmiotool.ko
root@localhost:/root> dmesg | tail -8 
0x10020144
0x31441414
0x1081007c
0x10310022
0x100200f9
0x70f902aa
0x2c243440
0xc192a737
\end{lstlisting}
Same as {\tt mmiotool} did.\\
\subsection{write specific registers \textcolor{green}{[pass]}}
\textcolor{red}{Warnning: success in M83xxx, failed in c1k. The manual is for M83xxx.}
Page 54:
\\[\intextsep]
\begin{minipage}{\textwidth}
\centering
\includegraphics[scale=.45]{mmiotool/glbl_sw_reset.png}
\end{minipage}
\\[\intextsep]
Means the 1st ({\tt Reset}) depends on the 2nd ({\tt Reset Status}), which is default 
to inactive.\\
Check first:
\begin{lstlisting}
root@localhost:/root> mmiotool -R 0x100b0104 8
          :         +0         +4
0x100b0104: 0x00000001 0x00000000
\end{lstlisting}
Match well with the guide this time.\\\\
Try to make it RESET directly:
\begin{lstlisting}
root@localhost:/root> mmiotool -w 0x100b0104 0x0
Wrote to  0x100b0104: 0x00000000
root@localhost:/root> mmiotool -r 0x100b0104
Read from 0x100b0104: 0x00000001
\end{lstlisting}
Reserved by hardware ({\tt Reset Status}). So still 0x00000001.\\
We may make it more intelligent.\\\\
Well first change {\tt Reset Status}:
\begin{lstlisting}
root@localhost:/root> mmiotool -w 0x100b0108 0x1
Wrote to  0x100b0108: 0x00000001
root@localhost:/root> mmiotool -r 0x100b0108
Read from 0x100b0108: 0x00000001
\end{lstlisting}
Done, then {\tt Reset}:
\begin{lstlisting}
root@localhost:/root> mmiotool -w 0x100b0104 0x0

DDR Training.............................................................Done

U-Boot 1.1.6 (Sep 25 2010 - 11:11:08) Mindspeed $Name: u-boot_6_00_4 $

DRAM:  256 MB
Comcerto Flash Subsystem Initialization
Flash: 32 MB
In:    serial
Out:   serial
Err:   serial
Reserve MSP memory
DDR Training : 
DLL_ADJ(0,1,2,3): 0x21,0x20,0x20,0x21
WR_DQS:delay0 0x87, delay1 0x87, delay2 0x86, delay3 0x87
Net:   comcerto_gemac0, comcerto_gemac1
Hit any key to stop autoboot:  0 
PHY 100Mbit FD
Using comcerto_gemac0 device
File transfer via NFS from server 128.224.165.20; our IP address is 128.224.165.249
Filename '/export/pxeboot/vlm-boards/18283/kernel'.
Load address: 0x80600000
\end{lstlisting}
Reset global software leads the whole system restart. Great.
%\chapter{mmiotool}


\chapter{mtd}
\section{Description}
Simple memory technology device (MTD) manipulation tool. A substitution for
{\tt mtd-utils} on embedded systems.
\section{Warning}
\begin{itemize}
    \item The old version from {\tt OpenWrt} only consists of {\tt mtd.c} and 
          {\tt mtd.h}, while using the newest here;
    \item SW1-3 on the board is at 0/ON postion to support NOR Flash, so
          {\tt yaffs} and {\tt yaffs2} tests failed;
    \item Issuing {\tt mkyaffs2} with 512 bytes per page will format the yaffs1 image;
\end{itemize}
\section{Installation Hierarchy}
\begin{lstlisting}
/usr/sbin/mtd
\end{lstlisting}
\section{Test}
\subsection{help info \textcolor{green}{[pass]}}
Just press {\tt mtd} to see whether it shows:
\begin{lstlisting}
root@localhost:/root> mtd
Usage: mtd [<options> ...] <command> [<arguments> ...] <device>[:<device>...]

The device is in the format of mtdX (eg: mtd4) or its label.
mtd recognizes these commands:
    unlock                  unlock the device
    refresh                 refresh mtd partition
    erase                   erase all data on device
    write <imagefile>|-     write <imagefile> (use - for stdin) to device
    jffs2write <file>       append <file> to the jffs2 partition on the device
Following options are available:
    -q                      quiet mode (once: no [w] on writing,
                                       twice: no status messages)
    -n                      write without first erasing the blocks
    -r                      reboot after successful command
    -f                      force write without trx checks
    -e <device>             erase <device> before executing the command
    -d <name>               directory for jffs2write, defaults to "tmp"
    -j <name>               integrate <file> into jffs2 data when writing an image
    -F <part>[:<size>[:<entrypoint>]][,<part>...]
                            alter the fis partition table to create new partitions replacing
                            the partitions provided as argument to the write command
                            (only valid together with the write command)

Example: To write linux.trx to mtd4 labeled as linux and reboot afterwards
         mtd -r write linux.trx linux
\end{lstlisting}
\subsection{write imagefile to device \textcolor{green}{[pass]}}
On the laptop make a jffs2 image file, use directory {\tt /etc/ssh/}:
\begin{lstlisting}
% ll /etc/ssh
total 156K
262642 -rw-------  1 root root 123K Nov 24 02:49 moduli
263182 -rw-r--r--  1 root root 2.1K Nov 24 02:49 ssh_config
263738 -rw-------  1 root root 3.7K Nov 24 02:49 sshd_config
265564 -rw-------. 1 root root  668 Feb 14 19:38 ssh_host_dsa_key
265565 -rw-r--r--. 1 root root  590 Feb 14 19:38 ssh_host_dsa_key.pub
265562 -rw-------. 1 root root  965 Feb 14 19:38 ssh_host_key
265563 -rw-r--r--. 1 root root  630 Feb 14 19:38 ssh_host_key.pub
265560 -rw-------. 1 root root 1.7K Feb 14 19:38 ssh_host_rsa_key
265561 -rw-r--r--. 1 root root  382 Feb 14 19:38 ssh_host_rsa_key.pub
% sudo mkfs.jffs2 -d /etc/ssh -o jffs2.bin  
% ll | grep jffs2 
1205280 -rw-r--r--   1 hask hask  31K Feb 25 14:45 jffs2.bin
\end{lstlisting}
Then place it into rootfs.\\
And on the board:
\begin{lstlisting}
root@localhost:/root> mtd write jffs2.bin /dev/mtd2
Unlocking /dev/mtd2 ...

Writing from jffs2.bin to /dev/mtd2 ...     
\end{lstlisting}
Mount {/dev/mtdblock2} to somewhere, like {\tt /mnt}:
\begin{lstlisting}
root@localhost:/root> mount -t jffs2 /dev/mtdblock2 /mnt/
root@localhost:/root> ll /mnt/
total 136
-rw------- 1 root root 125811 2010-09-23 03:15 moduli
-rw-r--r-- 1 root root   2000 2010-09-23 03:15 ssh_config
-rw------- 1 root root   3732 2010-09-23 03:15 sshd_config
-rw------- 1 root root    672 2011-02-25 05:30 ssh_host_dsa_key
-rw-r--r-- 1 root root    590 2011-02-25 05:30 ssh_host_dsa_key.pub
-rw------- 1 root root    963 2011-02-25 05:29 ssh_host_key
-rw-r--r-- 1 root root    627 2011-02-25 05:29 ssh_host_key.pub
-rw------- 1 root root   1675 2011-02-25 05:29 ssh_host_rsa_key
-rw-r--r-- 1 root root    382 2011-02-25 05:29 ssh_host_rsa_key.pub
\end{lstlisting}
Checkpoints:
\begin{itemize}
    \item {\tt mtd write} is ok, see the outputs. It's not \textcolor{red}
          {Could not get MTD device info from /dev/mtdblock2. 
           Can't open device for writing!};
    \item After {\tt mount}, contents match with what we've set into {\tt jffs2.bin};
\end{itemize}
\subsection{append file to device \textcolor{red}{[fail]}}
\begin{lstlisting}
root@localhost:/root> mtd jffs2write a.txt /dev/mtd2
Unlocking /dev/mtd2 ...
Appending a.txt to jffs2 partition /dev/mtd2
Error: No room for additional data
\end{lstlisting}
{\tt /dev/mtd2} has configured 2M space, and jffs2.bin is 31K.
\subsection{erase \textcolor{green}{[pass]}}
\begin{lstlisting}
root@localhost:/root> mtd erase /dev/mtd2
Unlocking /dev/mtd2 ...
Erasing /dev/mtd2 ...
root@localhost:/root> mount /dev/mtdblock2 /mnt/ -t jffs2
root@localhost:/root> ll /mnt/
total 0
\end{lstlisting}
Checkpoints:
\begin{itemize}
    \item The {mtd erase} command status;
    \item After {\tt mount}, it ought to be empty;
\end{itemize}
\subsection{unlock \textcolor{green}{[pass]}}
Every other functions do {\tt unlock} first, so no problem.
\subsection{refresh device \textcolor{red}{[no support in kernel]}}
\begin{lstlisting}
root@localhost:/root> mtd refresh /dev/mtd2
Refreshing mtd partition /dev/mtd2 ... 
Failed to refresh the MTD device
\end{lstlisting}
See the function in {\tt mtd.c}:
\begin{lstlisting}
static int
mtd_refresh(const char *mtd)
{
    int fd;

    if (quiet < 2)
            fprintf(stderr, "Refreshing mtd partition %s ... ", mtd);

    fd = mtd_check_open(mtd);
    if(fd < 0) {
            fprintf(stderr, "Could not open mtd device: %s\n", mtd);
            exit(1);
    }

    /* LOOK HERE */
    if (ioctl(fd, MTDREFRESH, NULL)) {              
            fprintf(stderr, "Failed to refresh the MTD device\n");
            close(fd);
            exit(1);
    }
    ...

\end{lstlisting}
Give method {\tt MTDREFRESH} to {\tt ioctl()}.\\
But in the mtd's {\tt ioctl()} implementation in {\tt drivers/mtd/mtdchar.c}, 
no this method:
\begin{lstlisting}
MEMGETREGIONCOUNT
MEMGETREGIONINFO
MEMGETINFO
MEMERASE
MEMERASE64
MEMWRITEOOB
MEMREADOOB
MEMWRITEOOB64
MEMREADOOB64
MEMLOCK
MEMUNLOCK
MEMGETOOBSEL
MEMGETBADBLOCK
MEMSETBADBLOCK
OTPSELECT
OTPGETREGIONCOUNT
OTPGETREGIONINFO
OTPLOCK
ECCGETLAYOUT
ECCGETSTATS
MTDFILEMODE
\end{lstlisting}
Not supported by the driver in kernel.\\
Could shutdown this function via a patch.
\subsection{reboot after done \textcolor{green}{[pass]}}
{\tt -r} option is for reboot system after a successful command:
\begin{lstlisting}
root@localhost:/root> mtd -r erase /dev/mtd2
Unlocking /dev/mtd2 ...
Erasing /dev/mtd2 ...
Rebooting ...

Broadcast message from root@localhost (pts/0) (Thu Jan  1 06:52:59 1970):

The system is going down for reboot NOW!
Connection to 128.224.165.252 closed by remote host.
Connection to 128.224.165.252 closed.
\end{lstlisting}
\subsection{integrate a file into jffs2 data when writing \textcolor{red}{[fail]}}
\begin{lstlisting}
root@localhost:/root> mtd -j a.txt write jffs2.bin /dev/mtd2
Unlocking /dev/mtd2 ...

Writing from jffs2.bin to /dev/mtd2 ...     
root@localhost:/root> mount -t jffs2 /dev/mtdblock2 /mnt/
\end{lstlisting}
Seems ok, but after mounting, check the contents:
\begin{lstlisting}
root@localhost:/root> mount -t jffs2 /dev/mtdblock2 /mnt/
root@localhost:/root> ll /mnt/
total 136
-rw------- 1 root root 125811 2010-11-24 07:49 moduli
-rw-r--r-- 1 root root   2104 2010-11-24 07:49 ssh_config
-rw------- 1 root root   3767 2010-11-24 07:49 sshd_config
-rw------- 1 root root    668 2011-02-15 00:38 ssh_host_dsa_key
-rw-r--r-- 1 root root    590 2011-02-15 00:38 ssh_host_dsa_key.pub
-rw------- 1 root root    965 2011-02-15 00:38 ssh_host_key
-rw-r--r-- 1 root root    630 2011-02-15 00:38 ssh_host_key.pub
-rw------- 1 root root   1675 2011-02-15 00:38 ssh_host_rsa_key
-rw-r--r-- 1 root root    382 2011-02-15 00:38 ssh_host_rsa_key.pub
\end{lstlisting}
No integration.
\subsection{use device label instead of device name \textcolor{green}{[pass]}}
On the laptop make another image with directory {\tt /etc/dhcp}:
\begin{lstlisting}
% sudo mkfs.jffs2 -d /etc/dhcp -o jffs_dhcp.bin
\end{lstlisting}
On the board, check the mtd partitions first -- {\tt mtd2} has label {\tt kernel}:
\begin{lstlisting}
root@localhost:/root> cat /proc/mtd 
dev:    size   erasesize  name
mtd0: 00040000 00020000 "u-boot (NOR)"
mtd1: 00020000 00020000 "u-boot env(NOR)"
mtd2: 00200000 00020000 "kernel"
mtd3: 00010000 00020000 "nvram"
root@localhost:/root> mtd write jffs_dhcp.bin kernel
Unlocking kernel ...

Writing from jffs_dhcp.bin to kernel ...     
root@localhost:/root> mount -t jffs2 /dev/mtdblock2 /mnt/
root@localhost:/root> ll /mnt/
total 2
drwxr-xr-x 2 root root    0 2010-11-25 16:15 dhclient.d
-rw-r--r-- 1 root root  193 2011-01-28 09:18 dhcpd6.conf
-rw-r--r-- 1 root root 1308 2011-02-17 21:44 dhcpd.conf
\end{lstlisting}
Checkpoints:
\begin{itemize}
    \item {\tt write} is ok;
    \item the contents match;
\end{itemize}
\subsection{cramfs \textcolor{green}{[pass]}}
The kernel defaultly support these flash file systems, check one-by-one.\\\\
Make it:
\begin{lstlisting}
% sudo mkfs.cramfs /etc/ssh/ cramfs.bin
\end{lstlisting}
On the board:
\begin{lstlisting}
root@localhost:/root> mtd write cramfs.bin /dev/mtd2
Unlocking /dev/mtd2 ...

Writing from cramfs.bin to /dev/mtd2 ...     
root@localhost:/root> mount -t cramfs /dev/mtdblock2 /mnt/
root@localhost:/root> ll /mnt/
total 136
-rw------- 1 root root 125811 1970-01-01 00:00 moduli
-rw-r--r-- 1 root root   2104 1970-01-01 00:00 ssh_config
-rw------- 1 root root   3767 1970-01-01 00:00 sshd_config
-rw------- 1 root root    668 1970-01-01 00:00 ssh_host_dsa_key
-rw-r--r-- 1 root root    590 1970-01-01 00:00 ssh_host_dsa_key.pub
-rw------- 1 root root    965 1970-01-01 00:00 ssh_host_key
-rw-r--r-- 1 root root    630 1970-01-01 00:00 ssh_host_key.pub
-rw------- 1 root root   1675 1970-01-01 00:00 ssh_host_rsa_key
-rw-r--r-- 1 root root    382 1970-01-01 00:00 ssh_host_rsa_key.pub
\end{lstlisting}
Checkpoints:
\begin{itemize}
    \item {\tt write} is ok;
    \item the contents match;
\end{itemize}
\subsection{yaffs \textcolor{red}{[design for NAND]}}
For now, SW1-3 on the board is at 0/ON postion to support NOR Flash. Change them
to support NAND then try these tests.\\
On the laptop:
\begin{lstlisting}
% sudo mkyaffs2 -p 512 /etc/ssh yaffs.bin
mkyaffs2-0.1.9: image building tool for YAFFS2
Processing directory /etc/ssh into image file yaffs.bin
object 257, /etc/ssh/ssh_config is a file
object 258, /etc/ssh/ssh_host_dsa_key.pub is a file
object 259, /etc/ssh/ssh_host_rsa_key is a file
object 260, /etc/ssh/ssh_host_dsa_key is a file
object 261, /etc/ssh/moduli is a file
object 262, /etc/ssh/ssh_host_key is a file
object 263, /etc/ssh/sshd_config is a file
object 264, /etc/ssh/ssh_host_rsa_key.pub is a file
object 265, /etc/ssh/ssh_host_key.pub is a file
operation complete.
10 objects in 1 directories
282 NAND pages
\end{lstlisting}
On the board:
\begin{lstlisting}
root@localhost:/root> mtd write yaffs.bin /dev/mtd2
Unlocking /dev/mtd2 ...

Writing from yaffs.bin to /dev/mtd2 ...     
root@localhost:/root> mount /dev/mtdblock2 /mnt/ -t yaffs
mount: wrong fs type, bad option, bad superblock on /dev/mtdblock2,
       missing codepage or helper program, or other error
       In some cases useful info is found in syslog - try
       dmesg | tail  or so

root@localhost:/root> dmesg | tail -4
yaffs: dev is 32505858 name is "mtdblock2"
yaffs: passed flags ""
yaffs: Attempting MTD mount on 31.2, "mtdblock2"
yaffs: MTD device is not NAND it's type 3
\end{lstlisting}
Checkpoints:
\begin{itemize}
    \item {\tt write} is ok;
    \item the contents match;
\end{itemize}
\subsection{yaffs2 \textcolor{red}{[design for NAND]}}
Only difference is {\tt mkyaffs2 -p} to specify a page size other than 512 bytes.
Checkpoints:
\begin{itemize}
    \item {\tt write} is ok;
    \item the contents match;
\end{itemize}
%\subsection{ubifs \textcolor{red}{[fail]}}
%Checkpoints:
%\begin{itemize}
%    \item {\tt write} is ok;
%    \item the contents match;
%\end{itemize}
%\chapter{mtd}


\chapter{ntfs-3g}
\section{Description}
It is an NTFS driver, which can create, remove, rename, move files, 
directories, hard links, and streams; it can read and write files, including 
streams, sparse files and transparently compressed files; it can handle 
special files like symbolic links, devices, and FIFOs; moreover it provides 
standard management of file ownership and permissions, including POSIX ACLs.
\section{Installation Hierarchy}
\begin{lstlisting}
bin/ntfs-3g
bin/lowntfs-3g
usr/bin/ntfs-3g.probe
usr/bin/ntfs-3g.usermap
usr/bin/ntfs-3g.secaudit
sbin/mount.ntfs
sbin/mount.ntfs-3g
sbin/mount.lowntfs-3g
lib/libntfs-3g.so
lib/libntfs-3g.so.80.0.0
lib/libntfs-3g.so.80
usr/lib/libntfs-3g.so
usr/lib/pkgconfig/libntfs-3g.pc
usr/include/ntfs-3g/layout.h
usr/include/ntfs-3g/logfile.h
usr/include/ntfs-3g/reparse.h
usr/include/ntfs-3g/cache.h
usr/include/ntfs-3g/attrib.h
usr/include/ntfs-3g/debug.h
usr/include/ntfs-3g/unistr.h
usr/include/ntfs-3g/ntfstime.h
usr/include/ntfs-3g/efs.h
usr/include/ntfs-3g/bitmap.h
usr/include/ntfs-3g/acls.h
usr/include/ntfs-3g/device.h
usr/include/ntfs-3g/volume.h
usr/include/ntfs-3g/lcnalloc.h
usr/include/ntfs-3g/compress.h
usr/include/ntfs-3g/misc.h
usr/include/ntfs-3g/types.h
usr/include/ntfs-3g/collate.h
usr/include/ntfs-3g/support.h
usr/include/ntfs-3g/attrlist.h
usr/include/ntfs-3g/mst.h
usr/include/ntfs-3g/inode.h
usr/include/ntfs-3g/compat.h
usr/include/ntfs-3g/object_id.h
usr/include/ntfs-3g/device_io.h
usr/include/ntfs-3g/param.h
usr/include/ntfs-3g/runlist.h
usr/include/ntfs-3g/security.h
usr/include/ntfs-3g/index.h
usr/include/ntfs-3g/endians.h
usr/include/ntfs-3g/mft.h
usr/include/ntfs-3g/dir.h
usr/include/ntfs-3g/bootsect.h
usr/include/ntfs-3g/logging.h
usr/share/hal/fdi/policy/10osvendor/25-ntfs-config-write-policy.fdi
\end{lstlisting}
\section{Warning}
\begin{itemize}
    \item It depends on {\tt fuse} kernel module, and will load it automatically.
          Our default kernel configuration has no {\tt fuse}, modify then recompile;
    \item The volume to be mounted can be either a block device or an image file;
    \item It comes in two variants {\tt ntfs-3g} and {\tt lowntfs-3g} with a few 
          differences;
    \item The executable are {\tt ntfs-3g lowntfs-3g ntfs-3g.probe ntfs-3g.secaudit
          ntfs-3g.usermap}, and {\tt mount.ntfs mount.ntfs-3g} are symbolic links to 
          {\tt ntfs-3g}, {\tt mount.lowntfs-3g} is symbolic to {\tt lowntfs-3g};
    \item Option {\tt ignore\_case} is only with {\tt lowntfs-3g}.
\end{itemize}
\section{Test Environment}
\subsection{Plug a hard disk with NTFS partition into the target board}
\subsection{Mount a NTFS image on the board}
Make a NTFS image:
\begin{lstlisting}
% dd if=/dev/zero of=ntfs.img bs=512 count=10000   
10000+0 records in
10000+0 records out
5120000 bytes (5.1 MB) copied, 0.0567654 s, 90.2 MB/s
% ll ntfs.img                                     
9574 -rw-r--r-- 1 hask hask 4.9M Mar 25 22:50 ntfs.img
\end{lstlisting}
Almost 5M.\\
{\tt fdisk} and {\tt cfdisk} are partition table hacking utilities, namely modify 
446th-510th bytes of the first sector of disk or partition.\\
To make a NTFS filesystem on it, we need {\tt mkfs.ntfs} 
which in package {\tt ntfsprogs}.\\
{\tt -F} means force mkntfs to run, even if the specified device is 
not a block special device, or appears to be mounted:
\begin{lstlisting}
% mkfs.ntfs -F ntfs.img
ntfs.img is not a block device.
mkntfs forced anyway.
The sector size was not specified for ntfs.img and it could not be obtained automatically.  
    It has been set to 512 bytes.
The partition start sector was not specified for ntfs.img and it could not be obtained automatically.  
    It has been set to 0.
The number of sectors per track was not specified for ntfs.img and it could not be obtained automatically.  
    It has been set to 0.
The number of heads was not specified for ntfs.img and it could not be obtained automatically.  
    It has been set to 0.
Cluster size has been automatically set to 4096 bytes.
To boot from a device, Windows needs the 'partition start sector', the 'sectors per track' and the 'number of heads' to be set.
Windows will not be able to boot from this device.
Initializing device with zeroes: 100% - Done.
Creating NTFS volume structures.
mkntfs completed successfully. Have a nice day.
% file ntfs.img
ntfs.img: x86 boot sector, code offset 0x52, OEM-ID "NTFS    ", 
sectors/cluster 8, reserved sectors 0, Media descriptor 0xf8, dos < 4.0 BootSector (0x80)
\end{lstlisting}
\section{Test}
\subsection{{\tt mount -t ntfs-3g} or {\tt ntfs-3g} to mount a image}
\begin{lstlisting}
% ll | grep mnt
133731 drwxr-xr-x   2 hask hask 4.0K Feb 11 13:45 mnt
% sudo ntfs-3g ntfs.img mnt
% ll | grep mnt           
     5 drwxrwxrwx   1 root root 4.0K Feb 11 13:06 mnt
\end{lstlisting}
By default, files and directories are owned by the effective user and group of the 
mounting process, and everybody has full read, write, execution and directory 
browsing permissions.\\
\subsection{Mount with standard Linux permissions applied}
\begin{lstlisting}
% sudo mount -t ntfs-3g -o permissions ntfs.img mnt
Using default user mapping
% ll | grep mnt                                   
     5 drwxr-xr-x   1 root root 4.0K Feb 11 13:06 mnt
\end{lstlisting}
Make some files:
\begin{lstlisting}
% su -c 'echo "hask" > mnt/hask.txt'                 
Password: 
% sudo chown hask.hask mnt/hask.txt                
% su -c 'echo "root" > mnt/root.txt'             
Password: 
% ll mnt                  
total 1.0K
64 -rw-r--r-- 1 hask hask 5 Feb 11 14:14 hask.txt
65 -rw-r--r-- 1 root root 5 Feb 11 14:15 root.txt
\end{lstlisting}
Umount then mount with no Linux standard permissions:
\begin{lstlisting}
% sudo umount mnt
% sudo ntfs-3g ntfs.img mnt
% ll mnt                  
total 1.0K
64 -rwxrwxrwx 1 root root 5 Feb 11 14:14 hask.txt
65 -rwxrwxrwx 1 root root 5 Feb 11 14:15 root.txt
\end{lstlisting}
All belong to {\tt root} with mode 777.\\\\
\subsection{Change file permissions}
\subsection{Change file owners}
\subsection{Symbolic link and original file both in NTFS dir}
\begin{lstlisting}
% cd mnt
% sudo ln -s hask.txt hask.bak
% ll                         
total 1.5K
66 lrwxrwxrwx 1 root root 24 Feb 11 14:31 hask.bak -> hask.txt
64 -rw-r--r-- 1 hask hask  5 Feb 11 14:14 hask.txt
65 -rw-r--r-- 1 root root  5 Feb 11 14:15 root.txt
% cat hask.bak              
hask
\end{lstlisting}\null\\
\subsection{Symbolic link in NTFS dir to a file in normal dir}
\begin{lstlisting}
% sud ln -s ~/test.txt test.bak
% ll | egrep "\->"
66 lrwxrwxrwx 1 root root 24 Feb 11 14:31 hask.bak -> hask.txt
67 lrwxrwxrwx 1 root root 46 Feb 11 14:35 test.bak -> /home/hask/test.txt
% cat test.bak
sshfs symbolic
\end{lstlisting}\null\\
\subsection{Symbolic link in normal dir to a file in NTFS dir}
\begin{lstlisting}
% ln -s ~/mnt/hask.txt ~/hask.bak
% cat !$                        
cat ~/hask.bak
hask
\end{lstlisting}\null\\
\subsection{Hard link}
\begin{lstlisting}
% sudo ln root.txt root.ba
% cat !$                 
cat root.bak
root
\end{lstlisting}\null\\
\subsection{Read only mount}
\begin{lstlisting}
% sudo ntfs-3g -o ro,permissions ntfs.img mnt
Using default user mapping
% ll mnt                            
total 2.5K
66 lrwxrwxrwx 1 root root 24 Feb 11 14:31 hask.bak -> hask.txt
64 -rw-r--r-- 1 hask hask  5 Feb 11 14:14 hask.txt
65 -rw-r--r-- 2 root root  5 Feb 11 14:15 root.bak
65 -rw-r--r-- 2 root root  5 Feb 11 14:15 root.txt
67 lrwxrwxrwx 1 root root 46 Feb 11 14:35 test.bak -> /home/hask/test.txt
% echo "again" >> mnt/hask.txt       
zsh: read-only file system: mnt/hask.txt
\end{lstlisting}\null\\
\subsection{Mount with specific uid}
\begin{lstlisting}
% cat /etc/passwd | grep 1003         
windriver:x:1003:1004::/home/windriver:/bin/bash
% sudo ntfs-3g -o uid=1003 ntfs.img mnt
% ll mnt
total 2.5K
66 lrwxrwxrwx 1 windriver root 24 Feb 11 14:31 hask.bak -> hask.txt
64 -rwxrwxrwx 1 windriver root  5 Feb 11 14:14 hask.txt
68 -rwxrwxrwx 1 windriver root  0 Feb 11 14:40 HelloWorld
65 -rwxrwxrwx 2 windriver root  5 Feb 11 14:15 root.bak
65 -rwxrwxrwx 2 windriver root  5 Feb 11 14:15 root.txt
67 lrwxrwxrwx 1 windriver root 46 Feb 11 14:35 test.bak -> /home/hask/test.txt
\end{lstlisting}\null\\
\subsection{Mount with specific umask}
\begin{lstlisting}
% sudo ntfs-3g -o permissions,umask=444 ntfs.img mnt
% ll mnt                                     
ls: cannot open directory mnt: Permission denied
(1)% sudo umount mnt                          
% sudo ntfs-3g -o permissions,umask=111 ntfs.img mnt
% ll mnt                                           
ls: cannot access mnt/hask.bak: Permission denied
ls: cannot access mnt/hask.txt: Permission denied
ls: cannot access mnt/root.bak: Permission denied
ls: cannot access mnt/root.txt: Permission denied
ls: cannot access mnt/test.bak: Permission denied
total 0
? ?????????? ? ? ? ?            ? hask.bak
? -????????? ? ? ? ?            ? hask.txt
? -????????? ? ? ? ?            ? root.bak
? -????????? ? ? ? ?            ? root.txt
? ?????????? ? ? ? ?            ? test.bak
(1)% sudo umount mnt                               
% sudo ntfs-3g -o permissions,umask=000 ntfs.img mnt
% ll mnt             
total 2.5K
66 lrwxrwxrwx 1 root root 24 Feb 11 14:31 hask.bak -> hask.txt
64 -rwxrwxrwx 1 root root  5 Feb 11 14:14 hask.txt
65 -rwxrwxrwx 2 root root  5 Feb 11 14:15 root.bak
65 -rwxrwxrwx 2 root root  5 Feb 11 14:15 root.txt
67 lrwxrwxrwx 1 root root 46 Feb 11 14:35 test.bak -> /home/hask/test.txt
\end{lstlisting}\null\\
\subsection{Ignore file name case}
Use {\tt ignore\_case} when mounting, only with {\tt lowntfs-3g}:
\begin{lstlisting}
% sudo lowntfs-3g -o permissions,ignore_case ntfs.img mnt
% touch mnt/HelloWorld
% ll mnt
total 2.5K
66 lrwxrwxrwx 1 root root 24 Feb 11 14:31 hask.bak -> hask.txt
64 -rw-r--r-- 1 hask hask  5 Feb 11 14:14 hask.txt
68 -rw-r--r-- 1 hask hask  0 Feb 11 15:32 helloworld
65 -rw-r--r-- 2 root root  5 Feb 11 14:15 root.bak
65 -rw-r--r-- 2 root root  5 Feb 11 14:15 root.txt
67 lrwxrwxrwx 1 root root 46 Feb 11 14:35 test.bak -> /home/hask/test.txt
\end{lstlisting}\null\\
\subsection{Windows name only}
Under Linux we could use nine characters {\tt " * / : < > ? \ |} and those whose code is
less than 0x20 into a file name:
\begin{lstlisting}
% sudo touch "mnt/a>b.txt"
% ll mnt | egrep "a>b"
69 -rw-r--r-- 1 root root  0 Feb 11 14:54 a>b.txt
\end{lstlisting}
Use option {\tt windows\_names} when mount:
\begin{lstlisting}
% sudo touch "mnt/a<b.txt"
touch: setting times of `mnt/a<b.txt': No such file or directory
\end{lstlisting}\null\\
\subsection{Hide Windows hidden files}
By default listing in Linux will show all files including the hidden.\\
Create some hidden files in Windows then mount it with {\tt hide\_hid\_files}

\subsection{A new rule in {\tt /etc/fstab} to auto mount every boot}
%\chapter{ntfs-3g}


\chapter{ntpclient}
\section{Description}
It is an NTP (RFC-1305) client for unix-alike computers. 
Its functionality is a small subset of xntpd, but has the potential to function 
better within that limited scope. Since it is much smaller than xntpd, 
it is also more relevant for embedded computers.
\section{Installation Hierarchy}
\begin{lstlisting}
usr/bin/adjtimex
usr/bin/ntpclient
usr/share/doc/ntpclient/rate.awk
usr/share/doc/ntpclient/rate2.awk
\end{lstlisting}
\section{Warning}
\begin{itemize}
    \item The known bugs:
    \begin{itemize}
        \item Doesn't understand the LI (Leap second Indicator) field of an NTP packet;
        \item Doesn't interact with {\tt adjtimex(2)} status value;
        \item Can't query multiple servers;
        \item IPv4 only;
        \item Requires Linux-style {\tt select()} semantics, 
              where timeout value is modified;
        \item Always returns success (0);
    \end{itemize}
    \item \href{http://www.pool.ntp.org/zone/cn}{Here} to get more NTP servers. Which I used here is 
          {\tt 1.asia.poll.ntp.org};
    \item {\tt rate.awk} and {\tt rate2.awk} will be in {\tt /usr/share/doc/ntpclient}, but 
          {\tt test.dat} is in source directory;
    \item Daily options:
    \begin{itemize}
        \item {\tt -h} mandatory, to specify NTP server, 
                       against which to measure system time
        \item {\tt -s} simple clock set (implies -c 1) (requires root access)
        \item {\tt -c} top after count time measurements (default 0 means go forever)
        \item {\tt -i} check time every interval seconds (default 600)
        \item {\tt -p} local NTP client UDP port (default 0 means "any available")
        \item {\tt -l} attempt to lock local clock to server using {\tt adjtimex(2)}
    \end{itemize}
    \item Although system time could be calibrated, always 8 hours equation of time.
\end{itemize}
\section{Test \textcolor{red}{[strange]}}
\begin{lstlisting}
root@localhost:/root> export NTPHOST=1.asia.poll.ntp.org
\end{lstlisting}
By default the board has no {\tt /etc/resolv.conf} to determine where is the DNS server.
Maybe you should use 123.146.124.31 instead of the domain name, or make that file:
\begin{lstlisting}
root@localhost:/root> cat /etc/resolv.conf 
search wrs.com
nameserver 128.224.160.11
nameserver 147.11.100.30
\end{lstlisting}\null\\
Check the date:
\begin{lstlisting}
root@localhost:/root> date
Thu Jan  1 05:41:48 GMT 1970
\end{lstlisting}\null\\
First, Set the time approximately right:
\begin{lstlisting}
root@localhost:/root> ntpclient -s -h $NTPHOST
25567 00743.472  128526.0  10.1  1297329120808591.2 145782.5  0
\end{lstlisting}
The column hearders are:
\begin{enumerate}
    \item day since 1900
    \item seconds since midnight
    \item elapsed time for NTP transaction (microseconds)
    \item internal server delay (microseconds)
    \item clock difference between your computer and the NTP server (microseconds)
    \item dispersion reported by server (microseconds)
    \item your computer's adjtimex frequency (ppm * 65536)
\end{itemize}
See it's 1297329120 seconds fast, compared to the clock on server. And 
25567 days since 1900, it's right, because it's 1/1/1970.\\\\
Meanwhile {\tt netstat} to check, with UDP and server port 123:
\begin{lstlisting}
root@localhost:/root> netstat -atunp | egrep "(ntpclient|State)"
Proto Recv-Q Send-Q Local Address         Foreign Address    State       PID/Program
udp        0      0 128.224.165.247:55121 123.146.124.31:123 ESTABLISHED 602/ntpclient
\end{lstlisting}\null\\
Check that the clock setting worked:
\begin{lstlisting}
root@localhost:/root> ntpclient -c 1 -h $NTPHOST
\end{lstlisting}\null\\
On to measure the frequency calibration for the system:
\begin{lstlisting}
root@localhost:/root> ntpclient -i 60 -c 20 -h $NTPHOST >$(hostname).ntp.log &
root@localhost:/root> cat localhost.ntp.log 
40571 22894.172  129338.0     15.2  -213404163.0  42068.5         0
40571 22954.239  124529.0     10.5  -240046273.5  42572.0         0
40571 23014.313  130449.0     11.0  -266695109.6  43075.6         0
40571 23074.402  151852.0     11.6  -293349488.9  43579.1         0
40571 23134.452  133911.0     13.7  -319984158.0  44082.6         0
40571 23254.533   78743.0     13.9  -373243841.0  45089.7         0
40571 23314.654  131552.0     14.9  -399914015.3  45578.0         0
40571 23374.707  117069.0     12.1  -426550348.4  46081.5         0
40571 23434.805  147269.0     14.6  -453209075.8  46585.1         0
40571 23494.877  151540.0     14.1  -479852964.6  47088.6         0
40571 23554.940  146328.0     12.8  -506495945.2  47592.2         0
40571 23614.978  116990.0     14.5  -533125038.6  48095.7         0
40571 23675.063  134239.0     13.0  -559777229.9  48599.2         0
40571 23735.148  150516.0     13.5  -586428978.9  49087.5         0
40571 23795.168  103025.0     11.7  -613048707.7  49606.3         0
40571 23855.215   82005.0     11.8  -639682128.3  50094.6         0
40571 23915.341  140266.0     14.1  -666354861.6  50598.1         0
40571 23975.388  119592.0     13.8  -692988135.7  51101.7         0
40571 24035.478  141508.0     14.0  -719642709.9  51605.2         0
\end{lstlisting}
If the last column (kernel frequency fine tune) ever changes, you haven't
turned off other time adjustment programs.  AFAIK the only programs around
that would move this number are ntpclient and xntpd.  On most out-of-the-box
systems, that last column should start zero and stay zero.\\\\
Use {\tt rate.awk} to determine the appropriate frequency value:
\begin{lstlisting}
root@localhost:/root> awk -f /usr/share/doc/ntpclient/rate.awk <test.dat
delta-t 119400 seconds
delta-o -142308 useconds
slope -1.19186 ppm
old frequency -1240000 ( -18.9209 ppm)
new frequency -1318109 ( -20.1127 ppm)
\end{lstlisting}\null\\
Now to plug in the new frequency value generated by this step:
\begin{lstlisting}
root@localhost:/root> adjtimex -f -1318109
    mode:         2
-o  offset:       0
-f  frequency:    -1318109
    maxerror:     16000000
    esterror:     16000000
    status:       64 ( UNSYNC )
-p  timeconstant: 2
    precision:    1
    tolerance:    32768000
-t  tick:         10000
    time.tv_sec:  1297330926
    time.tv_usec: 317969
    return value: 5 (clock not synchronized)
\end{lstlisting}\null\\
Reset the clock:
\begin{lstlisting}
root@localhost:/root> ntpclient -s -h $NTPHOST
\end{lstlisting}
If the frequency offset (absolute value) is greater than about 230 ppm
(15073280), you have a problem: you may be able to fix it with the {\tt -t}
option to {\tt adjtimex}, or you need to hack {\tt phaselock.c}, that has a
maximum adjustment extent of +/- 250 ppm built into {\tt phaselock.c} (change
the {\tt #define MAX\_CORRECT} and rebuild ntpclient). The author suggests to
replace the defective crystal instead, but that is rarely practical.\\\\
It suggests that use {\tt ntpclient -l -h \$NTPHOST} in the background now.
It will make small (probably less than 3 ppm) adjustments to the system 
frequency to keep the clocks locked. Typical performance over Ethernet 
(even through a few routers) is a worst case error of +/- 10 ms.
\begin{lstlisting}
root@localhost:/root> ntpclient -l -h $NTPHOST
40582 35978.908  119055.0     12.1  -937914652.7 124618.5  -1318109
40582 36579.053  154713.0     13.0  -1204099142.6 129623.4  -1318109
40582 37179.227  217428.0     12.2  -1470279516.2 134613.0  -1318109
40582 38379.345  131114.0     10.1  -2002614233.0 127410.9  -1318109
40582 38979.531  206276.0     14.3  -2268818855.6 132431.0  -1318109
40582 40179.726  179323.0     12.1  -2801143748.3 142440.8  -1318109
40582 41379.923  154063.0     12.5  -3333521863.1  95260.6  -1318109
40582 42580.167  176057.0     12.7  -3865880695.2 105285.6  -1318109
40582 43180.281  196748.0     10.9  -4132043207.6 113128.7  -1318109
40582 43780.316  121224.0     10.5  -4398202410.1 118133.5  -1318109
40582 44380.439  133105.0     14.6  -4664398666.0 123138.4  -1318109
40582 45580.759  231466.0     16.2  -5196755086.4 135757.4  -1318109
40582 46180.788  148858.0     13.7  -5462914860.4 140762.3  -1318109
40582 47381.003  142418.0     38.3  -5995293813.3 150772.1  -1318109
40582 47981.133  178752.0     15.0  -6261469990.9 120666.5  -1318109
# inconsistent
40582 48581.219  154334.0      9.8  -6527646468.1 125671.4  -1318109
# box [( -443570.954 , -6794013920.0 )  ( -443520.817 , -6793632010.9 )]  delta_f 20.000  computed_freq -2628829
40582 49181.330  153726.0     11.0  -6793819041.8 130691.5  -1318109
# box [( -443550.954 , -7060144492.2 )  ( -443487.426 , -7059699340.0 )]  delta_f 20.000  computed_freq -3939549
40582 49781.436  160764.0     13.8  -7059995000.4 135696.4  -2628829
# box [( -443530.954 , -7326263064.4 )  ( -443465.795 , -7325792507.0 )]  delta_f 20.000  computed_freq -5250269
40582 50381.585  205087.0     13.9  -7326144148.1 147354.1  -3939549
# box [( -443510.954 , -7592369636.6 )  ( -443440.566 , -7591937492.8 )]  delta_f 20.000  computed_freq -6560989
40582 50981.697  206053.0      8.8  -7592295120.1 152374.3  -5250269
# box [( -443490.954 , -7858464208.8 )  ( -443437.473 , -7858142016.0 )]  delta_f 20.000  computed_freq -7871709
40582 51581.725  135746.0     14.0  -7858430740.1 153778.1  -6560989
# inconsistent
40582 52781.935  135855.0     11.3  -8390732078.7 163803.1  -7871709
# box [( -443512.834 , -8657245854.0 )  ( -443436.122 , -8656938330.0 )]  delta_f 20.000  computed_freq -9182429
40582 53382.104  200334.0     12.7  -8656877512.0 168808.0  -7871709
# box [( -443480.344 , -8923223821.3 )  ( -443414.633 , -8922986214.6 )]  delta_f 20.000  computed_freq -10493149
40582 53982.143  128406.0     13.8  -8922999184.7  97030.6  -9182429
# box [( -443460.344 , -9189300027.5 )  ( -443394.633 , -9189022994.2 )]  delta_f 20.000  computed_freq -11803869
40582 54582.322  191209.0     10.2  -9189139467.7 102035.5 -10493149
# box [( -443440.344 , -9455364233.7 )  ( -443374.633 , -9455047773.9 )]  delta_f 20.000  computed_freq -13114589
40582 55182.386  149848.0     12.7  -9455273762.8 107055.7 -11803869
# box [( -443420.344 , -9987468646.1 )  ( -443371.623 , -9987226264.9 )]  delta_f 20.000  computed_freq -14425309
40582 56382.635  178609.0     12.6  -9987503408.6  99334.7 -13114589
# inconsistent
40582 56982.692  137178.0     14.7  -10253578190.5 104354.9 -14425309
# inconsistent
40582 57582.803  131400.0      9.6  -10519719535.1 127105.7 -14425309
# inconsistent
40582 58182.897  127455.0     10.3  -10785813006.9 132125.9 -14425309
# inconsistent
40582 58783.007  138767.0     10.9  -11051927042.4 137130.7 -14425309
# box [( -443448.320 , -11318271132.4 )  ( -443392.001 , -11318068019.5 )]  delta_f 20.000  computed_freq -15736029
40582 59383.140  172892.0     12.5  -11318030335.6 142135.6 -14425309
# box [( -443428.320 , -11584328124.7 )  ( -443372.001 , -11584091220.1 )]  delta_f 20.000  computed_freq -17046749
40582 59983.201  129807.0      9.5  -11584136274.9 147155.8 -15736029
# box [( -443418.433 , -11850379184.6 )  ( -443362.114 , -11850108488.5 )]  delta_f 20.000  computed_freq -17694720
40582 60583.288  118141.0     13.0  -11850212254.8 115600.6 -16384000
# box [( -443418.433 , -12116430244.6 )  ( -443362.114 , -12116125756.8 )]  delta_f 20.000  computed_freq -17694720
40582 61183.414  145932.0     13.1  -12116335923.0 120605.5 -16384000
# box [( -443418.433 , -12382481304.5 )  ( -443367.151 , -12382182318.4 )]  delta_f 20.000  computed_freq -17694720
40582 61783.523  156721.0     12.1  -12382438672.7 100433.3 -16384000
# inconsistent
40582 62383.609  143951.0      9.9  -12648524105.2 105438.2 -16384000
# box [( -443481.184 , -12914800037.7 )  ( -443380.674 , -12914344619.5 )]  delta_f 20.000  computed_freq -17694720
40582 62983.766  184442.0     12.9  -12914638704.7 110443.1 -16384000
# inconsistent
40582 63583.811  131252.0     12.7  -13180721224.9 115249.6 -16384000
cat# box [( -443435.150 , -13713154916.4 )  ( -443385.316 , -13712981288.2 )]  delta_f 20.000  computed_freq -17694720
40582 64784.018  140799.0     11.2  -13712889642.7 125274.7 -16384000
# box [( -443435.150 , -13979216006.7 )  ( -443385.316 , -13979012478.0 )]  delta_f 20.000  computed_freq -17694720
40582 65384.096  120991.0     11.6  -13978993982.1 130279.5 -16384000
# box [( -443435.150 , -14245277097.0 )  ( -443385.316 , -14245043667.8 )]  delta_f 20.000  computed_freq -17694720
40582 65984.225  151020.0     12.9  -14245114022.6  70526.1 -16384000
\end{lstlisting}
It run for several hours, and I found the results are more inaccurate.
\begin{lstlisting}
root@localhost:/root> date
Thu Feb 10 18:27:52 GMT 2011
\end{lstlisting}
Reset again, back to 8 hours equation of time:
\begin{lstlisting}
root@localhost:/root> ntpclient -s -h $NTPHOST
40582 66531.713  145547.0     11.3  -14487866148.2  75088.5 -16384000
root@localhost:/root> date
Thu Feb 10 14:27:43 GMT 2011
\end{lstlisting}
%\chapter{ntpclient}


\chapter{nvram}
\section{Description}
Non-Volatile-RAM. If a device has this feature, it may be called nvram. Not a 
special device, I think\\
{\tt OpenWrt} set the last 64K of flash as `nvram', also labeled as `nvram', to
record a lot of config settings.\\
Typically it should be like this on a board:
\begin{lstlisting}
$ cat /proc/mtd shows:
dev:    size   erasesize  name
mtd0: 00040000 00020000 "uboot"
mtd1: 01fa0000 00020000 "linux"
mtd2: 00687800 00020000 "rootfs"
mtd3: 01840000 00020000 "jffs2"
mtd4: 00020000 00020000 "nvram"
\end{lstlisting}
However the comcerto-c1k BSP doesn't initialize the flash in this way, for NOR 
flash:
\begin{lstlisting}
struct mtd_partition c1kmfcn_evm_nor_partition[] = {
    {
        .name           = "u-boot (NOR)",
        .offset         = 0,
        .size           = 2 * SZ_128K,
        .mask_flags     = MTD_WRITEABLE,
    },
    {
        .name           = "u-boot env(NOR)",
        .offset         = MTDPART_OFS_APPEND,
        .size           = SZ_128K,
        .mask_flags     = MTD_WRITEABLE,
    },
    {
        .name           = "kernel",
        .offset         = SZ_2M,
        .size           = MTDPART_SIZ_FULL,
        .mask_flags     = 0,
    },
};
\end{lstlisting}
For NAND flash:
\begin{lstlisting}
struct mtd_partition c1kmfcn_evm_nandflash_partition[] = {
    {
        .name           = "u-boot",
        .offset         = 0,
        .size           = 2 * SZ_128K,
        .mask_flags     = MTD_WRITEABLE,
    },
    {
        .name           = "u-boot env",
        .offset         = MTDPART_OFS_APPEND,
        .size           = SZ_128K,
        .mask_flags     = MTD_WRITEABLE,
    },
    {
        .name           = "kernel",
        .offset         = MTDPART_OFS_APPEND,
        .size           = SZ_4M,
        .mask_flags     = 0,
    },
    {
        .name           = "filesystem",
        .offset         = MTDPART_OFS_APPEND,
        .size           = MTDPART_SIZ_FULL,
        .mask_flags     = 0,
    },
};
\end{lstlisting}
Edit here could just get a partition labeled `nvram'. Though {\tt nvram} opens
{\tt /proc/mtd} to find the partition named `nvram', it also check the magic number
0x48534C46 (FLSH) and some else. I've searched a lot what a shame found nothing about
how to make it.\\
Another possible way is making clear of all checks of {\tt nvram}, to load a module
generating them all at certain physical addresses.\\\\
Typical usage:\\
\begin{itemize}
    \item \url{http://hi.baidu.com/westhack/blog/item/83fd4aadfd8e4b0e4b36d616.html}
    \item \url{http://blogold.chinaunix.net/u1/38213/showart\_1890362.html}
\end{itemize}
\section{Installation Hierarchy}
\begin{lstlisting}
/usr/sbin/nvram
\end{lstlisting}
%\chapter{nvram}


%\chapter{qoscom}
%\section{Description}
%\section{Installation Hierarchy}
%\section{Warning}
%\section{Test}
%\chapter{qoscom}

%\chapter{qosapp}
%\section{Description}
%\section{Installation Hierarchy}
%\begin{lstlisting}
%/etc/qos.conf
%\end{lstlisting}
%\section{Warning}
%\section{Test}
%\chapter{qoscom}


\chapter{sshfs}
\section{Description}
{\tt sshfs} (Secure SHell FileSystem) is a file system for Linux 
(and other operating systems with a FUSE (Filesystem in Userspace) 
implementation, such as Mac OS X or FreeBSD) capable of operating on 
files on a remote computer using just a secure shell login on the remote 
computer. On the local computer where the {\tt sshfs} is mounted, the 
implementation makes use of the {\tt fuse}  kernel module. The practical 
effect of this is that the end user can seamlessly interact with remote 
files being securely served over {\tt ssh} just as if they were local files 
on his/her computer. On the remote computer the {\tt sftp} subsystem of 
{\tt ssh} is used.
\section{Installation Hierarchy}
\begin{lstlisting}
usr/bin/sshfs
\end{lstlisting}
\section{Warning}
\begin{itemize}
    \item It depends on {\tt fuse} kernel module, and will load it automatically.
          Our default kernel configuration has no {\tt fuse}, modify then recompile;
    \item Our repo has {fuse} utility which provides {\tt fusermount} in userspace,
          but also not including by default;
\end{itemize}
\section{Test}
\subsection{Basic mount and umount \textcolor{red}{[Chinese encoding]}}
Before mounting, no {\tt fuse} module loaded:
\begin{lstlisting}
root@localhost:/root> lsmod | grep fuse
\end{lstlisting}
The laptop haven't started {\tt sshd} server:
\begin{lstlisting}
root@localhost:/root> sshfs hask@128.224.158.134:/home/hask /mnt
read: Connection reset by peer
\end{lstlisting}
Mount the home directory of my laptop to {\tt /mnt/} of the board system, 
user name is {\tt hask}:
\begin{lstlisting}
root@localhost:/root> sshfs hask@128.224.158.134:/home/hask /mnt/
hask@128.224.158.134's password: 
\end{lstlisting}
Listing on the board works well excluding Chinese encoding:
\begin{lstlisting}
root@localhost:/root> ll /mnt/
total 105392
drwxr-xr-x 1 1002 1002    12288 Jan 14  2011 Desktop
-rw-r--r-- 1 1002 1002  5977249 Jan 14  2011 Intel-Nehalem-EP?????????????????.pdf
drwxr-xr-x 1 1002 1002     4096 Jan 26  2011 VirtualBox VMs
-rw-r--r-- 1 1002 1002   688128 Jan 25  2011 WR Linux PICD 1_0 schedule.mpp
-rw-r--r-- 1 1002 1002        7 Feb  1  2011 counts_per_sec
drwxr-xr-x 1 1002 1002     4096 Jan 26  2011 dl
-rw-r--r-- 1 1002 1002  1162711 Dec 15  2010 driver.pdf
-rw-r--r-- 1 1002 1002  4088925 Jan  8  2011 ipsec_vpn.pdf
-rw-r--r-- 1 1002 1002   263116 Jan  6  2011 kernel_locking_CN.pdf
-rw-r--r-- 1 1002 1002 73577826 Jan 21  2011 linux-2.6.37.tar.bz2
drwxr-xr-x 1 1002 1002     4096 Jan 27  2011 log
...

\end{lstlisting}
The user and group id is 1002, here's the reason, executed on the laptop:
\begin{lstlisting}
% cat /etc/passwd | grep hask        
hask:x:1002:1002::/home/hask:/bin/zsh
\end{lstlisting}
While on the board id 1002 is undefined.\\\\
{\tt netstat} to check, server port 22, TCP:
\begin{lstlisting}
root@localhost:/root> netstat -atnp | egrep "(128\.224|State)"
Proto Recv-Q Send-Q Local Address         Foreign Address    State       PID/Program
tcp        0      0 128.224.165.247:40322 128.224.158.134:22 ESTABLISHED 713/ssh
\end{lstlisting}
Auto load module {\tt fuse}:
\begin{lstlisting}
root@localhost:/root> lsmod | egrep "(fuse|Module)"
Module                  Size  Used by
fuse                   53276  2 
\end{lstlisting}
{\tt fusermount} or {\tt umount} to umount:
\begin{lstlisting}
root@localhost:/root> fusermount -u /mnt/
root@localhost:/root> umount /mnt/
root@localhost:/root> ll /mnt/
total 0
\end{lstlisting}
Check kernel modules again, used by 0:
\begin{lstlisting}
root@localhost:/root> lsmod | egrep "(fuse|Module)"
Module                  Size  Used by
fuse                   53276  0 
\end{lstlisting}\null\\
\subsection{Make new file on the board \textcolor{green}{[pass]}}
\begin{lstlisting}
root@localhost:/root> echo "sshfs test" > /mnt/test.txt
root@localhost:/root> cat !$
cat /mnt/test.txt
sshfs test
\end{lstlisting}
On the laptop to check again, works great:
\begin{lstlisting}
% ll | grep test                       
781928 -rw-r--r--   1 hask hask   11 Feb  1 15:10 test.txt
% cat test.txt                        
sshfs test
\end{lstlisting}\null\\
\subsection{Make new file on the laptop \textcolor{green}{[pass]}}
The laptop:
\begin{lstlisting}
% echo "sshfs is great" > test.txt
\end{lstlisting}
The board:
\begin{lstlisting}
root@localhost:/mnt> cat test.txt 
sshfs is great
\end{lstlisting}\null\\
\subsection{Rename file on the board \textcolor{green}{[pass]}}
\begin{lstlisting}
root@localhost:/mnt> mv test.txt test_rename.txt
root@localhost:/mnt> cat !$
cat test_rename.txt
sshfs is great
root@localhost:/mnt> 
\end{lstlisting}
\begin{lstlisting}
% cat test_rename.txt                   
sshfs is great
\end{lstlisting}\null\\
\subsection{Rename file on the laptop \textcolor{green}{[pass]}}
\begin{lstlisting}
% mv test_rename.txt test_rename_again.txt
% cat !$                                 
cat test_rename_again.txt
sshfs is great
\end{lstlisting}
\begin{lstlisting}
root@localhost:/mnt> cat test_rename_again.txt 
sshfs is great
\end{lstlisting}\null\\
\subsection{Change file permissions on the board \textcolor{green}{[pass]}}
Set to be only executable:
\begin{lstlisting}
root@localhost:/mnt> ll test_rename_again.txt 
-rw-r--r-- 1 1002 1002 15 Feb 10 14:07 test_rename_again.txt
root@localhost:/mnt> chmod 111 test_rename_again.txt 
root@localhost:/mnt> ll test_rename_again.txt 
---x--x--x 1 1002 1002 15 Feb 10 14:07 test_rename_again.txt
root@localhost:/mnt> cat !$
cat test_rename_again.txt
cat: test_rename_again.txt: Permission denied
root@localhost:/mnt> echo "can't write" >> test_rename_again.txt 
-bash: test_rename_again.txt: Permission denied
\end{lstlisting}
\begin{lstlisting}
% ll test_rename_again.txt 
781886 ---x--x--x 1 hask hask 15 Feb 10 22:07 test_rename_again.txt
% cat test_rename_again.txt    
cat: test_rename_again.txt: Permission denied
(1)% echo "can't write" >> test_rename_again.txt
zsh: permission denied: test_rename_again.txt
\end{lstlisting}\null\\
\subsection{Change file permissions on the laptop \textcolor{green}{[pass]}}
Set to be only readable:
\begin{lstlisting}
% chmod 444 test_rename_again.txt
% ll test_rename_again.txt      
781886 -r--r--r-- 1 hask hask 15 Feb 10 22:07 test_rename_again.txt
% cat test_rename_again.txt 
sshfs is great
% echo "can't write" >> test_rename_again.txt
zsh: permission denied: test_rename_again.txt
\end{lstlisting}
\begin{lstlisting}
root@localhost:/mnt> ll test_rename_again.txt 
-r--r--r-- 1 1002 1002 15 Feb 10 14:07 test_rename_again.txt
root@localhost:/mnt> cat test_rename_again.txt 
sshfs is great
root@localhost:/mnt> echo "can't write" >> test_rename_again.txt 
-bash: test_rename_again.txt: Permission denied
\end{lstlisting}\null\\
\subsection{Symbolic link and the original file both in the mounted dir 
            \textcolor{green}{[pass]}}
\begin{lstlisting}
root@localhost:/mnt> ln -s test.txt test.bak
root@localhost:/mnt> rm test.txt 
root@localhost:/mnt> echo "sshfs symbolic" > test.txt
root@localhost:/mnt> cat test.bak 
sshfs symbolic
root@localhost:/mnt> 
\end{lstlisting}\null\\
\subsection{Symbolic link in the mounted dir to a file in the home dir
            \textcolor{green}{[pass]}}
\begin{lstlisting}
root@localhost:/mnt> echo "sshfs symbolic" > /root/test.txt 
root@localhost:/mnt> ln -s !$ /mnt/test.bak
ln -s /root/test.txt /mnt/test.bak
root@localhost:/mnt> cat test.bak 
sshfs symbolic
\end{lstlisting}
On the laptop:
\begin{lstlisting}
% ll | grep test.bak
781902 lrwxrwxrwx   1 hask hask   14 Feb  1 16:07 test.bak -> /root/test.txt
\end{lstlisting}
Shows that symbolic link depends on the file name.\\
And a stupid try:
\begin{lstlisting}
% cat test.bak          
cat: test.bak: Permission denied
% sudo cat test.bak
cat: test.bak: No such file or directory
\end{lstlisting}
Umount then mount then cat also works.
\begin{lstlisting}
root@localhost:/mnt> cd 
root@localhost:/root> umount /mnt/
root@localhost:/root> sshfs hask@128.224.158.134:/home/hask /mnt
hask@128.224.158.134's password: 
root@localhost:/root> cd /mnt/
root@localhost:/mnt> cat test.bak 
sshfs symbolic
root@localhost:/mnt> cat /root/test.txt 
sshfs symbolic
\end{lstlisting}
Make a new user {\tt windriver} and {\tt chmod} to him:
\begin{lstlisting}
root@localhost:/root> adduser windriver
root@localhost:/root> passwd !$
passwd windriver
New password: 
Retype new password: 
BAD PASSWORD: it is WAY too short
BAD PASSWORD: is a palindrome
passwd: password updated successfully
root@localhost:/root> cat /etc/passwd | grep windriver
windriver:x:500:500::/home/windriver:/bin/bash
root@localhost:/root> cd /mnt
root@localhost:/mnt> ll test.bak
lrwxrwxrwx 1 1002 1002 14 Feb 11  2011 test.bak -> /root/test.txt
root@localhost:/mnt> chown windriver.windriver test.bak 
root@localhost:/mnt> ll /root/ | grep windriver
-rw-r--r-- 1 windriver windriver     6 Feb 11  2011 test.txt
\end{lstlisting}
The original file is {\tt /root/test.txt}, on the board's filesystem, 
so can change it's owner.\\\\
\subsection{Symbolic link in the home dir to a file in the mounted dir
            \textcolor{green}{[pass]}}
\begin{lstlisting}
root@localhost:/mnt> echo "sshfs symbolic" > test.txt
root@localhost:/mnt> ln -s /mnt/test.txt /root/test.bak 
root@localhost:/mnt> ll /root/ | grep test
lrwxrwxrwx 1 root root    13 Feb 11  2011 test.bak -> /mnt/test.txt
root@localhost:/mnt> cat /root/test.bak 
sshfs symbolic
\end{lstlisting}
Umount then mount then cat also works:
\begin{lstlisting}
root@localhost:/mnt> cd 
root@localhost:/root> fusermount -u /mnt/
root@localhost:/root> cat test.bak 
cat: test.bak: No such file or directory
root@localhost:/root> sshfs hask@128.224.158.134:/home/hask /mnt
hask@128.224.158.134's password: 
root@localhost:/root> cd /mnt/
root@localhost:/mnt> cat /root/test.bak 
sshfs symbolic
\end{lstlisting}
{\tt chmod} to user {\tt windriver}:
\begin{lstlisting}
root@localhost:/mnt> ll | grep test
-rw-r--r-- 1 1002 1002       15 Feb 11  2011 test.txt
root@localhost:/mnt> chown windriver.windriver test.txt 
chown: changing ownership of `test.txt': Permission denied
root@localhost:/mnt> chown windriver.windriver /root/test.bak 
chown: changing ownership of `/root/test.bak': Permission denied
\end{lstlisting}
It's right.\\\\
\subsection{Hard link \textcolor{red}{[not support yet]}}
\begin{lstlisting}
root@localhost:/mnt> ln test.txt test.bak 
ln: creating hard link `test.bak' => `test.txt': Function not implemented
\end{lstlisting}\null\\
\subsection{Mount root directory of the laptop \textcolor{green}{[pass]}}
\begin{lstlisting}
root@localhost:/root> sshfs root@128.224.158.134:/ /mnt             
root@128.224.158.134's password: 
root@localhost:/root> cd /mnt/
root@localhost:/root> ll
total 96
drwxr-xr-x 1 root root  4096 Jan 31  2011 bin
drwxr-xr-x 1 root root  1024 Jan 11  2011 boot
drwxr-xr-x 1 root root  5700 Feb  1  2011 dev
drwxr-xr-x 1 root root  4096 Feb  1  2011 etc
drwxr-xr-x 1 root root  4096 Jan 17  2011 folk
drwxr-xr-x 1 root root  4096 Jan 17  2011 home
drwxr-xr-x 1 root root  4096 Jan 31  2011 lib
drwx------ 1 root root 16384 May 17  2010 lost+found
drwxr-xr-x 1 root root  4096 Feb  1  2011 media
drwxrwxrwx 1 root root  4096 Jan 20  2011 mnt
drwxr-xr-x 1 root root  4096 Jan 23  2011 opt
dr-xr-xr-x 1 root root     0 Feb  1  2011 proc
drwxr-x--- 1 root root  4096 Feb  1  2011 root
drwxr-xr-x 1 root root  4096 Jan 31  2011 sbin
drwxr-xr-x 1 root root  4096 Dec 14  2010 srv
drwxr-xr-x 1 root root     0 Feb  1  2011 sys
drwxrwxrwt 1 root root 12288 Feb  1  2011 tmp
drwxr-xr-x 1 root root  4096 Dec 15  2010 usr
drwxr-xr-x 1 root root  4096 Dec 18  2010 var
drwxr-xr-x 1 root root  4096 Jan 23  2011 wr
root@localhost:/mnt> echo "i want to rm -rf *" > root.txt
root@localhost:/mnt> ll | grep txt
total 104
-rw-r--r-- 1 root root    19 Feb  1  2011 root.txt
root@localhost:/mnt> cat root.txt 
i want to rm -rf *
\end{lstlisting}
And a dangerous try:
\begin{lstlisting}
root@localhost:/mnt> mv etc/ cte
% ll
total 85K
   8185 drwxr-xr-x   2 0 0 4.0K Jan 31 13:50 bin
      2 drwxr-xr-x   4 0 0 1.0K Jan 11 08:07 boot
   8181 drwxr-xr-x  83 0 0 4.0K Feb  1 07:53 cte
      4 drwxr-xr-x  19 0 0 5.6K Feb  1 07:52 dev
1578204 drwxr-xr-x   2 0 0 4.0K Jan 17 06:01 folk
      2 drwxr-xr-x   4 0 0 4.0K Jan 17 06:01 home
   8183 drwxr-xr-x  10 0 0 4.0K Jan 31 13:50 lib
     11 drwx------   2 0 0  16K May 17  2010 lost+found
   8202 drwxr-xr-x   4 0 0 4.0K Feb  1 01:20 media
   8206 drwxrwxrwx   4 0 0 4.0K Jan 20 09:38 mnt
   8208 drwxr-xr-x   8 0 0 4.0K Jan 23 07:41 opt
      1 dr-xr-xr-x 168 0 0    0 Feb  1 07:52 proc
   8207 drwxr-x---  11 0 0 4.0K Feb  1 08:04 root
  20051 -rw-r--r--   1 0 0   19 Feb  1 08:19 root.txt
   8184 drwxr-xr-x   2 0 0 4.0K Jan 31 13:53 sbin
2097269 drwxr-xr-x   4 0 0 4.0K Dec 14 07:09 srv
      1 drwxr-xr-x  12 0 0    0 Feb  1 07:52 sys
   8180 drwxrwxrwt  11 0 0  12K Feb  1 08:33 tmp
 539619 drwxr-xr-x  12 0 0 4.0K Dec 15 02:14 usr
  16353 drwxr-xr-x  14 0 0 4.0K Dec 18 08:34 var
\end{lstlisting}
Then on the laptop try to modify it as user {\tt hask} (uid 1002):
\begin{lstlisting}
% sudo mv cte etc
sudo: unknown uid: 1002
% su -
su: user root does not exist
\end{lstlisting}
On the board change it back:
\begin{lstlisting}
root@localhost:/mnt> mv cte/ etc
\end{lstlisting}
On the laptop to check:
\begin{lstlisting}
% ll /
total 85K
   8185 drwxr-xr-x   2 root root 4.0K Feb 10 21:20 bin
      2 drwxr-xr-x   4 root root 1.0K Feb 10 21:18 boot
      3 drwxr-xr-x  19 root root 5.6K Feb 11 09:29 dev
   8181 drwxr-xr-x  84 root root 4.0K Feb 11 09:31 etc
1578204 drwxr-xr-x   2 root root 4.0K Jan 17 14:01 folk
      2 drwxr-xr-x   4 root root 4.0K Jan 17 14:01 home
   8183 drwxr-xr-x  10 root root 4.0K Feb 10 21:20 lib
     11 drwx------   2 root root  16K May 17  2010 lost+found
   8202 drwxr-xr-x   4 root root 4.0K Feb  8 20:03 media
   8206 drwxrwxrwx   4 root root 4.0K Jan 20 17:38 mnt
   8208 drwxr-xr-x   8 root root 4.0K Jan 23 15:41 opt
      1 dr-xr-xr-x 171 root root    0 Feb 11 09:29 proc
   8207 drwxr-x---  11 root root 4.0K Feb 10 17:26 root
  20051 -rw-r--r--   1 root root   19 Feb  1 16:19 root.txt
   8184 drwxr-xr-x   2 root root 4.0K Feb 10 21:20 sbin
2097269 drwxr-xr-x   4 root root 4.0K Dec 14 15:09 srv
      1 drwxr-xr-x  12 root root    0 Feb 11 09:29 sys
   8180 drwxrwxrwt  12 root root  12K Feb 11 10:49 tmp
 539619 drwxr-xr-x  12 root root 4.0K Dec 15 10:14 usr
  16353 drwxr-xr-x  14 root root 4.0K Dec 18 16:34 var
\end{lstlisting}
%\chapter{sshfs}


\chapter{uclibc++-0.2.2}
\section{Description}
Is a {\tt C++} standard library targeted towards the embedded systems/software market. 
As such it may purposefully lack features which you might normally expect to find in 
a full fledged {\tt C++} standard library. The library will focus on space savings as 
opposed to performance.
\section{Installation Hierarchy}
\begin{lstlisting}
/lib/libuClibc++-0.2.2.so
/lib/libuClibc++.so.0
/lib/libuClibc++-so
\end{lstlisting}
\section{Warning}
\begin{itemize}
    \item Run {\tt menuconfig} first, the default will be good enough for now;
    \item If you are looking to statically compile an application, you must compile it
	  without the use of either rtti or exceptions.  The library should be compiled
	  likewise.  Since the application will be statically compiled, you may run a 
	  dynamic library with exception support;
    \item As of version 0.1.11 uClibc+ throws LOTS of warnings when compiling about inline
	  functions used but never defined.  There warnings are known errors in GCC
	  versions 3.3.3 3.4.0 4.0.0 4.1.0 [Bug #21627].  Either ignore these warnings or
	  upgrade your compiler.  Note that these warnings will exist when compiling software
	  as well as the library itself;
    \item uClibc++ is not capable of bootstrapping itself.  The library depends upon some
	  of the gcc support files and libstdc++ for exception support.  Linking in libsupc++
	  moves all of the required code into the library so that the GNU library is no
	  longer required AFTER COMPILATION.  Thus you can build uClibc++ on a development
	  system against libstdc++, but when deployed you will not need any of the GNU library
	  files. It will likely be in a location like: 
          {\tt /usr/lib/gcc-lib/i686-pc-linux-gnu/3.3.4/libsupc++.a};
    \item {\tt libsupc++.a} is in {\tt host-cross/toolchain/x86-linux2/arm-wrs-linux-gnueabi/lib/},
\begin{lstlisting}
libsupc++.a
softfp/libsupc++.a
thumb2-v7-a/libsupc++.a
thumb2-v7-a-neon/libsupc++.a
uclibc/softfp/libsupc++.a
uclibc/thumb2-v7-a/libsupc++.a
uclibc/thumb2-v7-a-neon/libsupc++.a
uclibc/libsupc++.a
\end{lstlisting}
          All symbolic links, and real ones have the same {\tt *.o} files as contents,
          however with various size;
\end{itemize}
\section{Test}
{\tt tests/*.c} are test cases, while {\tt tests/testoutput} has {\tt runtest.sh} and all 
corresponding correct outputs.\\
{\tt make menuconfig && make && make test} automatically compile all test cases then test
them, something like:
\begin{lstlisting}
algotest                 OK
bitsettest               OK
chartraitstest           OK
combotest                OK
dequetest                OK
excepttest               OK
fstreamtest              OK
functionaltest           OK
ioexceptiontest          OK
iotest                   OK
listtest                 OK
maptest                  OK
memorytest               OK
mmaptest                 OK
newdeltest               OK
settest                  OK
sstreamtest              OK
stacktest                OK
streambuftest            OK
stringtest               OK
utilitytest              OK
valarraytest             OK
vectortest               OK
wchartest                missing/not built
\end{lstlisting}
The newset version doesn't support wchar set, see:
\\[\intextsep]
\begin{minipage}{\textwidth}
\centering
\includegraphics[scale=.45]{uclibcpp/wchar.png}
\end{minipage}
\\[\intextsep]
%\chapter{uclibc++-0.2.2}


\end{document}
